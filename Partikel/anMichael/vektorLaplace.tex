\documentclass{scrartcl}

\usepackage[ngerman]{babel}
\usepackage[utf8]{inputenc}


\usepackage{amsmath}
\usepackage{amssymb}

\usepackage{dsfont}
\newcommand{\R}{\mathds{R}}
\newcommand{\Z}{\mathds{Z}}
\newcommand{\csd}{\text{csd}}
\renewcommand{\div}{\text{Div}}
\renewcommand{\hom}{\text{Hom}}
\newcommand{\err}{\text{Err}}
\newcommand{\id}{\text{Id}}
\newcommand{\D}{\text{D}}
\renewcommand{\d}{\mathrm{d}}
\newcommand{\exd}{\mathbf{d}}
\newcommand{\argmin}{\operatornamewithlimits{argmin}}
\newcommand{\sgn}{\mathop{\mathrm{sgn}}\nolimits}
\newcommand{\formpunkt}{\,\text{.}}
\newcommand{\formkomma}{\,\text{,}}
\newcommand{\formtext}[1]{\quad\text{#1}\quad}
\newcommand{\eps}{\varepsilon}
\newcommand{\vecflat}[1]{\vec{#1}^{\,\flat}}
\newcommand{\vecover}[2]{\vec{#1}^{\,#2}}
\newcommand{\diag}[1]{\text{diag}\left( #1 \right)}
\newcommand{\II}{I \! I}
\newcommand{\av}{\text{Av}}
\newcommand{\conn}{\text{Conn}}


\begin{document}
  Allgemein wäre der Laplace (de-Rham) für ein Vektorfeld \( \vec{p}\in TM \) für eine zweidimensionale Oberfläche
  \begin{align}
    \Delta_{dR} \vec{p} &:= -\left( \left( \delta\exd + \exd\delta \right)\vecflat{p}\right)^{\sharp} \in TM
  \end{align}
  mit \( \delta = - * \exd * \).
  \( \flat \) und \( \sharp \) sind hierbei eingeführt um alles kontravariant zu lassen, also nicht im Dualraum des Tangentialraumes zu rechnen.
  Ein wenig intuitiver ist es vielleicht nur den ersten Anteil, wie bei dem Laplace-Beltrami für Funktionen, zunutzen um so einen allgemeineren Laplace-Beltrami für Vectorfelder zu bekommen
  \begin{align}
    \Delta_{B} \vec{p}:= -\left( \delta\exd \vecflat{p}\right)^{\sharp} \in TM
  \end{align}
  Ich habe ihn mal für die Sphäre explizit ausgerechnet.
  Weiter unten ist die genutze Parametrisierung angegeben.
  Es ergibt sich somit für
  \begin{align}
    \vec{p} = p^{1}\partial_{u}\vec{x} + p^{2}\partial_{v}\vec{x} \left( + 0 \cdot \partial_{\nu}\vec{x} \right)
  \end{align}
  dass
  \begin{align}
    \Delta^{ \mathds{S}^{2}}_{B}\vec{p} &= \left( -2\cos u \partial_{v} p^{2} + \frac{1}{\sin u} \partial_{v}^{2} p^{1} - \sin u \partial_{uv} p^{2}\right) \partial_{u}\vec{x} \\
                   &\quad + \left( -2p^{2} + \frac{\cot u}{\sin^{2}u} \partial_{v}p^{1} + 3\cot u \partial_{u}p^{2}  
                              -\frac{1}{\sin^{2}u}\partial_{uv}p^{1} + \partial_{u}^{2} p^{2}\right) \partial_{v}\vec{x}
  \end{align}
  Die Umwandlung in Standard-\( \R^{3} \)-Koordinaten erspare ich dir und mir mal, das würde eine riesige Formel werden.
  Wir können aber ein einfaches Bsp. rechnen.
  Wählen wir \( p^{1} = 0 \) und \( p^{2} = \frac{1}{\sin u} \), also
  \begin{align}
    \vec{p} &= \frac{1}{\sin u} \partial_{v}\vec{x} 
          =\begin{bmatrix}
            -\sin v \\ \cos v \\ 0
           \end{bmatrix}
          = \frac{1}{\sqrt{1-z^{2}}}
            \begin{bmatrix}
            -y \\ x \\ 0
           \end{bmatrix}
  \end{align}
  dann gilt \( \|\vec{p}\| = 1 \) und der Laplace-Beltrami ist
  \begin{align}
    \Delta^{ \mathds{S}^{2}}_{B}\vec{p} &= -\frac{1}{\sin^{2}u} \vec{p} =
        -\frac{1}{\sqrt{\left( 1-z^{2} \right)^{3}}}
            \begin{bmatrix}
            -y \\ x \\ 0
           \end{bmatrix}
  \end{align}
  vlt. kannst du das als Testproblem zum vergleichen nutzen mit deinem laplace mit der Einschränkung auf die Oberfläche.

  Da sich der Laplace-Beltrami auch als Rot\( \circ \)rot schreiben lässt, liegt die Vermutung nahe, dass er sich auch allgemein für beliebige Oberflächen in \( \R^{3} \)-Koordinaten durch
  \begin{align}
    \Delta_{B}\vec{p} &= \vec{\nu} \times \nabla \left( \nabla \cdot \left( \vec{\nu} \times \vec{p} \right) \right) \in TM
  \end{align}
  schreiben lässt
  mit Normalenfeld \( \vec{\nu} \) und dem gewöhnlichen \( \R^{3} \)-Gradienten, -Divergenz bzw. -Kreuzprodukt.
  Dafür lege ich aber nicht meine Hand ins Feuer. Hier müsste noch ein Beweis geführt werden.

  Auch gebe ich keine Garantie für alle hier verwendeten Vorzeichen! :) 



  \subsection{Einheitssphäre}
    \label{sphere}

    \subsubsection{Parametrisierung}
      \begin{align}
      \begin{aligned}
        \vec{x}: \left( 0, \pi \right) \times \left[ 0 , 2\pi \right)
                    &\rightarrow \mathds{S}^{2} \subset \R^{3} \\
             \left( u,v \right) 
                    &\mapsto\begin{bmatrix}
                              x(u,v) \\ y(u,v) \\ z(u,v)
                            \end{bmatrix}
                    := \begin{bmatrix}
                        \sin u \cos v \\
                        \sin u \sin v \\
                        \cos u
                      \end{bmatrix}
      \end{aligned}
      \end{align}
      \( u \) heißt Breitengrad und \( v \) Längengrad.

    \subsubsection{Riemannsche Metrik}
      \begin{align}
        \partial_{u}\vec{x}
              &=\begin{bmatrix}
                  \cos u \cos v \\
                  \cos u \sin v \\
                  -\sin u
                \end{bmatrix}
              \quad\bot_{\R^{3}}\quad
              \partial_{v}\vec{x}
              =\begin{bmatrix}
                        -\sin u \sin v \\
                        \sin u \cos v \\
                        0
                \end{bmatrix}
      \end{align}
      \begin{align}
        \Rightarrow g = \begin{bmatrix}
              1 & 0 \\ 0 & \sin^{2} u
            \end{bmatrix}
          =: \diag{g_{u}, g_{v}}
      \end{align}
  
\end{document}
