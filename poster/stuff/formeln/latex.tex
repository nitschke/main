\documentclass{scrartcl}

\usepackage[english]{babel}

\usepackage[utf8]{inputenc}

\usepackage{color}


\usepackage{cite}

\usepackage{nicefrac}

\usepackage{amsmath}
%\allowdisplaybreaks
\usepackage{amssymb}


\usepackage[matrix,arrow]{xy}

\usepackage{tikz}
\usetikzlibrary{calc}

\usepackage{dsfont}
\newcommand{\R}{\mathds{R}}
\newcommand{\Z}{\mathds{Z}}
\newcommand{\csd}{\text{csd}}
\renewcommand{\div}{\text{Div}}
\renewcommand{\hom}{\text{Hom}}
\newcommand{\err}{\text{Err}}
\newcommand{\id}{\text{Id}}
\newcommand{\D}{\text{D}}
\renewcommand{\d}{\mathrm{d}}
\newcommand{\exd}{\mathbf{d}}
\newcommand{\argmin}{\operatornamewithlimits{argmin}}
\newcommand{\sgn}{\mathop{\mathrm{sgn}}\nolimits}
\newcommand{\formpunkt}{\,\text{.}}
\newcommand{\formkomma}{\,\text{,}}
\newcommand{\formtext}[1]{\quad\text{#1}\quad}
\newcommand{\eps}{\varepsilon}
\newcommand{\vecflat}[1]{\vec{#1}^{\,\flat}}
\newcommand{\vecover}[2]{\vec{#1}^{\,#2}}
\newcommand{\diag}[1]{\text{diag}\left( #1 \right)}
\newcommand{\II}{I \! I}
\newcommand{\av}{\text{Av}}
\newcommand{\conn}{\text{Conn}}


\usepackage{siunitx}

\renewcommand{\familydefault}{\sfdefault}
\setlength{\parindent}{0pt} 


\begin{document}
\pagestyle{empty}
\section*{Discrete Exterior Calculus (DEC) \\approximation of curvature on surfaces}
\pagebreak
\section*{Curvature vector}
\subsection*{Continuous Problem}
\begin{itemize}
  \item Inclusion map: \( \iota: \R^{3}|_{M} \hookrightarrow \R^{3},\quad \vec{x} \mapsto \vec{x}  \)
  \item Laplace-Beltrami-Operator for the inclusion map on a given manifold (componentwise)
      \begin{align*}
      \Delta_{B} \iota &= \left(* \exd * \exd  \right)  \iota
        = \frac{1}{\sqrt{\left| \det g \right|}} \sum_{i,j=1}^{2} \frac{\partial}{\partial x^{j}} \left( g^{ij}\sqrt{\left| \det g \right|} \frac{\partial\iota}{\partial x^{i}}
      \right)
      \end{align*}
  (\( g, g^{ij} \): metric tensor (e.g. Riemannian metric) resp. its inverse components)
  \item Curvature Vector, see \cite{flanders}:  \( \vec{H} = -\Delta_{B}\iota \)
  \item Mean curvature: \( H = \frac{1}{2}\left\| \vec{H} \right\| \)
\end{itemize}

\subsection*{Discrete Problem}
\begin{itemize}
  \item For a better FEM-like elementwise implementation, the discrete formulation on a vertex \( v_{i} \) is given with respect to the Hodge-/Geometric-Star-Operator:
    \begin{align*}
              \left\langle *\Delta_{B} \iota^{k} , \star v_{i} \right\rangle
                     &= \sum_{\sigma^{1}=\left[ v_{i}, v_{j} \right]} 
                     \frac{\left| \star\sigma^{1} \right|}{\left| \sigma^{1} \right|}
                      \left( \iota^{k}(v_{j}) - \iota^{k}(v_{i}) \right)\formkomma
    \end{align*}
    (\( \iota = \left[ \iota^{1}, \iota^{2}, \iota^{3} \right] \) and the global vertex indices \( i \) and \( j \))
  \item DEC-approximated mean curvature:
    \begin{align*}
  H_{d}(v_{i}) = \frac{1}{2\left| \star v_{i} \right|} \sqrt{\sum_{k=1}^{3} \left\langle *\Delta_{B} \iota^{k} , \star v_{i} \right\rangle^{2}} \formpunkt
    \end{align*}
\end{itemize}

\pagebreak
\section*{Weingarten map}
\subsection*{Continuous problem}
\begin{itemize}
  \item Extended Weingarten map: \(\bar{S}:= \nabla\vec{\nu} \in \R^{3 \times 3}: M \rightarrow \R^{3 \times 3}\)\\
      (\( \nabla \): surface gradient)
  \item The restriction of the extended  Weingarten map to the tangential space is the usual Weingarten map \( S \).
  \item The eigenvalues of \( S \) are the principal curvatures \( \kappa^{1} \) and \( \kappa^{2} \) of the Surface \( M \).
        The mean curvature and the Gaussian curvature is given by  \(H = \frac{\kappa^{1} + \kappa^{2}}{2}  \) resp. \( K = \kappa^{1} \cdot \kappa^{2} \).
\end{itemize}

\subsection*{Discrete problem}
\begin{itemize}
  \item Discrete surface normals \( \vec{\nu} \) on a vertex \( v \):
    \begin{itemize}
      \item Average of element normals \( \vecover{\nu}{\sigma^{2}} \): \hfill
        \( \vecover{\nu}{\av}(v) := \frac{1}{\left| \star v \right|} \sum_{\sigma^{2}\succ v} \left| \star v \cap \sigma^{2}\right| 
                                          \vecover{\nu}{\sigma^{2}} \)
      \item From a signed distance function \( \varphi:\R^{3} \rightarrow \R \): \hfill
      \(\vec{\nu}(v) = \frac{ \nabla_{\R^{3}}\varphi}{\left\| \nabla_{\R^{3}}\varphi \right\|}  \)
    \end{itemize}
 \item Discrete surface Gradient \( \nabla^{\overline{pd}} \) as average of the primal-dual-gradient \( \nabla^{pd} \), see \cite{hirani}:
 \begin{align*}
   \left( \nabla^{\overline{pd}} f\right)(v) 
        %&=  \frac{1}{\left| \star v \right|} \sum_{\sigma^{2}\succ v} \left| \star v \cap \sigma^{2} \right| 
        %          \left\langle \nabla^{pd} f , \star\sigma^{2} \right\rangle
        &= \frac{1}{\left| \star v \right|} \sum_{\sigma^{2}\succ v} \left| \star v \cap \sigma^{2} \right|
                 \sum_{\sigma^{0}\prec\sigma^{2}} \left( f(\sigma^{0}) - f(v) \right) \nabla\Phi_{\sigma^{0}}^{\sigma^{2}}
 \end{align*}
 (\(\nabla\Phi_{\sigma^{0}}^{\sigma^{2}}\): gradient of the linear basis function \( \Phi_{\sigma^{0}} \) on element \( \sigma^{2} \))

 \item Discrete formulation on a vertex \( v \) and for components with index \( i,j\in\left\{ 1,2,3 \right\} \):
 \begin{align*}
    \left| \star v \right| \bar{S}_{ij}(v) \approx 
           \left\langle *\left[ S^{\overline{pd}} \right]_{ij} , \star v \right\rangle 
    &:= \left\langle *\left[ \nabla^{\overline{pd}}\bar{\nu}^{i} \right]_{j} , \star v \right\rangle
 \end{align*}
 (\( \bar{\nu}^{i} \): \( i \)-th component of \(\vec{\nu} \) resp. \( \vecover{\nu}{\av} \))
 \item Calculation of the eigenvalues of DEC-approximated extended Weingarten map \( S^{\overline{pd}} \) on every vertex with QR-Algorithm and cancel out the additional (approx. 0) eigenvalue
\end{itemize}

\pagebreak
\bibliographystyle{alpha}
\bibliography{../bibl.bib}{}
\end{document}
