\chapter{Diskretes Äußeres Kalkül (DEC)}
\label{chapterDEC}

see \cite{Lee} \cite{FirstCourse}

\section{Diskrete Differentialformen}
  
  \begin{definition}
    Eine diskrete \( p \)-Form ist ein Homomorphismus vom Kettenkomplex \( C_{p}(K) \) nach \( \R \).
    Die Menge aller dieser Homomorphismen bezeichnen wir je nach Kontext mit \( C^{p}(K) \) (Menge der \( p \)-Koketten)
    oder \( \Omega^{p}_{d}(K) \) (Menge der diskrete \( p- \)(Differential-)Formen). 
    Das heißt
    \begin{align}
      \hom(C_{p}(K),\R) =: C^{p}(K) =: \Omega^{p}_{d}(K) \formpunkt
    \end{align}
    Desweiteren erfolgt die Addition punktweise, das heißt
    \begin{align}
      \left( \alpha + \beta \right)(c) &:= \alpha(c) + \beta(c)
    \end{align}
    für \( \alpha,\beta \in C^{p}(K) \) und \( c \in C_{p}(K) \).
  \end{definition}
  
  \begin{folgerung}
    Da \( \R \) mit der Addition eine abelsche Gruppe ist, können wir uns wieder die Universalitätseigenschaft \eqref{diagFreieAbelscheGruppe} des Kettenkomplexes zunutze machen.
    Für eine \( p \)-Kette \( c = \sum_{\sigma\in K^{(p)}} a_{\sigma} \sigma \in C_{p}(K)\) und eine  \( p \)-Kokette \( \alpha\in C^{p}(K) \) gilt
    \begin{align}
      \alpha(c) &= \alpha\left(\sum_{\sigma\in K^{(p)}} a_{\sigma} \sigma\right) = \sum_{\sigma\in K^{(p)}} a_{\sigma} \alpha(\sigma) \formkomma
    \end{align}
    damit reicht es auch hier aus die \( p \)-Koketten nur auf den \( p \)-Simplizes zu definieren.
  \end{folgerung}

  Hätten wir die Menge der \( p \)-Ketten als \( \R \)-Vektoraum eingeführt, so hätte uns die Frage nach einem inneren Produkt zwischen den Ketten und den Koketten zur dualen Paarung
  geführt und damit auch, dass \( C^{p}(K) = \left( C_{p}(K) \right)^{*}\) der Dualraum von \( C_{p}(K) \) ist.
  Nun hält uns aber auch nichts davon ab, dies auch für die hier eingeführten Ketten analog zu machen.

  \begin{definition}
    \begin{align}
      \begin{aligned}
        \left\langle \bullet,\bullet \right\rangle : C^{p}(K) \times C_{p}(K)  &\rightarrow \R \\
                                                     \left( \alpha, c  \right) &\mapsto \left\langle \alpha, c \right\rangle := \alpha(c)
      \end{aligned}
    \end{align}
    heißt natürliche Paarung zwischen den \( p \)-Koketten (-Formen) und den \( p \)-Ketten.
  \end{definition}


  
  

\section{Äußere Ableitung}
  

  

  

\section{Hodge-Operator}

\section{Laplace-Operator}

  \subsection{Beispiel: Laplace-Gleichung}

  \subsection{Beispiel: Krümmung Teil 1: Gauß-Bonnet-Operator}
  \subsection{Beispiel: Krümmung Teil 2: Weingarten-Abbildung}
  \subsection{Beispiel: Krümmung Teil 3: Krümmungsvektor}


\section{Lie-Ableitung und Jacobian}

  \subsection{Beispiel: Wirbelgleichung}


