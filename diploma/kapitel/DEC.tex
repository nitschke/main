\chapter{Diskretes Äußeres Kalkül (DEC)}
\label{chapterDEC}

see \cite{Lee} \cite{FirstCourse}

\section{Diskrete Differentialformen}
  
  \begin{definition}
    Eine diskrete \( p \)-Form ist ein Homomorphismus vom Kettenkomplex \( C_{p}(K) \) nach \( \R \).
    Die Menge aller dieser Homomorphismen bezeichnen wir je nach Kontext mit \( C^{p}(K) \) (Koketten)
    oder \( \Omega^{p}_{d}(K) \) (diskrete Differentialformen). 
    Das heißt
    \begin{align}
      \hom(C_{p}(K),\R) =: C^{p}(K) =: \Omega^{p}_{d}(K) \formpunkt
    \end{align}
  \end{definition}
  
  \begin{folgerung}
    Da \( \R \) mit der Addition eine abelsche Gruppe ist, können wir uns wieder die Universalitätseigenschaft \eqref{diagFreieAbelscheGruppe} des Kettenkomplexes zunutze machen.
    Für \( p \)-Kette \( c = \sum_{\sigma\in K^{(p)}} a_{\sigma} \sigma \in C_{p}(K)\) und diskrete \( p \)-Form \( \alpha_{d}\in\Omega^{p}_{d}(K) \) gilt
    \begin{align}
      \alpha_{d}(c) &= \alpha_{d}\left(\sum_{\sigma\in K^{(p)}} a_{\sigma} \sigma\right) = \sum_{\sigma\in K^{(p)}} a_{\sigma} \alpha_{d}(\sigma) \formkomma
    \end{align}
    damit reicht es auch hier aus die diskreten \( p \)-Formen nur auf den \( p \)-Simplizes zu definieren.
  \end{folgerung}


  
  

\section{Äußere Ableitung}
  

  

  

\section{Hodge-Operator}

\section{Laplace-Operator}

  \subsection{Beispiel: Laplace-Gleichung}

  \subsection{Beispiel: Krümmung Teil 1: Gauß-Bonnet-Operator}
  \subsection{Beispiel: Krümmung Teil 2: Weingarten-Abbildung}
  \subsection{Beispiel: Krümmung Teil 3: Krümmungsvektor}


\section{Lie-Ableitung und Jacobian}

  \subsection{Beispiel: Wirbelgleichung}


