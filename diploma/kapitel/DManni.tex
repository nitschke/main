\chapter{Diskrete Mannigfaltigkeit}

\section{Gittergenerierung}

  see \cite{meshCooper} \cite{meshHirani}

  \begin{ziel}
    Die Wohlzentriertheit eines Gitters ist Pflicht, da ohne sie kein brauchbares duales Gitter (Voronoigitter) erzeugt werden kann. 
    Diese zur Triangulierung duale Gebietsdiskretisierung wird aber benötigt um zum Beispiel ein diskreten Hodge-Stern-Operator sinnvoll zu entwickeln. 
    Bei einem nicht wohlzentrierten Dreieck liegt der Voronoiknoten \( \star\sigma^{2} \) nicht im Dreick \( \sigma^{2} \).
    Das Problem dabei ist, dass sich die Werte auf \( \star\sigma^{2} \) und \( \sigma^{2} \) nur um einen 
    metrischen Faktor\footnote{hier \( |\sigma^{2}| \) bzw. dessen Reziproke} unterscheiden sollten.
    Diese Vorraussetzung wäre aber nicht mehr haltbar, da die Gebiete, die beide Elemente einnehmen, disjunkt sind. 
    Sie können sogar \glqq sehr weit\grqq %todo  anführungsstriche ordentlich machen
    von einander entfernt liegen.
    Dann hätte die eine Größe fast nichts mehr mit der anderen gemein und die Linearität beider wäre nicht mehr gegeben.

    Wohlzentiertheit ist eine schwerwiegende Einschränkung an die Gitterstruktur. Sie verbietet unter anderem einen 1-Ring um einen Knoten aus vier oder weniger Dreickselementen.
    Für eine nicht planare Triangulierung mag ein 1-Ring aus vier Flächenelementen gerade noch funktionieren, da die Innenwinkelsumme der inneren Kanten weniger als \( 2\pi \) ist.
    Im planaren Fall erhalten wir aber für eine optimale\footnote{bzgl. der maximalen Winkel} Triangulierung Winkel von \( \frac{\pi}{2} \) 
    und somit nur Wohlzentriertheit im Limes\footnote{für planare äquidistante Gitter kann diese schwächere Restriktion dennoch sinnvoll sein, da somit bekannte Differenzenschematas entstehen können}.
  \end{ziel}
