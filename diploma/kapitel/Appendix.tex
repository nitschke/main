\chapter{Appendix}

\section{Algorithmen}
  
  
  \subsection{Element-Knotenkräfte}
    \label{AlgoForces}
    Berechnung der Knotenkräfte \texttt{Force}\( \in\R^{3}\times\R^{3} \) für ein Element:

    \begin{verbatim}
    for all v in T:
        T <-> [v,v0,v1]
        
        E0 = X(v0) - X(v)
        l0 = length(E0)
        forceLength = d * (l0/lRef - k)

        E1 = X(v1) - X(v)
        l1 = length(E1)
        forceAngle = (d - 1) * ((E0.E1) / (l0*l1) - c);

        Force(v) += project( (forceLength + forceAngle)*(E0/l0) 
                                          + forceAngle *(E1/l1) ) 
    \end{verbatim}

    Parameter \texttt{c,d,k}\( \in\R \), 
    Koordinatenabbildung \texttt{X}\(: \sigma^{0} \mapsto \vec{x}\in M \subset \R^{3} \) 
    und Tangentialprojektion \texttt{project}\(: \R^{3} \rightarrow T_{p}M \subset \R^{3} \)
    sind (approximativ oder exakt) gegeben.

    Zu Beachten ist hierbei, dass die Kantenkraft \texttt{forceLength} nur auf einem Knoten aufgetragen wird.
    Der andere Knoten der ebenfalls zu dieser Kante gehört bekommt die gleiche Kantenkraft aufdatiert, wenn die Knotenkräfte auf dem 2. Dreieckelement, das sich diese Kante teilt, berechnet
    werden.


% start merge
\section{Oberflächenbeispiele}

  \subsection{Ellipsoid}
    \label{heineC}
    
    \subsubsection{Level-Set-Funktion}
      \begin{align}
        \varphi(x,y,z) &:= \frac{1}{2}\left( (3x)^{2} + (6y)^{2} + (2z)^{2} - 9 \right) \\
        \nabla\varphi(x,y,z) &= \left[ 9x, 36y, 4z \right]^{T}
      \end{align}
%end merge
