\chapter{Anwendung: Oberflächenkrümmung}
  
  \begin{ziel}
    In vielen wissenschaftlichen Problemen kann die Krümmung
    einer Oberfläche eine entscheidene Rolle spielen. 
    Sie kann zum Beispiel ein wichtiger Bestandteil von Differentialgleichungen sein.
    So kommt die mittlere Krümmung \( H \) bei fluidmechanischen Formulierungen auf bewegliche Membranen 
    \( M_{t} \) in der
    Kontinuitätsgleichung
    \begin{align}
      \dot{\rho} + \rho\div\vec{v} - \rho v_{\vec{\nu}} H &= 0
    \end{align}
    vor (siehe \cite{desimone}).
    Ein weiteres einfaches Beispiel ist die Evolution von Oberflächen \( M_{t} \) über den mittleren
    Krümmungsfluss
    \begin{align}
      \dot{\vec{x}} = 2 H \vec{\nu} 
    \end{align}
    oder die Bestimmung des topologisch invarianten Geschlechts
    \begin{align}
      \mathfrak{g} = 1 - \frac{1}{4\pi}\int_{M}KdA
    \end{align}
    einer orientierten Mannigfaltigkeit \( M \) ohne Rand mit Gaußscher Krümmung \( K \),
    was dierekt aus dem Satz von Gauß-Bonnet 
    und den Zusammenhang für die Euler-Charakeristik \( \chi = 2-2\mathfrak{g} \) folgt.
    Unter den gewählten Voraussetzungen gibt \( \mathfrak{g} \) die Anzahl der "`Löcher"'/"`Poren"'/"`Henkel"' an.
    
    Es ließen sich noch weitere Beispiele finden in dem diverse Krümmungsgrößen für die Behandlung von
    Nöten wären.
    Oftmals liegt die Oberfläche allerdings nur als diskrete Eingangsgröße vor. 
    Somit sind viele
    geometrische Werte nicht bekannt und müssen Approximiert werden.
    Wir werden uns in diesem Kapitel speziell die Gaußsche und die mittlere Krümmung betrachten.
    Diese werden wir mit hilfe des DECs approximieren und zum einen mit den Ergebnissen von C.-J. Heine 
    \cite{heine} vergleichen und zum anderen, im Falle der Gauß-Krümmung, mit einer naiven 
    Gauß-Bonnet-Diskretisierung.
  \end{ziel}

\section{Weingartenabbildung}
\label{secWeingartenabbildung}
  Die Weingartenabbildung \( S \), auch Formoperator (Shapeoperator), ist im Prinzip die 2. Fundamentalform \( \II \) unter Beachtung der Metrik \(
 g \) (1. Fundamentalform).
  \( S \) gibt uns Informationen über die Gestalt von \( M \). 
  Sie misst in einer gewissen Weise wie sehr sich die Oberfläche von einer Ebenene unterscheidet, speziell ist \( M \) eine flache Ebene
  genau dann wenn die Weingartenabbildung identisch Null ist.
  Des Weiteren lassen sich aus der Weingartenabbildung die Hauptkrümmungen \( k_{1} \) und \( k_{2} \) herleiten und damit auch die
  Gaußkrümmung \( K = k_{1}*k_{2} \) und die mittlere Krümmung \( H = \frac{1}{2}\left( k_{1} + k_{2} \right) \).
  Die 1. Fundamentalform ist eine Größe der inneren Geometry und kann somit alleine auf der Oberfläche selbst bestimmt werden
  invariant gegenüber jeglicher Einbettungen/Parametrisierungen.
  Das ist bei der Weingartenabbildung anders, sie hängt von der äußeren Geometrie ab, d.h. die der Einbettung in einem Ambienteraum.
  In unserem Fall ist das der \( \R^{3} \) und die Weingartenabbildung lässt sich über das Normalenvektorfeld (Gaussabbildung)
  \begin{align}
    \vec{\nu} = \left[ \nu^{1}, \nu^{2}, \nu^{3} \right]^{T}: M &\rightarrow TN \subset \R^{3} \\
                         \vec{x} &\mapsto \vec{\nu}_{\vec{x}} \in T_{\vec{x}}N \bot T_{\vec{x}}M
  \end{align}
  berechnen. 
  Hierbei ist \( \vec{\nu}_{\vec{x}} \) der Normalenvektor im Punkt \( \vec{x} \) und \( T_{\vec{x}}N \) der eindimensionale Normalenraum
  der orthogonal zum Tangentialraum ist. 
  Da meistens klar ist, dass wir Punktweise arbeiten, kann auch hier die Indizierung weggelassen werden und wir schreiben nur 
  \( \vec{\nu} \) statt \( \vec{\nu}_{\vec{x}} \).
  \begin{definition}
    Die Weingartenabbildung ist in jedem Punkt der Oberfläche eine lineare Abbildung und definiert sich dort durch
    \begin{align}
      S: T_{\vec{x}}M &\rightarrow T_{\vec{x}}M \\
                    \vec{v} &\mapsto \exd\vec{\nu} \left( \vec{v} \right) \formkomma
    \end{align}
    wobei \( \exd\vec{\nu} \left( \vec{v} \right) \) der \( \R^{3} \)-Vektor der Richtungsableitungen in Richtung \( \vec{v}
    \) ist, wie sich einfach nachrechnen ließe (\( \vec{\nu} \) ist eine \( \R^{3} \)-vektorisierte 0-Form).
  \end{definition}
  Bei der Definition von \( S \) ist etwas Obacht geboten, weil sie oft auch mit negativen Vorzeichen definiert wird.
  Da wir uns aber vordergründig für dessen Eigenwerte interessieren spielt das Vorzeichen aber keine Rolle für uns.
  Anders als in \cite{heine} angegeben ist die Weingartenabbildung (\( (1,1) \)-Tensor) nicht die 2. Fundamentalform
  (\( (0,2) \)-Tensor). In der Matrixdarstellung unterscheiden sich beide in der Höhe der Indizes, das heißt
  \( S = \left(h^{i}_{\phantom{i}j}\right)\) und \( \II = \left( h_{ij} \right) \) oder anders geschrieben   \( S = (g)^{-1} \II \)
  (vgl. \cite{FirstCourse}).
  Somit sind auch die Eigenwerte beider Matrizen im Allgemeinen nicht gleich.
  Die Komponenten der Weingartenabbildung lassen sich für eine lokales orthogonales Koordinatensystem \( \left( x^{1},x^{2} \right) \)
  mit Riemannmetrik \( g=\diag{g_{1},g_{2}} \) berechnen durch
  \begin{align}
    h^{i}_{\phantom{i}j} &=  g^{i}h_{ij} = g^{i} \exd\vec{\nu}\left(\frac{\partial}{\partial x^{i}}\right) 
                                                  \cdot \frac{\partial}{\partial x^{j}} \\
                          &= \nabla_{i} \vec{\nu} \cdot \frac{\partial}{\partial x^{j}}
  \end{align}
  für \( i,j\in\left\{ 1,2 \right\} \) und \( \nabla_{i} \vec{\nu} \), die \( i \)-ten Komponente (Zeile) des Gradiente des
  Normalenvektors,
  das heißt
  \begin{align}
    \nabla\vec{\nu} &= \left( \exd \vec{\nu}\right)^{\sharp}
                     = \left( \sum_{k=1,2} \frac{\partial}{\partial x^{k}} \vec{\nu} dx^{k} \right)^{\sharp} \\
                    &= \sum_{k=1,2} g^{k}\frac{\partial}{\partial x^{k}} \vec{\nu}\frac{\partial}{\partial x^{k}}  \\
                    &=: \sum_{k=1,2} \nabla_{k}\vec{\nu}\frac{\partial}{\partial x^{k}} \formkomma
  \end{align}
  Zu beachten ist dabei, dass \( \nabla_{i} \) sich von der Richtungsableitung in der \( i \)-ten Basisrichtung um einen metrischen Faktor
  unterscheidet.

  Nun wollen wir aber nicht in lokalen sondern in globalen \( \R^{3} \)-Koordinaten rechnen in denen auch das Bild des Normalenvektorfeldes
  definiert ist. 
  Dazu definieren wir die erweiterte Weingartenabbildung in diesen Koordinaten.
  \begin{definition}
    Die erweiterte Weingartenabbildung \( \bar{S} \) sei an jedem Punkt der Oberfläche gegeben durch
    \begin{align}
      \bar{S}:= \nabla\vec{\nu} \in \R^{3 \times 3} \formpunkt
    \end{align}
    Dabei sei das Ableiten in Normalenrichtung zulässig ergibt jedoch den Wert Null (konstante Erweiterung in Normalenrichtung).
  \end{definition}
  Die erweiterte Weingartenabbildung eingeschränkt auf \( T_{\vec{x}}M \)\ in lokalen Koordinaten ist dann gerade \( S \),
  \todo{genauer}
  wobei die Eigenwerten von \( \bar{S} \) gleich \( \left\{ k_{1},k_{2},0 \right\} \) sind.
  Einen Beweis dazu findet sich in \cite{kimura} (Part 2, Kap. 2).
  
  \subsection{Implementierung}
    Die Komponenten der erweiterte Weingartenabbildung können wir nun mit Hilfe des diskreten Primär-Dual-Gradienten im Mittel approximieren, das heißt
    \begin{align}
      \bar{h}^{i}_{\phantom{i}j} &= \left[ \nabla\vec{\nu} \right]_{ij} = \nabla_{j}\nu^{i}
                                 \approx \left[ \nabla^{\overline{pd}}\nu^{i} \right]_{j}
                                 =: \left[ S^{\overline{pd}} \right]_{ij}
    \end{align}
    mit \( i,j \in \left\{ 1,2,3 \right\} \).
    Für die Berechnung der Eigenwerte \( \left\{ \lambda_{0}, \lambda_{1}, \lambda_{2} \right\} \) an jedem Gitterpunkt 
    nutzen wir den von der MTL4 (Matrix Template Library 4) bereitgestellten QR-Algorithmus.
    Der betragsmäßig kleinste Eigenwert, der annährend null ist, ist o.E.d.A. \( \lambda_{0} \).
    Somit können die Gaußsche Krümmung und die mittlere Krümmung durch
    \begin{align}
      K &\approx \lambda_{1}*\lambda_{2} \formtext{bzw.} \\
      H &\approx \frac{1}{2}\left( \lambda_{1}+\lambda_{2}\right)    
    \end{align}
    approximiert werden.
    Mit dem Normalenraum auf den Gitterpunkten haben wir das gleiche Problem, wie schon mit dem Tangentialraum.
    Es ist nicht klar wo der Normalenraum auf den Ecken des Polyeders \( |K| \) liegen soll, denn eindeutig ist er nur auf den Inneren der Dreiecke
    bzw. auf den Voronoiecken.
    Deshalb mitteln wir wieder über einen 1-Ring des Primärgitterpunktes \( v \) zu\footnote{Der obere Index \( pp \) steht für
    Primär-Primär da die Auswertung einer vektorvertigen primären 0-Form auf einer primären Kante gemacht wird.}
    \begin{align}
    \label{eqNormalMittel}
    \begin{aligned}
      \vecover{\nu}{pp}(v) &= \left\langle \vecover{\nu}{pp}, v \right\rangle 
          := \frac{1}{\left| \star v \right|} \sum_{\sigma^{2}\succ v} \left| \star v \cap \sigma^{2}\right| 
                      \left\langle \vec{\nu}, \star\sigma^{2} \right\rangle \\
          &= \frac{1}{\left| \star v \right|} \sum_{\sigma^{2}\succ v} \left| \star v \cap \sigma^{2}\right| 
                                          \vecover{\nu}{\sigma^{2}} \formpunkt
    \end{aligned}
    \end{align}
    Die Elementnormalen \( \vecover{\nu}{\sigma^{2}} \) werden von AMDiS bereit gestellt.
    Wenn wir \( \vec{\nu}_{i} \) wieder im "`Dualem"' lösen wollen, das heißt anwenden des HodgeOperators auf die Gleichung
    \eqref{eqNormalMittel} und damit durchmultiplizieren der Gleichung mit \( \left| \star v_{i} \right| \), dann können auch hier die
    Eckennormalen elementweise bestimmt werden.
    Dadurch ergibt sich das lineare Gleichungssystem
    \begin{align}
        \label{eqWeingartenEq1}
        \left\langle *\bar{\nu}^{i} , \star v \right\rangle 
                &= \left\langle *\left[ \vecover{\nu}{pp} \right]_{i}, \star v \right\rangle \\
        \label{eqWeingartenEq2}
        \left\langle *\left[ \nabla^{\overline{pd}}\bar{\nu}^{i} \right]_{j} , \star v \right\rangle
            - \left\langle *\left[ S^{\overline{pd}} \right]_{ij} , \star v \right\rangle 
                &= 0
    \end{align}
    für alle \( i,j\in\left\{ 1,2,3 \right\} \) und \( v\in K^{(0)} \) in dem alle Operatoren elementweise bestimmt werden können.
    Zusammen ergibt das 12 DOFs pro Knoten \( v \).
    Gleichung \eqref{eqWeingartenEq1} (3 Dofs pro Knoten) und \eqref{eqWeingartenEq2} (9 Dofs pro Knoten) 
    könnenten zwar auch seperat gelöst werden wodurch wir
    Speichervorteile hätten, aber das Lösen der beiden kleineren Gleichungssystem wäre nicht schneller, da der Lösungsaufwand für die
    resultierenden dünnbesetzten Systeme mit angepassteen Lösern etwa linear ist.
    Hinzu würden außerdem noch Initialisierungskosten kommen.
    Die Weingartenabbildung \( S \) ist symmetrisch, das können wir von der numerisch ermittelten nicht exakt erwarten.
    Deshalb berechnen wir auch alle Einträge von \( S^{\overline{pd}} \) und nicht nur die obere Dreicksmatrix. 
    Die Eigenwerte können dann über die symmetrisch gemittelten Matrix erhalten werden, also von der Matrix
    \begin{align}
      \frac{1}{2} \left( S^{\overline{pd}} + \left( S^{\overline{pd}} \right)^{T} \right) \formpunkt
    \end{align}
    \todo[inline]{verweis auf experiment}



\section{Gauß-Bonnet-Operator}

\section{Krümmungsvektor}

\section{Numerisches Experiment}

%\section{Approximation der Krümmung}
%  \subsection{Beispiel: Krümmung Teil 1: Gauß-Bonnet-Operator}
%  \subsection{Beispiel: Krümmung Teil 2: Weingarten-Abbildung}
%  \subsection{Beispiel: Krümmung Teil 3: Krümmungsvektor}
%  \label{subsecKruemmungsvektor}

