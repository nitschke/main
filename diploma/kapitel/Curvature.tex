\chapter{Anwendung: Oberflächenkrümmung}
  
  \begin{ziel}
    In vielen wissenschaftlichen Problemen kann die Krümmung
    einer Oberfläche eine entscheidene Rolle spielen. 
    Sie kann zum Beispiel ein wichtiger Bestandteil von Differentialgleichungen sein.
    So kommt die mittlere Krümmung \( H \) bei fluidmechanischen Formulierungen auf bewegliche Membranen 
    \( M_{t} \) in der
    Kontinuitätsgleichung
    \begin{align}
      \dot{\rho} + \rho\div\vec{v} - \rho v_{\vec{\nu}} H &= 0
    \end{align}
    vor (siehe \cite{desimone}).
    Ein weiteres einfaches Beispiel ist die Evolution von Oberflächen \( M_{t} \) über den mittleren
    Krümmungsfluss
    \begin{align}
      \dot{\vec{x}} = 2 H \vec{\nu} 
    \end{align}
    oder die Bestimmung des topologisch invarianten Geschlechts
    \begin{align}
      \mathfrak{g} = 1 - \frac{1}{4\pi}\int_{M}KdA
    \end{align}
    einer orientierten Mannigfaltigkeit \( M \) ohne Rand mit Gaußscher Krümmung \( K \),
    was dierekt aus dem Satz von Gauß-Bonnet 
    und den Zusammenhang für die Euler-Charakeristik \( \chi = 2-2\mathfrak{g} \) folgt.
    Unter den gewählten Voraussetzungen gibt \( \mathfrak{g} \) die Anzahl der "`Löcher"'/"`Poren"'/"`Henkel"' an.
    
    Es ließen sich noch weitere Beispiele finden in dem diverse Krümmungsgrößen für die Behandlung von
    Nöten wären.
    Oftmals liegt die Oberfläche allerdings nur als diskrete Eingangsgröße vor. 
    Somit sind viele
    geometrische Werte nicht bekannt und müssen Approximiert werden.
    Wir werden uns in diesem Kapitel speziell die Gaußsche und die mittlere Krümmung betrachten.
    Diese werden wir mit hilfe des DECs approximieren und zum einen mit den Ergebnissen von C.-J. Heine 
    \cite{heine} vergleichen und zum anderen, im Falle der Gauß-Krümmung, mit einer naiven 
    Gauß-Bonnet-Diskretisierung.
  \end{ziel}

\section{Weingartenabbildung}
\label{secWeingartenabbildung}

\section{Gauß-Bonnet-Operator}

\section{Krümmungsvektor}

\section{Numerisches Experiment}

%\section{Approximation der Krümmung}
%  \subsection{Beispiel: Krümmung Teil 1: Gauß-Bonnet-Operator}
%  \subsection{Beispiel: Krümmung Teil 2: Weingarten-Abbildung}
%  \subsection{Beispiel: Krümmung Teil 3: Krümmungsvektor}
%  \label{subsecKruemmungsvektor}

