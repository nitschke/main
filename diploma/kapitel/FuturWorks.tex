\chapter{Fazit und Ausblicke}

\todo[inline]{unabhängigkeit zur ambiente dimension}

\todo[inline]{wechsel zu baryzentrischer Wohlzentriertheit => neuer Hodge Stern}
Es gilt \( dx^{2}= \sin\left( \gamma_{i} \right) d\xi \), 
wobei \( \left( x^{1},x^{2} \right) \) die orthogonalen Koordinaten sind,
bei der \( \frac{\partial}{\partial x^{1}} \) in Richtung der Kante \( \sigma^{1} \) zeigt,
\( \xi \) die Koordinate entlang \( \star\sigma^{1}\cap\sigma^{2}_{i} \) ist
und \( \gamma_{i} \) die Winkel zwischen \( \sigma^{1} \) und den elementaren Dualkanten
\( \star\sigma^{1}\cap\sigma^{2}_{i} = \pm \left[ c(\sigma^{1}),c(\sigma^{2}_{i})  \right] \) sind.
Somit kann man sich leicht überlegen, dass für eine diskrete Form \( \alpha\in\Omega^{1}_{d}(K) \)
\begin{align}
  \left\langle *\alpha, \star\sigma^{1} \right\rangle
      &= \frac{\sin\left( \gamma_{1} \right)\left| \star\sigma^{1}\cap\sigma^{2}_{1} \right|
              +\sin\left( \gamma_{2} \right)\left| \star\sigma^{1}\cap\sigma^{2}_{2} \right| }
              {\left| \sigma^{1} \right|}
           \left\langle \alpha,\sigma^{1} \right\rangle
\end{align}
den Hodge-Stern-Operator in dem von uns verwendeten Sinne approximiert.
\todo[inline]{Oberflächeninterpolation höher als linear -> interpolation der Differentalformen}
\todo[inline]{höhere dimensionen der mannigfaltigkeit}
\todo[inline]{voller satz v. baukastenopratoren: d,*,flat,sharp,wedge}
\todo[inline]{Randbedingungen}
\todo[inline]{Interpolation von diffformen}

