\documentclass[a4paper,11pt]{report}
\usepackage{amsmath}
\usepackage{a4wide}
\usepackage{ngerman}

\renewcommand{\familydefault}{\sfdefault}

\parindent0cm

% Muster zur Gestaltung des Deckblattes einer Diplomarbeit

\begin{document}
\thispagestyle{empty}
\begin{center}
\Huge

Technische Universit"at Dresden

Fachrichtung Mathematik

\bigskip

\LARGE

Institut f"ur Wissenschaftliches Rechnen % Institut eintragen

\vfill

\huge

\textbf{Diskretes "Au"seres Kalk"ul (DEC) \\
        auf Oberfl"achen ohne Rand}

\LARGE

\vfill

Diplomarbeit \\
zur Erlangung des ersten akademischen Grades

\bigskip

\textbf{Diplommathematiker(in)}
%\textbf{(Wirtschaftsmathematik)} % Nur Zutreffendes stehen lassen
\textbf{(Technomathematik)} % Nur Zutreffendes stehen lassen

%{\large \textit{("`Wirtschaftsmathematik"' ggf. streichen bzw. durch "`Technomathematik"' ersetzen)}}

\end{center}

\vfill

\Large

vorgelegt von

\bigskip \bigskip

\begin{tabular}{@{}lp{4cm}@{\qquad}rl@{}}
Name: & Nitschke & Vorname: & Ingo\\[2.0ex]
geboren am: & 04.05.1983 & in: & K"onigs Wusterhausen
\end{tabular}

\bigskip \bigskip \bigskip \bigskip

Tag der Einreichung: \ 08.09.2014 %TT.MM.JJJJ

\bigskip \smallskip

\begin{tabular}{@{}ll@{}}
Betreuer: & Prof. Dr.rer.nat.habil Axel Voigt  %\qquad \qquad \textit{(bitte mit vollst"andigem akademischen Grad)}
\\[1.5ex] % Vollst�ndiger Titel + Name
          %& ........  % Nur bei zwei Betreuern
\end{tabular}

\normalsize

\end{document}
