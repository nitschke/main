%\documentclass{beamer}
\documentclass[handout]{beamer}
\usepackage[ngerman]{babel}
\usepackage[utf8]{inputenc}
\usepackage{graphicx}
\usepackage{amsmath}
\usepackage{amssymb}
\usepackage{dsfont}
\usepackage[T1]{fontenc}
\usepackage{pstricks}
\usepackage{pst-node}
%\usepackage[english]{babel}
%\usepackage[fixlanguage]{babelbib}
\usepackage{multimedia}


\usetheme{Warsaw}
%\useinnertheme{rounded}
\useoutertheme{infolines}
%\setbeamercovered{transparent}

\title[Einführung in DEC]{Einführung in das Kalkül diskreter Differentialformen (DEC)}
\author{Ingo Nitschke}
\institute{IWR - TU Dresden}

\beamertemplatenavigationsymbolsempty

\newcommand{\R}{\mathds{R}}
\newcommand{\eps}{\varepsilon}
\newcommand{\qqquad}{\qquad\qquad}
\newcommand{\rot}{\text{rot}}
\newcommand{\sgn}{\text{sgn}}
\renewcommand{\div}{\text{div}}
\renewcommand{\d}{\textbf{d}}
\newcommand{\ablx}[2]{\frac{\partial #1}{\partial x^{#2}}}


\begin{document}
  
  \begin{frame}
    \begin{align}
      \left\{\text{p1}^{(2,0)}(u,v)+\cot (u) \text{p1}^{(1,0)}(u,v)+\csc (u) \left(\text{p1}^{(0,2)}(u,v)-\csc (u) \text{p1}(u,v)\right)-(\sin (u)-1) \text{p2}^{(1,1)}(u,v)-2 \cos (u)
   \text{p2}^{(0,1)}(u,v),\csc ^2(u) \left(2 \cot (u) \text{p1}^{(0,1)}(u,v)+\text{p2}^{(0,2)}(u,v)\right)+\text{p2}^{(2,0)}(u,v)+3 \cot (u) \text{p2}^{(1,0)}(u,v)-2 \text{p2}(u,v)\right\}
    \end{align}
  \end{frame}

\end{document}
