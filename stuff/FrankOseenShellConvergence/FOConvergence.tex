\documentclass[a4paper,11pt]{scrartcl}

\usepackage{ngerman}
\usepackage[utf8]{inputenc}
%\usepackage[T1]{fontenc}

\usepackage{hyperref}
\usepackage{color}
\usepackage{amsmath}
\usepackage{amsfonts}
\usepackage{amssymb}
\usepackage{amsbsy}
\usepackage{bm}
\usepackage{graphicx}
\usepackage{extarrows}

\newcommand{\ie}{i.\,e.}%
\newcommand{\eg}{e.\,g.}%
\newcommand{\etal}{et\,al.}%
\newcommand{\wrt}{w.r.t.}%
\newcommand{\sign}{\operatorname{sign}}

\newcommand{\leviCivita}{\varepsilon}

% comma and period at end of formulas
\newcommand{\formComma}{\,\text{,}}
\newcommand{\formPeriod}{\,\text{.}}

% real and complex numbers
\newcommand{\R}{\mathbb{R}}%
% \newcommand{\C}{\mathbb{C}}%

% bold symbols
\newcommand{\xb}{\mathbf{x}}%
\newcommand{\rb}{\mathbf{r}}%
\newcommand{\ub}{\mathbf{u}}%
\newcommand{\nb}{\mathbf{n}}%
\newcommand{\eb}{\bm{e}}%
\newcommand{\gb}{\mathbf{g}}%
\newcommand{\Xb}{\mathbf{x}}% parametrization X:{(theta,phi)} --> R3

\newcommand{\alphab}{\bm{\alpha}}% for \pb^{\flat}
\newcommand{\betab}{\bm{\beta}}% for \qb^{\flat}

\newcommand{\eqAngle}{\phi}

% director and its components
\newcommand{\pb}{\mathbf{p}}%
\newcommand{\p}{\textup{p}}%
% test funciton and its components
\newcommand{\q}{\textup{q}}%
\newcommand{\qb}{\mathbf{q}}%
\newcommand{\tb}{\mathbf{t}}%
\newcommand{\Eb}{\mathbf{E}}%
\newcommand{\Mbb}{\mathbf{M}}%
\renewcommand{\sb}{\mathbf{s}}%
\newcommand{\vb}{\mathbf{q}}%to unify test functions

\newcommand{\alphav}{\underline{\bm{\alpha}}} % for vector-1-forms, like PD-1-form
\newcommand{\betav}{\underline{\bm{\beta}}} % for vector-1-forms, like PD-1-form

\newcommand{\abs}[1]{\lvert#1\rvert}%
\newcommand{\norm}[1]{\lVert#1\rVert}%
\newcommand{\scalarprod}[1]{\big\langle{#1}\big\rangle}%
\newcommand{\Scalarprod}[1]{\left\langle{#1}\right\rangle}%

% derivatives
\newcommand{\dif}{\textup{d}}
\newcommand{\exd}{\mathbf{d}} %exterior derivative
\newcommand{\ds}{\,\dif{s}}
\newcommand{\dr}{\,\dif\rb}
\newcommand{\dx}{\,\dif\xb}
\newcommand{\dxhat}{\,\dif\hat{\xb}}
\newcommand{\dt}{\partial_t}

% differential forms for integration
\newcommand{\dS}{\,\dif{\surf}}
\newcommand{\dV}{\,\dif{V}}

% functional derivative
\newcommand{\fdif}{\operatorname{\delta}\!}
\newcommand{\Fdif}[2]{\frac{\fdif{#1}}{\fdif{#2}}}% dF / du	
\newcommand{\FFdif}[3]{\frac{\fdif^2{#1}}{\fdif{#2}\fdif{#3}}}% d^2F / dudv	

\newcommand{\FF}{\mathrm{F}}
\newcommand{\F}[1]{\FF_\mathrm{#1}}

% surface and domain
\newcommand{\Sp}{\mathbb{S}^2}
\newcommand{\ellipsoid}{\mathcal{E}}
\newcommand{\surf}{\mathcal{S}}
\newcommand{\domain}{\Omega}

\newcommand{\meanCurvature}{\mathcal{H}}
\newcommand{\gaussianCurvature}{\kappa}

% surface (differential operators)
\newcommand{\Grad}{\operatorname{Grad}}
\renewcommand{\div}{\operatorname{div}}%
\newcommand{\Rot}{\operatorname{Rot}}%
\newcommand{\rot}{\operatorname{rot}}%
\newcommand{\DivSurf}{\Div_{\surf}}%
\newcommand{\GradSurf}{\Grad_{\surf}}
\newcommand{\laplace}{\Delta}
\newcommand{\laplaceBeltrami}{\Delta_{\surf}}
\newcommand{\vecLaplace}{\boldsymbol{\Delta}}
%\newcommand{\laplaceDeRahm}{\vecLaplace^{\textup{dR}}_{\surf}}
\newcommand{\laplaceDeRham}{\vecLaplace^{\textup{dR}}}
\newcommand{\laplaceDeRahm}{\laplaceDeRham}
\newcommand{\laplaceRotRot}{\vecLaplace^{\textup{RR}}}
\newcommand{\laplaceGradDiv}{\vecLaplace^{\textup{GD}}}

\newcommand{\laplaceDeRhamTilde}{\widehat{\vecLaplace}^{\textup{dR}}}
\newcommand{\NablaSurf}{\nabla_{\surf}}
\newcommand{\gDerivative}{D}

\newcommand{\laplaceDeRhamDiffuse}{\widehat{\vecLaplace}^{\textup{dR}}_{\phi}}

\newcommand{\surfNormal}{\boldsymbol{\nu}}
\newcommand{\surfNormalI}{\nu}

\newcommand{\ProjectSurf}{\pi_\surf}

\newcommand{\Tangent}{\mathsf{T}}

% General
\newcommand{\vect}[1]{\mathbf{#1}}
\newcommand{\tensor}[1]{\mathbf{#1}}

\newcommand{\Span}[1]{\operatorname{Span}\!\left\{ #1 \right\}}

\newcommand{\AMDIS}{\texttt{AMDiS}}
\newcommand{\PETSC}{\texttt{PETSc}}

% Constants
\newcommand{\K}{{K}} % one-constant
\newcommand{\Ki}{{K_1}} % frank-constant K1
\newcommand{\Kii}{{K_2}} % frank-constant K2
\newcommand{\Kiii}{{K_3}} % frank-constant K3
\newcommand{\Kn}{{\omega_n}} % penalty constant for normality
\newcommand{\Kt}{{\omega_t}} % penalty constant for tangentiality

% Energies
\newcommand{\EF}{E^{\textup{F}}} % 3D Frank-Energy
\newcommand{\EFSurf}{E^{\textup{F}}_{\surf}} % Frank-Energy on surface
\newcommand{\ESurf}{E_{\surf}} % intrinsic Frank-Energy plus normalizing term on surface

\newcommand{\EE}{\F{\omega}^\surf}
\newcommand{\E}[1]{\EE[#1]}


\newcommand{\Ltwo}[2]{L^2( #1;\, #2)}%
\newcommand{\LS}{L^2(\surf)}%
\newcommand{\LtwoProd}[1]{\big({#1}\big)_{\LS}}%

\newcommand{\C}[2]{{C}( #1;\, #2)}
\newcommand{\Csurf}[1]{{C}^{#1}(\surf)}
\newcommand{\Cdomain}[1]{{C}^{#1}(\domain)}
\newcommand{\extend}[1]{\widehat{#1}}

\newcommand{\Hdr}[2]{H^{\textup{DR}}( #1;\, #2)} %
\newcommand{\HdrExt}[2]{H^{\textup{DR}}( #1;\, #2)} % \extend{H}^{\textup{DR}}\left( #1 ; #2 \right)
\newcommand{\HdrDiffuse}[2]{H^{\textup{DR}}( #1;\, #2)} %\extend{H}^{\textup{DR}}_{\phi}\left( #1 ; #2 \right)

\newcommand{\Hi}[1]{H^{1}(#1)}

\newcommand{\pExt}{\extend{\pb}}
\newcommand{\qExt}{\extend{\qb}}
\newcommand{\nExt}{\extend{\surfNormal}}

% rotation angle
\newcommand{\rotAngle}{\psi}

% DEC declarations
\newcommand{\SC}{\mathcal{K}} % simplicial complex
\newcommand{\Vs}{\mathcal{V}} % set of vertices
\newcommand{\Es}{\mathcal{E}} % set of edges
\newcommand{\Fs}{\mathcal{T}} % set of faces
\newcommand{\face}{T} % one face
\newcommand{\FormSpace}{\Lambda^{1}} % space of 1 forms
\newcommand{\PDT}{\mathfrak{T}} % PD-Tangential-Space

% for commends among each other
\newcommand{\simon}[1]{{\color{red}#1}}
\newcommand{\ingo}[1]{{\color{blue}#1}}
\newcommand{\micha}[1]{{\color{green}#1}}

\newcommand{\hidden}[1]{}

%\theoremstyle{plain}
\newtheorem{lem}{Lemma}
\newtheorem{thm}{Theorem}
\newtheorem{prop}[thm]{Proposition}
\newtheorem{rem}{Remark}[section]
\newtheorem{exmp}{Example}[section]

%\theoremstyle{definition}
\newtheorem{defn}{Definition}

\makeatother
\providecommand*{\thmautorefname}{Theorem}
\providecommand*{\remautorefname}{Remark}
\providecommand*{\defnautorefname}{Definition}
\providecommand*{\algorithmautorefname}{Algorithm}
%\addto\extrasenglish{%
  %\renewcommand{\chapterautorefname}{Chapter}%
  %\renewcommand{\sectionautorefname}{Section}%
  %\renewcommand{\subsectionautorefname}{Section}%
  %\renewcommand{\subsubsectionautorefname}{Section}%
  %\renewcommand{\paragraphautorefname}{Section}%
%}
\newcommand*{\End}{\hfill\ensuremath{\vartriangleleft}}%
\newcommand*{\END}{\hfill\ensuremath{\blacktriangleleft}}%




\usepackage{tensor}

\newcommand{\Tr}{\text{Tr}}
\newcommand{\RSpan}[1]{\text{Span}_{\R}\left\{ #1 \right\}}

\newcommand{\surfh}{\surf_{h}}
\newcommand{\landau}{\mathcal{O}}
\newcommand{\adj}{\text{adj}}
\newcommand{\tildeCh}[3]{\widetilde{\Gamma}_{#1 #2}^{#3}} %Chrisoffel-Symb. 2. Art in der Dünnschicht
\newcommand{\Ch}[3]{\Gamma_{#1 #2}^{#3}} %Chrisoffel-Symb. 2. Art auf der Oberfläche


\newcommand{\lGd}{\Delta^{\text{Gd}}}
\newcommand{\ldG}{\Delta^{\text{dG}}}
\newcommand{\lRr}{\Delta^{\text{Rr}}}


\title{Notizen zur Konvergenz der Frank-Oseen-Energie im Dünnschichtmodell gegen die Oberflächenformulierung} 
\author{Ingo Nitschke}

\begin{document}
\maketitle

\section{Vorüberlegungen / Konvergenzdefinition}

Im Folgendem betrachten wir die Dünnschicht \( \surfh = \surf \times \left[ -h/2 , h/2 \right] \) 
der Dicke \( h \) um eine Oberfläche \( \surf \).
Wir behalten hierbei die gewählten lokalen Koordinaten \( \left\{ u,v \right\} \) der Oberfläche bei und ergänzen diese um eine Koordinate
\( \xi \), welche in Normalenrichtung wirkt.
Somit lässt sich jeder Ort \( \Xb\in\surfh \) der Dünnschicht durch einem Ort \( \xb\in\surf \) auf der Oberfläche und der orthogonalen Entfernung von
dieser beschreiben durch
\begin{align}
  \Xb = \Xb(u,v,\xi) = \xb(u,v) + \xi\surfNormal(u,v) = \xb + \xi\surfNormal \formComma
\end{align}
Dabei sei \( h \) klein genug gewählt, dass obige Gleichung eindeutig ist. \ingo{Lässt sich vermutlich über die Hauptkrümmungen abschätzen
(vgl. Napoli). Ist aber wegen dem Grenzübergang nur von theoretischer Relevanz.}
Dadurch ergibt sich eine natürliche Metrik in \( \surfh \) durch
\begin{align}
  G_{IJ} = \partial_{I}\Xb \cdot \partial_{J}\Xb \formPeriod
\end{align}
Wir wollen hier und im Folgenden festlegen, dass groß geschriebene Laufindizes \( I,J,\ldots,M \) alle drei Komponenten \( u,v \) und \(
\xi\) erfassen und kleine Laufindizes \( i,j,\ldots,m \) nur die tangentialen Komponenten \( u \) und \( v \).
Somit erhalten wir die Basisvektoren
\begin{align}
  \partial_{i}\Xb &= \partial_{i}\xb + \xi\partial_{i}\surfNormal \formComma \\
  \partial_{\xi}\Xb &= \surfNormal \formPeriod
\end{align}
Mit dem Shape-Operator (bzw. 2. Fundamentalform) der Oberfläche \( B_{ij} = -\partial_{i}\surfNormal\cdot\partial_{j}\xb \),
dessen Symmetrie und dessen Quadrat \( [B^{2}]_{ij} = \tensor{B}{_{i}^{k}}B_{kj} =  \partial_{i}\surfNormal\cdot\partial_{j}\surfNormal\)
\ingo{(das muss noch formal bewiesen werden bzw. Quelle finden. Wurde aber schon ganz allg. mit Mathematica gezeigt und gilt somit als fast
sicher.)}
ergibt sich nun für den metrischen Tensor
\begin{align}
  G_{ij} &= g_{ij} - 2\xi B_{ij} + \xi^{2}[B^{2}]_{ij} 
    = \left( \tensor{\delta}{_{i}^{k}} - \xi\tensor{B}{_{i}^{k}} \right) \left( g_{ij} - \xi B_{ij} \xi \right)
    = \left[ \left( g-\xi B \right)^{2} \right]_{ij}\formComma \\
    %=: [G_{T}]_{ij} \formComma \\
  G_{\xi\xi} &= 1 \formComma\\
  G_{i\xi} &= G_{\xi i} = 0 \formPeriod
\end{align}
Dabei ist \( g=\left\{ G_{ij} \right\}|_{\surfh} \) der metrische Tensor der Oberfläche \( \surf \).
Für das integrieren skalarer Quantitäten in der Dünnschicht benötigen wir das Volumenelement
\begin{align}
  \mu_{\surfh} = \sqrt{|G|} d\xi \wedge du \wedge dv  = \sqrt{|G|} d\xi  du  dv\formPeriod
\end{align}

\begin{thm}
  Die Komponente des Volumenelements in der Dünnschicht \( \surfh \) lässt sich darstellen als
  \begin{align}
    \sqrt{|G|} = \left( 1 + \xi\meanCurvature + \xi^{2}\gaussianCurvature \right)\sqrt{|g|}
  \end{align}
  mit der Komponente des Volumenelements auf der Oberfläche \( \sqrt{|g|} \), 
  der mittleren Krümmung \( \meanCurvature \) und der Gaußschen Krümmung \( \gaussianCurvature \) der Oberfläche \( \surf \). 
\end{thm}
\begin{proof}
  Für die Determinante des metrischen Tensors ergibt sich durch
  \begin{align}
    |G| = G_{\xi\xi}|\{G_{ij}\}| = |\left( g-\xi B \right)^{2}| \label{eq:Det2Full}
  \end{align}
  Für die Determinante eines covarianten 2-Tensor-Quadrats ergibt sich allgemein für \( t=\left\{ t_{ij} \right\} \)
  \begin{align}
    |t^{2}| = |t\cdot g^{-1}\cdot t| = |t|^{2}|g^{-1}| = \frac{|t|^{2}}{|g|}\formPeriod
  \end{align}
  Somit erhalten wir
  \begin{align}
    |G| = \frac{|g-\xi B |^{2}}{|g|}\formPeriod\label{eq:detG2}
  \end{align}
  Schreiben wir nun die Wurzel des Zählers explizit aus erhalten wir
  \begin{align}
    |g-\xi B | &= \left( g_{uu}-\xi B_{uu} \right)\left( g_{vv} - \xi B_{vv} \right) - \left( g_{uv} - \xi B_{uv} \right)^{2}\\
     &= \left( g_{uu}g_{vv} - g^{2}_{uv} \right)
        + \xi \left( -g_{vv}B_{uu} - g_{uu}B_{vv} + 2g_{uv}B_{uv}  \right)
        + \xi^{2} \left( B_{uu}B_{vv} - B^{2}_{uv} \right) \label{eq:DetExpand} \formPeriod
  \end{align}
  Der erste Summand in \eqref{eq:DetExpand} ist die Determinante \( |g| \)
  und der dritte Summand ist die Determinante \( |B| \) der zweiten Fundamentalform, wobei gilt, dass
  \begin{align}
    |B| = |g\cdot \{\tensor{B}{^{i}_{j}}\}| = |g| |^{\sharp}B| = |g| \gaussianCurvature \formPeriod
  \end{align}
  Die Diagonaleinträge des Shape-Operators \( ^{\sharp}B = g^{-1} \cdot B \) sind
  \begin{align}
    \tensor{B}{^{u}_{u}} &= g^{ui} B_{iu} = \frac{g_{vv}B_{uu}-g_{uv}B_{uv}}{|g|} \formComma \\
    \tensor{B}{^{v}_{v}} &= g^{vi} B_{iv} = \frac{g_{uu}B_{vv}-g_{uv}B_{uv}}{|g|}
  \end{align}
  und somit ergibt sich der zweite Summand in \eqref{eq:DetExpand} durch
  \begin{align}
    -|g|\meanCurvature = |g|\Tr(^{\sharp}B) = g_{vv}B_{uu} + g_{uu}B_{vv} - 2g_{uv}B_{uv}\formComma
  \end{align}
  also
  \begin{align}
    |g-\xi B | = |g|\left(  1 + \xi\meanCurvature + \xi^{2}\gaussianCurvature \right) \formPeriod
  \end{align}
  Mit \eqref{eq:Det2Full} und \eqref{eq:detG2} folgt schlußendlich die Behauptung.
\end{proof}

Das Volumen der Dünnschicht \( \surfh \) berechnet sich nun über
\begin{align}
  |\surf_{h}| &= \int_{\surfh}\mu_{\surfh} 
               = \int_{\surfh}\sqrt{|G|} d\xi du dv
  = \int_{\surf}\int_{-\frac{h}{2}}^{\frac{h}{2}} \left( 1 + \xi\meanCurvature + \xi^{2}\gaussianCurvature \right) d\xi \sqrt{|g|} du dv\\
  &= \int_{\surf} h + \frac{h^{3}}{12}\gaussianCurvature \mu
   = h \left(|S| + \frac{h^{2}\pi}{6}\chi(\surf)\right)\formPeriod
\end{align}
Somit folgt für \( h\rightarrow 0 \), dass
\begin{align}
  \frac{|\surfh|}{h} \rightarrow |S| \formPeriod
\end{align}
Da wir nun einen zu erwartenden Grenzübergang der Eins und somit auch für alle konstanten Dichten gefunden haben,
scheint es sinnvoll zu sein diesen Konvergenzbegriff auch auf beliebige hinreichend glatte Dichten zu übertragen.

\begin{defn}\label{def:Convergence}
  Wir sagen die Dichte \( f:\surfh \rightarrow \R \) konvergiert gegen die Dichte 
  \( f^{\surf}: \surf \rightarrow \R \), genau dann wenn für \( h \rightarrow 0 \)
  \begin{align}
    \frac{1}{h}\int_{\surfh} f \mu_{\surfh} \rightarrow \int_{\surf} f^{\surf} \mu
  \end{align}
  gilt.
\end{defn}

Durch die Darstellung des Volumenelement als
\begin{align}
 \mu_{\surfh} =  \left(  1 + \xi\meanCurvature + \xi^{2}\gaussianCurvature \right) d\xi \wedge\mu
              = \left( 1 + \landau(\xi) \right) d\xi \wedge\mu
\end{align}
ist folgendes Lemma ersichtlich.

\begin{lem}\label{lem:Convergence}
  Wenn für die Dünnschichtdichte \( f \), Oberflächendichte \( f^{\surf} \) und \( h \rightarrow 0 \)
  \begin{align}
    \frac{1}{h}\int_{-\frac{h}{2}}^{\frac{h}{2}} f d\xi \rightarrow f^{\surf}
  \end{align}
  gilt, dann konvergiert \( f \) gegen \( f^{\surf} \) im Sinne von \autoref{def:Convergence}.
\end{lem}

\begin{con}\label{con:Convergence}
  Speziell für \( f(\Xb) = f^{\surf}(\xb) + \landau(\xi) \) folgt, dass \( f \) gegen \( f^{\surf} \) konvergiert.
\end{con}
\begin{proof}
  Mit
  \begin{align}
    \frac{1}{h}\int_{-\frac{h}{2}}^{\frac{h}{2}} f^{\surf}(\xb) + \landau(\xi) d\xi
        = f^{\surf} + \landau(h) \rightarrow f^{\surf}
  \end{align}
  und \autoref{lem:Convergence} folgt die Behauptung.
\end{proof}

\section{Approximation metrischer Größen}

Wie wir noch sehen werden, reicht für unsere Zwecke eine Approximation 0-ter Ordnung (in \( \xi \)) des metrischen Tensors aus.
Einzig für die Berechnung der Christoffel-Symbole benötigen wir wegen der partiellen Ableitungen von \( G \) eine Approximation erster
Ordnung.
Das heißt
\begin{align}
  G_{\xi\xi} &= 1\formComma & G_{i\xi} &= G_{\xi i} = 0 \text{ und}\\
  G_{ij} &= g_{ij} + \landau(\xi)_{ij} \text{ bzw.} &   G_{ij} &= g_{ij} - 2\xi B_{ij} + \landau(\xi^{2})_{ij} \formPeriod
\end{align}

Hieraus lässt sich eine natürliche Norm für Vektorfelder \( \pExt:\surfh \rightarrow \Tangent\surfh \cong \R^{3} \) mit
\( \pExt(\Xb) = \pb(\xb) + \landau(\xi)   \) und \( \pb:\surf \rightarrow \Tangent\surf\times\Tangent\surf^{\bot} \cong \R^{3} \)
entwickeln zu
\begin{align}
  \left\| \pExt \right\|^{2}_{\surfh} &= G_{IJ}\pIExt^{I}\pIExt^{J}
                                       = g_{ij}p^{i}p^{j} + \left( p^{\xi} \right)^{2} + \landau(\xi) \\
                                      &= \left\| \pb \right\|^{2} + \left( p^{\xi} \right)^{2} + \landau(\xi) 
                                            \label{eq:norm}\formComma
\end{align}
wobei \( \left\| \pb \right\| := \left\| \pb \right\|_{\surf} \) die (Tangential-)Norm auf \( \surf  \) ist.

Für die inverse Metrik \( G^{-1} =: \left\{ G^{IJ} \right\} \) ergibt sich über Blockinversion und Schurkomplement gleich 1
\begin{align}
  G^{-1} =
    \begin{bmatrix}
      \{G^{ij}\} & \{G^{i\xi}\} \\
      \{G^{\xi i}\} & G^{\xi\xi}
    \end{bmatrix} 
    =
    \begin{bmatrix}
      G_{T}^{-1} & 0 \\
      0 & 1
    \end{bmatrix}
\end{align}
mit \( G_{T} = \{G_{ij}\} \). 
Dabei nutzen wir Taylor an \( \xi = 0 \), so dass
\begin{align}
  \{G^{ij}\} =
  G_{T}^{-1} &= \frac{1}{|G_{T}|}\adj G_{T}
            = \frac{1}{|g + \landau(\xi)|}\left( \adj g + \adj\landau(\xi) \right) \\
            &\overset{\text{Taylor}}{=} \frac{1}{|g|}\adj g + \landau(\xi)
            = g^{-1} + \landau(\xi) =: \{g^{ij} + \landau(\xi)^{ij}\}
\end{align}
gilt.
\ingo{Exakt ließe sich auch zeigen, dass
\begin{align}
  G^{ij} = \frac{g^{ij} - 2\xi\gaussianCurvature[B^{-1}]^{ij} + \xi^{2}\gaussianCurvature^{2}[B^{-2}]^{ij}}
                {(1+\xi\meanCurvature + \xi^{2}\gaussianCurvature)^{2}}
\end{align}
gilt.}

Die partiellen Ableitungen des Metriktensors lassen sich darstellen durch
\begin{align}
  \partial_{k}G_{ij} &= \partial_{k}g_{ij} + \landau(\xi)_{ijk} \\
  \partial_{\xi}G_{ij} &= -2B_{ij} + \landau(\xi)_{ij} \\
  \partial_{K}G_{\xi I} &=  \partial_{K}G_{I \xi} = 0
\end{align}
Somit ergeben sich für die Christoffel-Symbole \( \tildeCh{I}{J}{K} \) in der Dünnschicht
\begin{align}
  \tildeCh{I}{J}{K} &= \frac{1}{2}G^{KL}\left( \partial_{I}G_{JL} + \partial_{J}G_{IL} - \partial_{L}G_{IJ} \right) \\
  \tildeCh{i}{j}{k} &= \frac{1}{2}G^{kL}\left( \partial_{i}G_{jL} + \partial_{j}G_{iL} - \partial_{L}G_{ij} \right) \\
                    &= \frac{1}{2}\left( g^{kl} + \landau(\xi)^{kl} \right)
                                  \left( \partial_{i}g_{jl} + \partial_{j}g_{il} - \partial_{l}g_{ij} + \landau(\xi)_{ijl} \right) \\
                    &= \Ch{i}{j}{k} + \landau(\xi)_{ij}^{k} \\
  \tildeCh{i}{j}{\xi} &= \frac{1}{2}G^{\xi\xi}\left( \partial_{i}G_{j\xi} + \partial_{j}G_{i\xi} - \partial_{\xi}G_{ij} \right) 
                       = -\frac{1}{2}\partial_{\xi}G_{ij} \\
                      &= B_{ij} + \landau(\xi)_{ij} \\
  \tildeCh{i}{\xi}{k} = \tildeCh{\xi}{i}{k} 
                    &= \frac{1}{2}G^{kL}\left( \partial_{i}G_{\xi L} + \partial_{\xi}G_{iL} - \partial_{L}G_{i\xi} \right) \\
                    &= \frac{1}{2}G^{kl}\partial_{\xi}G_{il}
                     = \frac{1}{2}\left( g^{kl} +  \landau(\xi)^{kl}\right)\left( -2B_{il} + \landau(\xi)_{il} \right) \\
                    &= -\tensor{B}{_{i}^{k}} + \landau(\xi)_{i}^{k}\\
  \tildeCh{\xi}{\xi}{K} &= 0 \\
     \tildeCh{I}{\xi}{\xi} = \tildeCh{\xi}{I}{\xi} &= 0
\end{align}

Der Levi-Civita-Tensor \( \widetilde{E} \) in der Dünnschicht beschreibt das Verhalten des Volumenelements als 3-Form, d.h.
\begin{align}
  \tilde{E}_{IJK} = \mu_{\surfh}(\partial_{I}\Xb,\partial_{J}\Xb,\partial_{K}\Xb)
                  = \sqrt{|G|}\varepsilon_{IJK} 
                  = \sqrt{|g|}\varepsilon_{IJK} + \landau(\xi)_{IJK}
\end{align}
mit den Levi-Civita-Symbolen \( \varepsilon_{IJK}\in\left\{ -1,0,1 \right\} \).
Da jedes nicht verschwindendes Levi-Civita-Symbol genau einen \( \xi \)-Index und zwei Tangentialindices hat, ergibt sich somit
\begin{align}
  \tilde{E}_{\xi i j} 
  = - \tilde{E}_{i \xi j}
  = \tilde{E}_{i j \xi}
  = E_{ij} + \landau(\xi)_{ij}
\end{align}
mit dem Oberflächen-Levi-Civita-Tensor \( E_{ij} = \sqrt{|g|}\varepsilon_{ij}=\mu\left( \partial_{i}\xb, \partial_{j}\xb \right) \).


\section{Grenzübergang der Frank-Oseen-Energie vom Dünnschichtmodell zum Oberflächenmodell}
Die 3-dimensionale Formulierung der Frank-Oseen-Energie in der Dünnschicht 
mit Straftermen für die Normierung zu eins und der Tangentialität zur Oberfläche \( \surf \) 
ist gegeben durch
\begin{align}
    F\left[ \pExt, \surfh \right] 
        &= \frac{1}{2}\int_{\surfh} \Ki f_{\text{splay}}[\pExt] + \Kii f_{\text{twist}}[\pExt] +  \Kiii f_{\text{bend}}[\pExt]
                                    +\frac{\Kn}{2}f_{\text{norm}}[\pExt] + \Kt f_{\text{tan}}[\pExt]\mu_{\surfh}
\end{align}
wobei die einzelnen Energiedichten gegeben sind mit
\begin{align}
  f_{\text{splay}}[\pExt] &= \left( \nabla\cdot\pExt \right)^{2} \\
  f_{\text{twist}}[\pExt] &= \left( \pExt\cdot\left( \nabla\times\pExt \right) \right)^{2} \\
  f_{\text{bend}}[\pExt] &= \left\| \pExt\times\left( \nabla\times\pExt \right) \right\|^{2}_{\surfh} \\
  f_{\text{norm}}[\pExt] &= \left( \left\| \pExt \right\|^{2}_{\surfh} - 1 \right)^{2} \\
  f_{\text{tan}}[\pExt] &= \left( \pExt \cdot \surfNormal \right)^{2}
\end{align}
Für das Vektorfeld \( \pExt:\surfh \rightarrow \Tangent\surfh \cong \R^{3} \) setzen wir voraus, 
dass es sich parallel und lengentreu in Normalenrichtung fortsetzt,
d.h. \( \tensor{\pIExt}{^{I}_{;\xi}} = 0 \).
Um Missverständnisse vorzubeugen legen wir fest, dass die covarianten Ableitungen in der Dünnschicht \( \surfh \) mit einem Semikolon gekennzeichnet
werden, d.h.
\begin{align}
  \tensor{\pIExt}{^{I}_{;J}} = \partial_{J}\pIExt^{I} + \tildeCh{J}{K}{I}\pIExt^{K} 
\end{align}
und die covarianten Ableitung auf der Oberfläche \( \surf \) mit einem senkrechten Strich, d.h.
\begin{align}
  \tensor{p}{^{i}_{|j}} = \partial_{j}p^{i} + \Ch{j}{k}{l}p^{l} \formPeriod
\end{align}
Vorerst lassen wir zu, dass \( \pExt \) auch eine nicht verschwindende Normalenkomponente \( \pIExt^{\xi} \) auf \( \surf \) haben darf.
Da
\begin{align}
  0 &= \tensor{\pIExt}{^{i}_{;\xi}} = \partial_{\xi}\pIExt^{i} + \tildeCh{\xi}{K}{i}\pIExt^{K} 
                                    = \partial_{\xi}\pIExt^{i} - \tensor{B}{_{l}^{i}}\pIExt^{l} + \landau(\xi)^{i}
  \text{ und} \\
  0 &= \tensor{\pIExt}{^{\xi}_{;\xi}} = \partial_{\xi}\pIExt^{\xi} + \tildeCh{\xi}{K}{\xi}\pIExt^{K}
                                      = \partial_{\xi}\pIExt^{\xi}
\end{align}
gilt, erhalten wir durch Taylor-Entwicklung
\begin{align}
  \pIExt^{\xi}(\Xb) &= \pIExt^{\xi}(\xb + \xi\surfNormal) = p^{\xi}(\xb)
  \text{ und} \\
  \pIExt^{i}(\Xb) &= \pIExt^{i}(\xb + \xi\surfNormal)
                   = p^{i}(\xb) + \landau(\xi)^{i}
  \text{ bzw.} \\
  \pIExt^{i}(\Xb) &=  p^{i}(\xb) + \xi\tensor{B}{_{l}^{i}}(\xb)p^{l}(\xb) + \landau(\xi^{2})^{i} \formPeriod
\end{align}
Somit lässt sich \( \pExt \) linear in \( \xi \) durch 
\( \pb:\surf \rightarrow \Tangent\surf\times\Tangent\surf^{\bot} \cong \R^{3} \) abschätzen.

Wir wollen nun die einzelnen Energiedichten nacheinander so abschätzen, dass wir \autoref{con:Convergence} ausnutzen können.

\subsection{Splay}

Die Divergenz eines Vektorfeldes ergibt sich aus der Kontraktion der covarianten Ableitungen.
Bzgl. der obigen Dünnschichtmetrik erhalten wir
\begin{align}
  \nabla\cdot\pExt &= \tensor{\pIExt}{^{I}_{;I}}
                    = \partial_{I} \pIExt^{I} + \tildeCh{I}{J}{I} \pIExt^{J}
                    = \partial_{i} p^{i} + \tildeCh{i}{j}{i} p^{j} + \tildeCh{i}{\xi}{i} p^{\xi} + \landau(\xi)\\
                   &= \partial_{i} p^{i} + \Ch{i}{j}{i} p^{i} - \tensor{B}{_{i}^{i}} p^{\xi} + \landau(\xi)
                    = \tensor{p}{^{i}_{|i}} + \meanCurvature p^{\xi} + \landau(\xi)\\
                   &= \div{\pb} + \meanCurvature p^{\xi} + \landau(\xi) 
\end{align}
und somit
\begin{align}
  f_{\text{splay}}[\pExt] &= \left( \nabla\cdot\pExt \right)^{2} 
        = \left( \div{\pb} + \meanCurvature p^{\xi} \right)^{2} + \landau(\xi) \formPeriod
\end{align}

\subsection{Twist}

Die Rotation eines Vektorfeldes erhält man durch doppelte Kontraktion des covarianten Levi-Civita-Tensors mit den contravarianten
Ableitungen, d.h.
\begin{align}
  \left[ \nabla\times\pExt \right]_{I} = -\widetilde{E}_{IJK}\pIExt^{J;K}
\end{align}
Somit ergibt sich für die covariante Normalenkomponente der Rotation
\begin{align}
  \left[ \nabla\times\pExt \right]_{\xi} 
        &= -\widetilde{E}_{\xi jk}\pIExt^{j;k}
         = -E_{jk}G^{kL}\tensor{\pIExt}{^{j}_{;L}} + \landau(\xi)\\
        &= -E_{jk}g^{kl}\tensor{p}{^{j}_{;l}} + \landau(\xi)\formPeriod
\end{align}
Mit
\begin{align}
  \tensor{p}{^{j}_{;l}} &= \partial_{l}p^{j} + \tildeCh{l}{I}{j}p^{I}
                         = \partial_{l}p^{j} + \Ch{l}{i}{j}p^{i} + \tildeCh{l}{\xi}{j}p^{\xi}\\
                        &= \tensor{p}{^{j}_{|l}} - \tensor{B}{_{l}^{j}}p^{\xi}
                        \text{ auf }\surf
\end{align}
erhalten wir nun
\begin{align}
  \left[ \nabla\times\pExt \right]_{\xi}
    &= -E_{jk}p^{j|k} + E_{jk}B^{jk}p^{\xi} + \landau(\xi)\\
    &= \rot\pb + \landau(\xi)
                                       \label{eq:rotnorm}
\end{align}
da wegen der Symmetrie des Shape-Operators \(  E_{jk}B^{jk} \) verschwindet.
Für die covarianten Tangentialkomponenten der Rotation erhalten wir
\begin{align}
  \left[ \nabla\times\pExt \right]_{i}
    &= -\widetilde{E}_{iJK}\pIExt^{J;K}
     = -\left( \widetilde{E}_{ij\xi}\pIExt^{j;\xi} +  \widetilde{E}_{i\xi j}\pIExt^{\xi;j}\right) \\
    &= E_{ij}\left(\pIExt^{\xi;j} - \pIExt^{j;\xi}\right) + \landau(\xi)_{i} \formPeriod
\end{align}
Die contravarianten Ableitungen ergeben sich aus
\begin{align}
  \pIExt^{\xi;j} &= G^{jK}\tensor{\pIExt}{^{\xi}_{;K}} = g^{jk}\tensor{p}{^{\xi}_{;k}} +  \landau(\xi)^{j}\\
  \pIExt^{j;\xi} &= G^{\xi \xi}\tensor{\pIExt}{^{j}_{;\xi}} = 0 \text{ (n.V.)}
\end{align}
und
\begin{align}
  \tensor{p}{^{\xi}_{;k}} &= \partial_{k}p^{\xi} + \tildeCh{k}{L}{\xi}p^{L}
                           =\partial_{k}p^{\xi} + \tildeCh{k}{l}{\xi}p^{l}\\
                          &= \left( p^{\xi} \right)_{|k} + B_{kl}p^{l} \text{ auf }\surf \formPeriod
\end{align}
Somit ergeben sich die covarianten Tangentialkomponenten der Rotation zu
\begin{align}
  \left[ \nabla\times\pExt \right]_{i} &= E_{ij}\left( \left( p^{\xi} \right)^{|j} + \tensor{B}{^{j}_{l}}p^{l}\right)  + \landau(\xi)_{i}
                                       \label{eq:rottang}\\
                                       &= -\left[ \Rot p^{\xi} \right]_{i} - \left[ *B\pb \right]_{i} + \landau(\xi)_{i} \formPeriod
\end{align}
Für die Twist-Energiedichte erhalten wir nun
\begin{align}
  f_{\text{twist}}[\pExt] &= \left( p^{i}\left[ \Rot p^{\xi} \right]_{i} + p^{i}\left[ *B\pb \right]_{i}  
                                    - p^{\xi}\rot\pb \right)^{2} + \landau(\xi) \\
                          &= \left( B_{*\pb,\pb} + \left\langle \pb, \Rot p^{\xi} \right\rangle - p^{\xi}\rot\pb \right)^{2}
                              + \landau(\xi)
\end{align}
mit \(  B_{*\pb,\pb} = B(*\pb,\pb) = (*\pb)B\pb = [*\pb]^{i}B_{ij}p^{j} \).
\ingo{Es gilt \( \left\langle \pb, \Rot p^{\xi} \right\rangle = \left\langle *\pb, \Grad p^{\xi} \right\rangle \)
und somit \(  B_{*\pb,\pb} + \left\langle \pb, \Rot p^{\xi} \right\rangle = \left\langle *\pb, B\pb + \Grad p^{\xi} \right\rangle \).}

\subsection{Bend}

Die covarianten Komponenten des Kreuzproduktes von \( \pExt \) mit der Rotation von \( \pExt \) berechnen sich allgemein durch
\begin{align}
  \left[ \pExt\times\left( \nabla \times \pExt \right) \right]_{I}
    &= \widetilde{E}_{IJK} \pIExt^{J} \left[ \nabla \times \pExt \right]^{K} \formPeriod
\end{align}
Mit \eqref{eq:rottang} und der Identität 
\( E_{ik}E_{jl}g^{lk} = E_{ik}\tensor{E}{_{j}^{k}} = g_{ij} \)
ergibt sich für die Normalenkomponente
\begin{align}
  \left[ \pExt\times\left( \nabla \times \pExt \right) \right]_{\xi}
    &= \widetilde{E}_{\xi j k} p^{j} G^{kl} \left[ \nabla \times \pExt \right]_{l} + \landau(\xi) \\
    &= E_{jk}E_{li}g^{kl}p^{j}\left( \tensor{B}{^{i}_{m}}p^{m} + \left( p^{\xi} \right)^{|i} \right) + \landau(\xi) \\
    &= -\left( p^{j}B_{jm}p^{m} + p^{j}\left( p^{\xi} \right)_{|j} \right) + \landau(\xi) \\
    &= -\left( B_{\pb,\pb} + \left\langle \pb, \Grad p^{\xi} \right\rangle\right) + \landau(\xi)
\end{align}
und mit \eqref{eq:rottang} und \eqref{eq:rotnorm} die covarianten Tangentialkomponenten
\begin{align}
  \left[ \pExt\times\left( \nabla \times \pExt \right) \right]_{i}
    &= \widetilde{E}_{i j \xi} p^{j} G^{\xi\xi} \left[ \nabla \times \pExt \right]_{\xi}
        + \widetilde{E}_{i \xi j} p^{\xi} G^{jk} \left[ \nabla \times \pExt \right]_{k} + \landau(\xi)_{i} \\
    &= E_{ij}p^{j}\rot\pb - p^{\xi}E_{ij}E_{kl}g^{jk}\left( \tensor{B}{^{l}_{m}}p^{m} + \left( p^{\xi} \right)^{|l} \right) + \landau(\xi)_{i} \\
    &= -\left[ *p \right]_{i} \rot\pb + p^{\xi}\left( B_{im}p^{m} + \left(p^{\xi} \right)_{|i} \right) + \landau(\xi)_{i}
    \formPeriod
\end{align}
Daraus erhalten wir mit \eqref{eq:norm} die Bend-Energiedichte
\begin{align}
   f_{\text{bend}}[\pExt] &= \left\| \pExt\times\left( \nabla \times \pExt \right) \right\|^{2}
                            + \left[ \pExt\times\left( \nabla \times \pExt \right) \right]_{\xi}^{2} + \landau(\xi) \\
                          &= \left\| -(\rot\pb)(*\pb) + p^{\xi}\left( \Grad p^{\xi} + B\pb \right) \right\|^{2}
                              + \left( B_{\pb,\pb} + \left\langle \pb, \Grad p^{\xi} \right\rangle\right)^{2} + \landau(\xi)
                                \formPeriod
\end{align}

\subsection{Normierung und Tangentialisierung}
\begin{align}
  f_{\text{norm}}[\pExt] &= \left( \left\| p \right\|^{2} + \left( p^{\xi} \right)^{2} - 1 +\landau(\xi) \right)^{2} 
                          = \left( \left\| p \right\|^{2} + \left( p^{\xi} \right)^{2} - 1  \right)^{2}+\landau(\xi)\\
  f_{\text{tan}}[\pExt] &= \left( \pIExt^{\xi} \right)^{2} = \left( p^{\xi} \right)^{2}
\end{align}


\subsection{Zusammenfassung und Spezialfälle}
Mit \autoref{con:Convergence} erhalten wir nun
\begin{align}
    F^{\surf}\left[ \pb, \surf \right] 
        &= \frac{1}{2}\int_{\surf} \Ki f^{\surf}_{\text{splay}}[\pb] + \Kii f^{\surf}_{\text{twist}}[\pb] +  \Kiii
        f^{\surf}_{\text{bend}}[\pb]
                                    +\frac{\Kn}{2}f^{\surf}_{\text{norm}}[\pb] + \Kt f^{\surf}_{\text{tan}}[\pb]\mu
\end{align}
mit
\begin{align}
  f^{\surf}_{\text{splay}}[\pb] &= \left( \div{\pb} + \meanCurvature p^{\xi} \right)^{2} \\
  f^{\surf}_{\text{twist}}[\pb] &= \left( B_{*\pb,\pb} + \left\langle \pb, \Rot p^{\xi} \right\rangle - p^{\xi}\rot\pb \right)^{2}\\
                          &\ingo{= \left(  \left\langle *\pb, B\pb + \Grad p^{\xi} \right\rangle - p^{\xi}\rot\pb \right)^{2}}\\
  f^{\surf}_{\text{bend}}[\pb]  &= \left\| -(\rot\pb)(*\pb) + p^{\xi}\left( \Grad p^{\xi} + B\pb \right) \right\|^{2}
                              + \left( B_{\pb,\pb} + \left\langle \pb, \Grad p^{\xi} \right\rangle\right)^{2}\\
  f^{\surf}_{\text{norm}}[\pb]  &= \left( \left\| \pb \right\|^{2} + \left( p^{\xi} \right)^{2} - 1  \right)^{2} 
                          \ingo{=\left( \left\| \pb \right\|^{2}_{\R^{3}}  - 1  \right)^{2} }\\
  f^{\surf}_{\text{tan}}[\pb]   &= \left( p^{\xi} \right)^{2} \formComma
\end{align}
wobei alle in der Dünnschicht definierten \( f_{i} \) gegen die hier angegebenen Dichten \( f^{\surf}_{i} \) konvergieren.

\ingo{Wir verzichten hier auf eine Diskussion der Art \( K_{i} \leadsto hK_{i} \) (vgl. Napoli) zur Anpassung der
Parameterdimensionierung.}

\ingo{Die Terme, die vom Shape-Operator abhängen, sind auch dem Ansatz \( \pExt_{;\xi}=0 \), bzw. insbesondere
\( \partial_{\xi}\pIExt^{i} = [B\pb]^{i}+ \landau(\xi)^{i} \), für den Paralleltransport geschuldet. 
Könnten die betroffenen Terme somit nicht auch als resultierende Scheinwirkungen des Ansatzes gedeutet werden?
Schließlich beinhaltet dieser Ansatz einen Zwang an \( \pExt \).
Wäre stattdessen \( \tensor{\pIExt}{^{i}_{;\xi}}= [B\pb]^{i} + \landau(\xi)^{i} \), d.h 
\( \partial_{\xi}\pIExt^{i} = 2[B\pb]^{i}+ \landau(\xi)^{i} \),
also \(  \pIExt^{i}(\Xb) =  p^{i}(\xb) + 2\xi\tensor{B}{_{l}^{i}}(\xb)p^{l}(\xb) + \landau(\xi^{2})^{i}\), 
dann würden diese Terme verschwinden (s. Herleitung der Rotation) 
und im speziellen das intrinsische
Modell ohne \( B \)-Terme entstehen. Aber wie wäre dieser Ansatz zu motivieren?
(Um eine Betrachtung der höheren Terme in \( \xi \) für eine Erleuchtung wird man dabei wohl nicht herum kommen.)}

Spezialfälle:
  \subsubsection{Nur Tangentialität ($ \pb^{\xi} = 0 $):}
  \begin{align}
      F^{\surf}\left[ \pb, \surf \right] 
          &= \int_{\surf} \frac{\Ki}{2} \left( \div\pb \right)^{2} 
                                     +\frac{\Kii}{2}  \left(B_{*\pb,\pb}\right)^{2}
                                     +\frac{\Kiii}{2} \left( \left\| \pb \right\|^{2}(\rot\pb)^{2} + \left(B_{\pb,\pb}\right)^{2}\right)
                                     +\frac{\Kn}{4} \left( \left\| \pb \right\|^{2}  - 1  \right)^{2} \mu
  \end{align}
  \paragraph{One-Constant (\( \Ki=\Kii=\Kiii=:\K \)):}
  \begin{align}
      F^{\surf}\left[ \pb, \surf \right] 
          &= \int_{\surf} \frac{\K}{2} \left( \left( \div\pb \right)^{2} 
                                         +\left\| \pb \right\|^{2}\left( (\rot\pb)^{2} + \left\| B\pb \right\|^{2} \right)\right)
                                     +\frac{\Kn}{4} \left( \left\| \pb \right\|^{2}  - 1  \right)^{2} \mu \formComma
  \end{align}
  weil für \( \qb := B\pb = q^{\pb}\pb + q^{*\pb}(*\pb) \) 
  und der \( \left\{ \pb, *\pb \right\} \)-Basis entsprechenden lokalen orthogonalen (Primal-Dual-)Metrik \( \overset{\pb}{g}_{ij} = \left\| \pb \right\|^{2}\delta_{ij} \)
  mit \( i,j\in\left\{ \pb,*\pb \right\}_{\text{(syntaktisch)}} \) folgt, dass
  \begin{align}
    \left(B_{*\pb,\pb}\right)^{2} + \left(B_{\pb,\pb}\right)^{2} 
        &= \left\langle *\pb,\qb \right\rangle^{2} + \left\langle \pb,\qb \right\rangle^{2}
          = \left\| \pb \right\|^{4} \left( q^{\pb} \right)^{2} + \left\| \pb \right\|^{4}\left( q^{*\pb} \right)^{2}  \\
         &= \left\| \pb \right\|^{4} \delta_{ij}q^{i}q^{j}
          = \left\| \pb \right\|^{2}  \overset{\pb}{g}_{ij} q^{i}q^{j}
          = \left\| \pb \right\|^{2} \left\| \qb \right\|^{2} \\
         &= \left\| \pb \right\|^{2} \left\| B\pb \right\|^{2} \formPeriod
  \end{align}

  \ingo{Wie man sieht, stimmt durch die One-Constant-Voraussetzung die Dimensionierung nicht mehr.
        Hier müsste in der entsprechenden Lektüre geprüft werden, inwiefern die Normierung schon in der 3D-Formulierung eingegangen ist.
        Ansonsten wäre ein Ansatz \( \K := \Ki \approx \left\| \pb \right\|^{2}\Kii =  \left\| \pb \right\|^{2}\Kiii \)
        schon von der Dimension der Dichte her viel sinniger als obige Annahme und wir erhalten das ''schönere'' Funktional
        \begin{align}
          F^{\surf}\left[ \pb, \surf \right] 
          &= \int_{\surf} \frac{\K}{2} \left( \left( \div\pb \right)^{2} 
                                         +(\rot\pb)^{2} + \left\| B\pb \right\|^{2} \right)
                                     +\frac{\Kn}{4} \left( \left\| \pb \right\|^{2}  - 1  \right)^{2} \mu \formPeriod
        \end{align}
        }
  
  \subsubsection{One-Constant (\( \Ki=\Kii=\Kiii=:\K \)) allgemein:}
  \begin{align}
      F^{\surf}\left[ \pb, \surf \right] 
          &= \int_{\surf} \frac{\K}{2} \left\{ 
               \left( \div{\pb} + \meanCurvature p^{\xi} \right)^{2} 
                                              +\left( \left\| \pb \right\|^{2} + \left( p^{\xi} \right)^{2} \right) 
                                                 \left((\rot\pb)^{2} + \left\| \nabla p^{\xi} + B\pb \right\|^{2} \right)\right.\\
          &\hspace{50pt}\left. -4p^{\xi}\rot\pb\left\langle *\pb,  \nabla p^{\xi} + B\pb\right\rangle \right\}\\
          &\hspace{50pt} +\frac{\Kn}{4} \left( \left\| \pb \right\|^{2} + \left( p^{\xi} \right)^{2}  - 1  \right)^{2} 
                        + \frac{\Kt}{2}\left( p^{\xi} \right)^{2} \mu \\
          &\ingo{= \int_{\surf} \frac{\K}{2} \left\{ \left( \nabla\cdot\pExt \right)^{2} 
                                  +\left\| \pExt \right\|^{2}_{\R^{3}}\left\| \nabla\times\pExt \right\|^{2}_{\R^{3}}
                             -4p^{\xi}\rot\pb\left\langle *\pb,  \nabla p^{\xi} + B\pb\right\rangle \right\}}\\
          &\ingo{\hspace{50pt} +\frac{\Kn}{4} \left( \left\| \pExt \right\|^{2}_{\R^{3}}  - 1  \right)^{2} 
                        + \frac{\Kt}{2}\left( p^{\xi} \right)^{2} \mu
                             }
  \end{align}
  \ingo{Ich vermute, dass der ''\( -4p^{\xi}\rot\pb\left\langle *\pb,  \nabla p^{\xi} + B\pb\right\rangle \)''-Term,
        der zu gleichen Teilen aus der Twist- und Benddichte kommt, nicht richtig ist und 0 sein müsste.
        Vermutlich ein Vorzeichenfehler, den ich aber gerade nicht finden kann.}
    



\end{document}
