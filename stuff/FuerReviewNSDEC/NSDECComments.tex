\documentclass[a4paper,11pt]{scrartcl}

\usepackage{ngerman}
\usepackage[utf8]{inputenc}

\usepackage{amsmath}
\usepackage{amsthm}
\usepackage{amsfonts}
\usepackage{amssymb}
\usepackage{amsbsy}


\usepackage{tensor}

\usepackage{tikz}

\usepackage{hyperref}
\renewcaptionname{ngerman}{\figurename}{Abb.}
\newcaptionname{ngerman}{\figureautorefname}{Abb.}

%Bezeichner
\newcommand{\U}{u} %Komponenten des Geschwindigkeitfeldes
\newcommand{\Ub}{\mathbf{\U}} %Geschwindigkeitsfeld
\newcommand{\tU}{\tilde{u}} %Komponenten des Geschwindigkeitfeldes
\newcommand{\tUb}{\mathbf{\tU}} %Geschwindigkeitsfeld
\newcommand{\V}{v} %Komponenten des Geschwindigkeitfeldes
\newcommand{\Vb}{\mathbf{\V}} %Geschwindigkeitsfeld
\renewcommand{\P}{p} %Druck
\newcommand{\g}{\mathbf{g}} % metrischer Tensor, Oberflaechenidentitaet
\newcommand{\lc}{\mathbf{E}} % Levi-Civita-Tensor
\newcommand{\gauss}{\mathcal{K}} % Gauss-Kruemmung
\newcommand{\ekin}{\textup{E}_{\textup{Kin}}} %kinetische Energie
\newcommand{\dtekin}{\dot{\textup{E}}_{\textup{Kin}}} %kinetische Energie
\newcommand{\landau}{\mathcal{O}}

% General
\newcommand{\vect}[1]{\mathbf{#1}}


%Raeume
\newcommand{\surf}{\mathcal{S}} %Oberflaeche
\newcommand{\uspace}{\mathcal{T}^{(1)}(\surf)} % Typ 1 Tensor; Co- bzw. Contravarintes Tangentialbuendel
\newcommand{\Tangent}{\mathsf{T}}
\newcommand{\R}{\mathbb{R}}

%Operatoren
\renewcommand{\div}{\operatorname{div}} %Divergenzoperator
\newcommand{\rot}{\operatorname{rot}} %Rotationsoperator
\newcommand{\lie}{\mathcal{L}} %Lie-Ableitung
\newcommand{\exd}{\mathbf{d}} %Aeussere Ableitung
\newcommand{\lrr}{\Delta^{\textup{Rr}}} %Rot-rot-Laplace
\newcommand{\ldg}{\Delta^{\textup{dG}}} %div-Grad-Laplace
\newcommand{\lgd}{\Delta^{\textup{Gd}}} %Grad-div-Laplace
\newcommand{\ldr}{\Delta^{\textup{DR}}} %Laplace-deRham
\newcommand{\jup}{\textup{j}} % teilweises inneres Produkt

% DEC declarations
\newcommand{\SC}{\mathcal{K}} % simplicial complex
\newcommand{\Vs}{\mathcal{V}} % set of vertices
\newcommand{\Es}{\mathcal{E}} % set of edges
\newcommand{\Fs}{\mathcal{T}} % set of faces
\newcommand{\face}{T} % one face
\newcommand{\FormSpace}{\Lambda^{1}} % space of 1 forms
\newcommand{\flatgb}[2]{\overset{#1 #2}{\g}}
\newcommand{\flatg}[2]{\overset{#1 #2}{g}}

\newcommand{\facesim}{\overset{\face}{\sim}}


%Zeug
\newcommand{\formComma}{\,\text{,}}
\newcommand{\formPeriod}{\,\text{.}}
\newcommand{\ingo}[1]{{\color{blue}#1}}



\title{Erläuterungen zum NSDEC-Paper} 
\author{Ingo Nitschke}

\begin{document}
\maketitle
%\tableofcontents

\section{Konsistenz zeitdiskreter Advektion}

Wir bedienen uns im Folgenden der Lagrange-Darstellung des Fluids auf der Oberfläche.
D.h. die Wahl lokaler Koordinaten soll weder von der Strömung \( \Ub \) noch von der Zeit \( t \) abhängen.
Somit gilt insbesondere \( \partial_{t}\g = 0 \).
Folglich sind kovariante Ortsableitungen und die Zeitableitung zu einander kommutativ in der Reihenfolge der Verkettung.
Wir bezeichen mit \( \Ub = \Ub(t) = \Ub_{k+1} \) das Geschwindigkeits(-1-Form-)feld zur aktuellen Zeit \( t =t_{k+1}\)
und \( \tUb = \Ub(t-\tau) =\Ub_{k}\) das Geschwindigkeitsfeld zur vergangenen Zeit \( t-\tau = t_{k} \).
Tayler-Approximation vorwärts in der Zeit, für die relevanten Größen, liefert
\begin{align}
  \left[ \tUb \right]^{j} &= \tU^{j} = \U^{j} - \tau\partial_{t}\U^{j} + \landau\left( \tau^{2} \right) \\
  \left[ \nabla\tUb \right]_{ij} &= \tU_{i|j} = \U_{i|j} - \tau\partial_{t}\U_{i|j} + \landau\left( \tau^{2} \right) \formPeriod
\end{align}
Somit erhalten wir die Entwicklungen
\begin{align}
  \left[ \nabla_{\Ub}\tUb \right]_{i}
        &=\U^{j}\tU_{i|j} = \U^{j}\U_{i|j} - \tau\U^{j}\partial_{t}\U_{i|j} + \landau\left( \tau^{2} \right) \\
        &= \left[ \nabla_{\Ub}\Ub - \tau\nabla_{\Ub}\partial_{t}\Ub \right]_{i} + \landau\left( \tau^{2} \right) \\
  \left[ \nabla_{\tUb}\Ub \right]_{i}
        &= \tU^{j}\U_{i|j} = \U^{j}\U_{i|j} - \tau\left(\partial_{t}\U^{j}\right)\U_{i|j} + \landau\left( \tau^{2} \right) \\
        &= \left[ \nabla_{\Ub}\Ub - \tau\nabla_{\partial_{t}\Ub}\Ub \right]_{i} + \landau\left( \tau^{2} \right) \\
  \left[ \nabla_{\tUb}\tUb \right]_{i}
        &= \tU^{j}\tU_{i|j} = \left( \U^{j} - \tau\partial_{t}\U^{j} \right)\left(  \U_{i|j} - \tau\partial_{t}\U_{i|j} \right) + \landau\left( \tau^{2} \right)\\
        &= \U^{j}\U_{i|j} - \tau \left( \U^{j}\partial_{t}\U_{i|j} + \left(\partial_{t}\U^{j}\right)\U_{i|j} \right) + \landau\left( \tau^{2} \right) \\
        &= \left[ \nabla_{\Ub}\Ub - \tau \left( \nabla_{\Ub}\partial_{t}\Ub + \nabla_{\partial_{t}\Ub}\Ub \right) \right]_{i} + \landau\left( \tau^{2} \right)
\end{align}
Für den linearisierten Advektionsterm erhalten wir nun
\begin{align}\label{eq:adlin}
  \nabla_{\Ub}\tUb + \nabla_{\tUb}\Ub - \nabla_{\tUb}\tUb 
        &=  \nabla_{\Ub}\Ub + \landau\left( \tau^{2} \right)
\end{align}
und somit eine Konsistenzordnung von 2 in der Zeit.

In ''NAVIER–STOKES EQUATIONS IN ROTATION FORM: A ROBUST
MULTIGRID SOLVER FOR THE VELOCITY PROBLEM'' (2002),  by M. A. Olshanskii
A. Reusken, findet effektiv nur der erste Term Beachtung\footnote{\( \nabla_{\Ub}\Ub = \rot(\Ub)(*\Ub) + \frac{1}{2}\exd\left\| \Ub \right\|^{2} \),
wobei \( (\rot\Ub)(*\Ub) = \left( \text{curl}\Ub \right)\times\Ub \) im flachen Fall und die exakte Form \( \frac{1}{2}\exd\left\| \Ub \right\|^{2} \)
lässt sich in den generalisierten Druck schieben.
Linearisiert wird der Term \( \left( \text{curl}\Ub \right)\times\Ub \) zu \( \left( \text{curl}\tUb \right)\times\Ub \).},
d.h.
\begin{align}
  \nabla_{\Ub}\tUb &= \nabla_{\Ub}\Ub + \landau\left( \tau \right)\formComma
\end{align}
was nur eine Konsistenzordnung von 1 in der Zeit bringt.

Um eine diskrete Lie-Ableitung, wie in ''Discrete Lie Advection of
Differential Forms'' (2011), by P. Mullen, A. McKenzie , D. Pavlov, L.
Durant, Y. Tong, E. Kanso, J.E. Marsden, and M. Desbrun,
sinnvoll nutzen zu können, muss die zeitdiskrete Advektion ihre Symmetrieeigenschaft beibehalten, 
da es sonst zu viele zusätzliche (antisymmetrische) Terme gibt.
Gemeint ist, dass zwar
\begin{align}\label{eq:liesimple}
  \nabla_{\Ub}\Ub &= \lie_{\Ub^{\sharp}}\Ub - \frac{1}{2}\exd\left\| \Ub \right\|^{2}
\end{align}
gilt, aber allgemeiner auch
\begin{align}\label{eq:liefull}
  \nabla_{\Ub}\tUb &= \lie_{\Ub^{\sharp}}\tUb - \frac{1}{2}\left\langle \Ub , \tUb \right\rangle
                          +\frac{1}{2}\left( \left( \rot\Ub \right)\left( *\tUb \right) -  \left( \rot\tUb \right)\left( *\Ub \right)
                          +\left( \div\tUb \right)\Ub - \left( \div\Ub \right)\tUb
                          -*\exd\left\langle \Ub, *\tUb \right\rangle\right)\formComma
\end{align}
was zu der schon schlechteren zeitlichen Konsistenz auch keinen Gewinn in der Ortsdiskretisierung ergeben würde.
Deshalb könnte man gleich die Lie-Advection der rechten Seite von \eqref{eq:liesimple} linearisieren, 
was uns zu  (analog zu oben)
\begin{align}
  \nabla_{\Ub}\Ub &= \lie_{\tUb^{\sharp}}\Ub + \lie_{\Ub^{\sharp}}\tUb - \lie_{\tUb^{\sharp}}\tUb - \frac{1}{2}\exd\left\| \Ub \right\|^{2} + \landau\left( \tau^{2} \right)
\end{align}
führt.
Oder wir setzen \eqref{eq:liefull} in die schon linearisierte Advektion \eqref{eq:adlin} ein.
Wegen der Symmetrie der Form \eqref{eq:adlin} in \( \Ub \) und \( \tUb \), verschwinden die zusätzlichen Terme und wir erhalten
\begin{align}
  \nabla_{\Ub}\Ub &= \lie_{\tUb^{\sharp}}\Ub + \lie_{\Ub^{\sharp}}\tUb - \lie_{\tUb^{\sharp}}\tUb + \frac{1}{2}\exd\left( \left\| \tUb \right\|^{2} - 2\left\langle \Ub, \tUb \right\rangle \right)
                      + \landau\left( \tau^{2} \right) \formPeriod
\end{align}
Beide Varianten unterscheiden sich augenscheinlich nur um eine exakte 1-Form, welche sich wieder in gewohnter Weise in den generalisierten Druck überführen lässt.
Genauer ist es sogar so, dass 
\begin{align}
  \frac{1}{2}\exd\left( \left\| \tUb \right\|^{2} - 2\left\langle \Ub, \tUb \right\rangle \right) + \frac{1}{2}\exd\left\| \Ub \right\|^{2} 
         &= \frac{1}{2}\exd\left\| \Ub-\tUb \right\|^{2} = \landau\left( \tau^{2} \right)
\end{align}
gilt und sich beide Varianten nur um \( \landau\left( \tau^{2} \right) \) unterscheiden.

Es sei außerdem darauf hingewiesen, dass in 'Discrete Lie Advection of
Differential Forms'' zwar ein allgemeines Vorgehen angegeben ist, aber explizit
ist die Diskrete Lie-Ableitung hier nur für flache uniforme Vierecksgitter gegeben.

\end{document}
