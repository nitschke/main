\documentclass{scrartcl}

\usepackage[english]{babel}
\usepackage[utf8]{inputenc}

\usepackage{amsmath}
\usepackage{amssymb}

\newcommand{\exd}{\mathbf{d}}
\newcommand{\Div}{\text{Div}}
\newcommand{\Rot}{\text{Rot}}
\newcommand{\vecflat}[1]{\vec{#1}^{\,\flat}}
\renewcommand{\i}{\mathbf{i}}
\renewcommand{\L}{\mathbf{\mathcal{L}}}


\begin{document}

Frank-Oseen energy density:
\begin{align}
  e[\vec{p}] = \frac{K_{0}}{2}\left( \left\| \Rot\vec{p} \right\|^{2} +  \left\| \Div\vec{p} \right\|^{2}\right)
\end{align}
with \( \left\| \vec{p} \right\| = 1 \).

\( *\vec{p} := \left( *\vecflat{p} \right)^{\sharp} \) is the Hodge dual of \( \vec{p} \), i.e. \( \vec{p}\bot (*\vec{p}) \) 
and  \(\left\| *\vec{p} \right\| = \left\| \vec{p} \right\| = 1 \).
\begin{align}
  \vec{q} := \cos\phi\vec{p} + \sin\phi (*\vec{p})
\end{align}
is a length preserving linear combination of the orthonormal system  \( \left\{\vec{p}, *\vec{p}\right\} \), i.e. \( \left\| \vec{q} \right\| = 1 \),
with a (space-)constant rotation angle \( \phi \), i.e. \( \exd\phi=0 \).
Straight forward calculations implies
\begin{align}
   \left\| \Rot (*\vec{p}) \right\| &=  \left\| *\exd*\vecflat{p} \right\| = \left\| \Div\vec{p} \right\|\\
   \left\| \Div (*\vec{p}) \right\| &= \left\| *\exd**\vecflat{p}  \right\| = \left\| *\exd\vecflat{p}  \right\| = \left\|\Rot\vec{p}\right\|\\ 
   \left\| \Rot\vec{q} \right\|^{2} &= \left\| *\exd\vecflat{q} \right\|^{2} = \left\| \exd\vecflat{q} \right\|^{2} \\
                                    &=  \cos^{2}\phi\left\| \Rot\vec{p} \right\|^{2} 
                                      + \sin^{2}\phi\left\| \Div\vec{p} \right\|^{2}
                                      + 2\cos\phi\sin\phi\left\langle \exd\vecflat{p}, \exd * \vecflat{p} \right\rangle\\
  \left\| \Div\vec{q} \right\|^{2} &= \left\| *\exd*\vecflat{q} \right\|^{2} = \left\| \exd*\vecflat{q} \right\|^{2}\\
                                   &= \cos^{2}\phi\left\| \Div\vec{p} \right\|^{2} 
                                      + \sin^{2}\phi\left\| \Rot\vec{p} \right\|^{2}
                                      - 2\cos\phi\sin\phi\left\langle \exd\vecflat{p}, \exd * \vecflat{p} \right\rangle
\end{align}
Finally we get
\begin{align}
  e[\vec{q}] = \frac{K_{0}}{2}\left( \left\| \Rot\vec{q} \right\|^{2} +  \left\| \Div\vec{q} \right\|^{2}\right)
             = \frac{K_{0}}{2}\left( \left\| \Rot\vec{p} \right\|^{2} +  \left\| \Div\vec{p} \right\|^{2}\right)
             = e[\vec{p}]
\end{align}

  
\end{document}
