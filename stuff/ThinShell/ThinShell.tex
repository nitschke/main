\documentclass[a4paper,11pt]{scrartcl}

\usepackage[english]{babel}
\usepackage[utf8]{inputenc}

\usepackage{hyperref}
\usepackage{color}

\usepackage{tensor}

\usepackage{amsmath}
\usepackage{amssymb}
\usepackage{amsfonts}
\usepackage{amsthm}

\usepackage{bm}

\usepackage{hyperref}

\newcommand{\surf}{\mathcal{S}}
\newcommand{\surfh}{\surf_{h}}
\newcommand{\R}{\mathbb{R}}

\newcommand{\tangentspace}{T}
\newcommand{\tensorspace}{\mathcal{T}}

\newcommand{\Tr}{\text{Tr}}

\newcommand{\landau}{\mathcal{O}}
\newcommand{\adj}{\text{adj}}

\renewcommand{\Im}{\text{Im}}

\newcommand{\xb}{\mathbf{x}}
\newcommand{\txb}{\tilde{\xb}}
\newcommand{\nub}{\bm{\nu}}
\newcommand{\gb}{\mathbf{g}}
\newcommand{\tgb}{\tilde{\gb}}
\newcommand{\tg}{\tilde{g}}
\newcommand{\tb}{\mathbf{t}}
\newcommand{\Bb}{\mathbf{B}}
\newcommand{\alphab}{\bm{\alpha}}
\newcommand{\deltab}{\bm{\delta}}

\newcommand{\ttt}{\tilde{t}}

\newcommand{\tch}[2]{\widetilde{\Gamma}_{#1}^{#2}}
\newcommand{\ch}[2]{\Gamma_{#1}^{#2}}

\newcommand{\meanc}{\mathcal{H}}
\newcommand{\gaussc}{\mathcal{K}}

\newcommand{\fraki}{\mathfrak{i}}
\newcommand{\frakk}{\mathfrak{k}}
\newcommand{\frakl}{\mathfrak{l}}

\newcommand{\formComma}{\,\text{,}}
\newcommand{\formPeriod}{\,\text{.}}
\newcommand{\ie}{i.\,e.}%
\newcommand{\eg}{e.\,g.}

\newcommand{\newterm}[1]{\textbf{\textit{#1}}}

\newtheorem{theorem}{Theorem}
\newtheorem{conclusion}{Conclusion}

\title{Thin Shell Stuff} 
\author{Ingo Nitschke}


\begin{document}
\maketitle
\tableofcontents

\section{Metric Quantities}
In the following, we consider a \newterm{thin shell} of constant thickness \( h\in\R \) around an oriented, boundarieless, compact Riemannian 2-manifold (surface) \( \surf \)
defined by
\begin{align}
 \surfh := \surf\times\left[ -\frac{h}{2}, \frac{h}{2} \right] \subset \R^{3} \formPeriod
\end{align}
Constant thickness means, that the orthogonal measurement of the  two disjoint boundaries \( \partial^{+}\surfh\sqcup\partial^{-}\surfh = \partial\surfh \) is \( h \)
at all boundary points.
Thereby, be \( h \) small enough, so that  \( \surfh\subset \R^{3} \) contains no overlaps,
\ie, it exists a surjection \( \surfh \twoheadrightarrow \surf \).

\subsection{Coordinates}
We define the coordinate in normal direction \( \nub \) of the surface \( \surf \) by \( \xi\in\left[ -\frac{h}{2}, \frac{h}{2} \right] \).
If we use any choice of local coordinates \( (u,v)\in U \) of the surface, so that the immersion 
\( \xb:U \rightarrowtail \R^{3} \) parameterize \( \surf= \Im(\xb)\),
then we can define the immersion \( \txb: U\times\left[ -\frac{h}{2}, \frac{h}{2} \right] \rightarrowtail \R^{3} \) with  \( \surfh= \Im(\txb)\) by
\begin{align}
  \txb &= \txb(u,v,\xi) := \xb(u,v) + \xi\nub(u,v) = \xb + \xi\nub\formPeriod 
\end{align}

\subsection{Arrangements}
\begin{description}
  \item[Lowercase letters] \( i,j,k,\ldots \) are used as index for \( u \) and \( v \), \eg, \( \alpha_{i}dx^{i} \) is an 1-form in \( T^{*}\surf \).
  \item[Uppercase letters] \( I,J,K,\ldots \)  are used for \( u \), \( v \) and \( \xi \), \eg, 
  \( \tilde{\alpha}^{I}\partial_{I}\txb = \tilde{\alpha}^{i}\partial_{i}\txb +\tilde{\alpha}^{\xi}\partial_{\xi}\txb \) is a (contravariant) vector in \( T\surfh \).
  \item[The Tilde] are used for quantities and relations in context of \( \surfh \),
        \eg, \(  \tilde{\alphab}\in T\surfh \) but \( \alphab\in  T\surf\) and we can construct a relation \( \tilde{\alphab} = \alphab + \alpha^{\xi}\nub \).
  \item[Full covariant descriptions] (lower indices) are always used, unless otherwise is defined,
        \eg, \( \Bb = \left\{ B_{ij} \right\} \) is the full covariant shape operator, \ie, the second fundamental form in this representation.
  \item[Indexing and collector brackets], \( \left[  \right] \) and \( \left\{  \right\} \), are used to switch between components and object representations, \eg,
      \( \left[ \tb \right]_{ij} = t_{ij} \) and \( \left\{ t_{ij} \right\} = \tb \).
  \item[Sharp and flat operator on tensors] are generalisations of the usual flat and sharp operator on vector valued quantities and can be realized by matrix multiplications with the
  metric tensor \( \gb \) and its inverse \( \gb^{-1} \),
          \eg, \( \tensor[^\flat]{\left\{ \tensor{t}{^{i}_{j}} \right\}}{^{\sharp}} = \gb\left\{ \tensor{t}{^{i}_{j}} \right\}\gb^{-1} 
                  =  \left\{ \tensor{t}{_{i}^{j}} \right\} = \tb^{\sharp} \).
  \item[Tensor product] means always the contraction of the last component of a tensor with the first of another tensor,
          \eg, \( \left[ \bm{s}\tb \right]_{ij} = \tensor{s}{_{i}^{k}}t_{kj} \), \eg, with an usual matrix product \( \cdot \), 
            this implies \( \bm{s}\tb = \bm{s}\cdot\gb^{-1}\cdot\tb \).
\end{description}

\subsection{The Metric Tensor and Shape Operator}
With an arbitrary choice of surface coordinates \( (u,v)\in U \), we can calculate the canonical basic vectors \( \partial_{I}\txb\in T\surfh \) by
\begin{align}
  \partial_{i}\txb &= \partial_{i}\xb + \xi\partial_{i}\nub \\
  \partial_{\xi}\txb &= \nub \formPeriod
\end{align}

The \newterm{metric tensor} (\newterm{first fundamental form}) of thin shell is given by its components \( \tg_{IJ} = \left\langle \partial_{I}\txb , \partial_{J}\txb \right\rangle_{\surfh} \).
Therefore, for the mixed tangential-normal components holds \( \tg_{i\xi} = \tg_{\xi i} = 0 \), 
because 
\begin{align}
 \left\langle \partial_{i}\nub , \nub \right\rangle_{\surfh} = \frac{1}{2}\partial_{i}\left\| \nub \right\|^{2}_{\surfh} = 0 \formPeriod
\end{align}
For the pure normal component, we obtain \( \tg_{\xi\xi} = \left\| \nub \right\|^{2}_{\surfh} = 1 \),
\ie, the co- and contravariant normal components of a tensor quantity are equivalently,
\eg, (detailed)
\begin{align}
 \tensor{\ttt}{^{I}_{\xi J}} &= \tg_{\xi K} \tensor{\ttt}{^{IK}_{J}} 
                          = \tg_{\xi k} \tensor{\ttt}{^{Ik}_{J}} + \tg_{\xi \xi} \tensor{\ttt}{^{I\xi}_{J}}
                          = \tensor{\ttt}{^{I\xi}_{J}} \formPeriod
\end{align}
For the pure tangential components, we get a second degree tensor polynomial in \( \xi \)
\begin{align}
  \tg_{ij} &= g_{ij} - 2\xi B_{ij} + \xi^{2}\left[ \Bb^{2} \right]_{ij}
            = \left[ \left( g - \xi\Bb \right)^{2} \right]_{ij}\\
           &= g_{ij} - 2\xi B_{ij} + \xi^{2}\left( \meanc B_{ij} - \gaussc g_{ij} \right) \formPeriod
\end{align}
where the \newterm{covariant shape operator} (\newterm{second fundamental form}) is given by 
\begin{align}
 B_{ij} = -\left\langle \partial_{i}\xb , \partial_{j}\nub  \right\rangle_{\surfh} 
\end{align}
and the \newterm{third fundamental form} by
\begin{align}
 \left[ \Bb^{2} \right]_{ij} &= \left\langle \partial_{i}\nub , \partial_{j}\nub  \right\rangle_{\surfh}\formComma
\end{align}
see \cite{HartmanWintner1953}.
\( \gaussc = \left| \Bb^{\sharp} \right| \) is the \newterm{Gaussian curvature} and \( \meanc := \Tr\Bb = \tensor{B}{^{i}_{i}} \) the \newterm{mean curvature}.
(In a more differential geometrical context on surfaces, this is minus twice the mean curvature.)
A more classical representation of the third fundamental form \( \Bb^{2} \) is
\begin{align}\label{eq:shapesquare}
  \Bb^{2} = \meanc\Bb - \gaussc\gb\formPeriod
\end{align}

\begin{theorem}
  For the inverse thin shell metric \( \tgb^{-1} \) holds
  \begin{align}
    \tg^{ij} &= \left( g^{ik} + \sum_{\frakl = 1}^{\infty} \xi^{\frakl}\left[ \Bb^{\frakl} \right]^{ik} \right)
               \left( \delta_{k}^{j} + \sum_{\frakk = 1}^{\infty} \xi^{\frakk}\tensor{\left[ \Bb^{\frakk} \right]}{_{k}^{j}} \right) \\
             &= \left[\left( \gb + \sum_{\frakk = 1}^{\infty} \xi^{\frakk} \Bb^{\frakk} \right)^{2}\right]^{ij}\formComma\\
    \tg^{\xi\xi} &= 1 \formComma\\
    \tg^{i\xi} &= \tg^{\xi i} = 0 \formPeriod
  \end{align}
\end{theorem}
\begin{proof}
  First we define the pure tangential components of the thin shell metric tensor as \( \tgb_{t} := \left\{ \tg_{ij} \right\} \).
  With \( \deltab = \left\{ \delta^{i}_{j} \right\} \) the Kronecker delta, we can write down in usual matrix notation
  \begin{align}
    \tgb\cdot\tgb^{-1} &=
      \begin{bmatrix}
        \tgb_{t} & O \\
          O      & 1
      \end{bmatrix}
      \cdot
      \begin{bmatrix}
         \left\{ \tg^{ij} \right\} & \left\{ \tg^{i\xi} \right\} \\
         \left\{ \tg^{\xi i} \right\} & \tg^{\xi\xi}
      \end{bmatrix}
          =
             \begin{bmatrix}
                \deltab & O \\
                O       & 1
             \end{bmatrix} \formPeriod
  \end{align}
  Thus, we obtain
  \begin{align}
    \tg^{\xi\xi} &= 1 \formComma\\
    \tg^{i\xi} &= \tg^{\xi i} = 0 \formComma\\
    \left\{ \tg^{ij} \right\} &= \tgb_{t}^{-1}
                         =\left( \gb - \xi\Bb \right)^{-2}
                         = \left( \gb - \xi\Bb \right)^{-1} \cdot \left( \deltab - \xi\Bb^{\sharp} \right)^{-1}\formPeriod
  \end{align}
  For \( h \) small enough, so that \( \xi\left\| \Bb \right\| \le h\left\| \Bb \right\| < 1 \) and exponent with a dot indicate matrix (endomorphism) power, we can use the Neumann serie 
  \begin{align}
    \left( \deltab - \xi\Bb^{\sharp} \right)^{-1} = \deltab + \sum_{\frakk = 1}^{\infty}\xi^{\frakk}\left( \Bb^{\sharp} \right)^{\cdot\frakk} \formComma
  \end{align}
  and therefore the assertion, 
  because 
  %for a linear idempotent invertible operator \( F \), like the left or right sharp or flat operator, and \( \frakk\in\mathbb{Z} \) is
  %\begin{align}
  %  \left( F[t] \right)^{l} &= F^{l}[t^{l}] = F[t^{l}]
  %\end{align}
  %valid on a tensor \( t \) (\( \gb \) is a homomorphism with respect to tensor product).
  with \( \Bb^{\frakk}  = \left( \Bb\cdot\gb^{-1} \right)^{\cdot\frakk} \cdot \gb\) we get
  \begin{align}
    \left( \Bb^{\sharp} \right)^{\cdot\frakk} &= \left( \Bb\cdot\gb^{-1} \right)^{\cdot\frakk} = \Bb^{\frakk}\cdot\gb^{-1} = \left( \Bb^{\frakk} \right)^{\sharp}
  \end{align}
  and
  \begin{align}
    \left( \gb - \xi\Bb \right)^{-1} 
          &= \left( \left( \deltab - \xi\Bb^{\sharp} \right)\cdot \gb \right)^{-1}
          = \gb^{-1} \cdot \left( \deltab - \xi\Bb^{\sharp} \right)^{-1}
          = \tensor[^{\sharp}]{\left( \deltab - \xi\Bb^{\sharp} \right)}{^{-1}} \formPeriod
  \end{align}
\end{proof}
Therefore, we get for \( \tgb_{t}^{-1} \) a polynomial in \( \xi\Bb \) and with successively applying \eqref{eq:shapesquare},\ie,
\( \Bb^{\frakk} = \meanc\Bb^{\frakk-1} - \gaussc\Bb^{\frakk-2} \), we can always find
polynomials \( p \) and \( q \) in \( \gaussc \), \( \meanc \) and \( \xi \), so that holds
\( \tgb_{t}^{-1} = p(\gaussc, \meanc, \xi)\gb^{-1} + q(\gaussc, \meanc, \xi)\tensor[^{\sharp}]{\Bb}{^{\sharp}} \).
We will not carry this out in full generality, but let us mention some developments in \( \xi \).
\begin{conclusion}
  The developments of \( \tgb_{t}^{-1} \) up to second degree in \( \xi \) are
  \begin{align}
    \tgb_{t}^{-1} &= \gb^{-1} + \landau(\xi)\formComma\\
    \tgb_{t}^{-1} &= \gb^{-1} + 2\xi\tensor[^{\sharp}]{\Bb}{^{\sharp}} + \landau(\xi^{2})\formComma \\
    \tgb_{t}^{-1} &= \gb^{-1} + 2\xi\tensor[^{\sharp}]{\Bb}{^{\sharp}} +  3\xi^{2}\tensor[^{\sharp}]{\left( \Bb^{2} \right)}{^{\sharp}} + \landau(\xi^{3})\\
                  &= \left( 1 - 3 \xi^{2}\gaussc \right)\gb^{-1} + \xi\left( 2 + 3\xi\meanc \right)\tensor[^{\sharp}]{\Bb}{^{\sharp}} + \landau(\xi^{3}) \formPeriod
  \end{align}
\end{conclusion}


\subsubsection{The Volume Element}
  To develop the thin shell volume element \( \tilde{\mu} \) in normal direction at the surface volume element \( \mu \), 
  we need a development of the determinant of the metric tensor \( \tgb \).
  \begin{theorem}
    For the determinant of the thin shell metric tensor \( \left| \tgb \right| \) holds
    \begin{align}
       \left| \tgb \right| &= \left( 1 - \xi\meanc + \xi^{2}\gaussc \right)^{2}\left| \gb \right|\formComma
    \end{align}
  \end{theorem}
  \begin{proof}
    The mixed components are zero, so we get
    \begin{align}
      \left| \tgb \right| &= \tg_{\xi\xi}\left| \tgb_{t} \right| = \left| \tgb_{t} \right|\formPeriod
    \end{align}
    Now, we define \( \sqrt{\tgb_{t}^{\sharp}}   := \left(\gb - \xi \Bb\right)^{\sharp} \) as a square root of \( \tgb_{t}^{\sharp} \),
    because
    \begin{align}
      \tgb_{t}^{\sharp} &= \left( \left( \gb -\xi\Bb \right)^{2} \right)^{\sharp}
              = \left( \left( \gb -\xi\Bb \right)^{\sharp}\left( \gb -\xi\Bb \right)  \right)^{\sharp}
              = \left( \gb -\xi\Bb \right)^{\sharp}\left( \gb -\xi\Bb \right)^{\sharp} 
              = \left( \sqrt{\tgb_{t}^{\sharp}} \right)^{2} \formPeriod
    \end{align}
    Hence, we can calculate
    \begin{align}
      \left| \tgb \right| &= \left| \tgb_{t} \right| = \left| \tgb_{t}^{\sharp}\gb \right|
                    =\left| \tgb_{t}^{\sharp} \right| \left| \gb \right|
                    =\left| \sqrt{\tgb_{t}^{\sharp}} \right|^{2} \left| \gb \right| \formPeriod
    \end{align}
    For the determinant of \( \sqrt{\tgb_{t}^{\sharp}} \), we regard that \( \gb^{\sharp} \) is the Kronecker delta,
    so we obtain
    \begin{align}
       \left| \sqrt{\tgb_{t}^{\sharp}} \right| 
            &= \left| \gb^{\sharp} - \xi\Bb^{\sharp} \right|
             = \left( 1 - \xi\tensor{B}{_{u}^{u}} \right)\left( 1 - \xi\tensor{B}{_{v}^{v}} \right)
                  - \xi^{2}\tensor{B}{_{u}^{v}}\tensor{B}{_{v}^{u}} \\
            &= 1 - \xi\left( \tensor{B}{_{u}^{u}} + \tensor{B}{_{v}^{v}} \right) 
                  + \xi^{2}\left( \tensor{B}{_{u}^{u}} \tensor{B}{_{v}^{v}} - \tensor{B}{_{u}^{v}}\tensor{B}{_{v}^{u}} \right)
             =\left( 1 - \xi\meanc + \xi^{2}\gaussc \right) \formPeriod
    \end{align}
  \end{proof}
  Therefore a representation of the \newterm{thin shell volume element} \( \tilde{\mu} \), depending on the surface volume element \( \mu \), is
  \begin{align}
    \tilde{\mu} &= \sqrt{\left| \tgb \right|} du\wedge dv \wedge d\xi
            = \left(  1 - \xi\meanc + \xi^{2}\gaussc\right)  \mu \wedge d\xi \\
                &= \left(  1 - \xi\meanc + \xi^{2}\gaussc\right)  d\xi \wedge \mu \formPeriod
  \end{align}

\subsubsection{The Levi-Civita Tensor and Hodge-Dualism}

\subsection{Christoffel Symbols}
  The \newterm{Christoffel symbols} are needed to define an unique metric compatible derivation (Levi-Civita connection).
  With a choice of coordinates, the christoffel symbols (of second kind) on the thin shell are
  \begin{align}
    \tch{IJ}{K} &= \frac{1}{2} \tg^{KL}\left( \partial_{I}\tg_{JL} + \partial_{J}\tg_{IL} - \partial_{L}\tg_{IJ} \right)\formPeriod
  \end{align}
  On the surface \( \surf \), they are equal defined, just omit the tilde and use lowercase letters for indexing.
  \begin{theorem}
    With \( \tensor{\beta}{_{ij}^{k}} := \tensor{B}{_{i}^{k}_{|j}} +  \tensor{B}{_{j}^{k}_{|i}} - \tensor{B}{_{ij}^{|k}}\)
    the second order expansions in normal direction of Christoffel symbols are
    \begin{align}
      \tch{ij}{k} &= \ch{ij}{k} - \xi\tensor{\beta}{_{ij}^{k}} + \landau(\xi^{2})\\
      \tch{ij}{\xi} &= B_{ij} - \xi \left[ \Bb^{2} \right]_{ij} 
                          &= \left( 1 - \xi\meanc \right)B_{ij} + \xi\gaussc g_{ij}\\
      \tch{i\xi}{k} = \tch{\xi i}{k} &= -\tensor{B}{_{i}^{k}} - \xi\tensor{\left[ \Bb^{2} \right]}{_{i}^{k}} + \landau(\xi^{2})
                                        &= -\left( 1 + \xi\meanc \right)\tensor{B}{_{i}^{k}} + \xi\gaussc\delta_{i}^{k} + \landau(\xi^{2})\\
      \tch{\xi\xi}{K} &= 0 \\
      \tch{I\xi}{\xi} = \tch{\xi I}{\xi} &= 0 \formPeriod
    \end{align}
  \end{theorem}
  \begin{proof}
    Properties of the thin shell metric \( \tgb \) are the mixed tangential-normal components are zero (the same holds for the inverse metric) and the pure normal component is constant.
    Hence, we get
    \begin{align}
      \tch{\xi\xi}{K} &= \frac{1}{2}\tg^{KL}\left( \partial_{\xi}\tg_{\xi L} + \partial_{\xi}\tg_{\xi L} - \partial_{L}\tg_{\xi\xi} \right)
                        = 0 \formComma\\
      \tch{I\xi}{\xi} &= \frac{1}{2} \tg^{\xi \xi}\left( \partial_{I}\tg_{\xi\xi} + \partial_{\xi}\tg_{\xi L} - \partial_{\xi}\tg_{I \xi} \right)\formPeriod
    \end{align}
    The partial derivative in normal direction of the tangential part of thin shell metric is
    \begin{align}
      \partial_{\xi}\tg_{ij} &= 2\left( -B_{ij} + \xi\left[ \Bb^{2} \right]_{ij} \right)\formPeriod
    \end{align}
    Therefore, we obtain
    \begin{align}
      \tch{ij}{\xi} &= \frac{1}{2} \tg^{\xi\xi}\left( \partial_{i}\tg_{j\xi} + \partial_{j}\tg_{i\xi} - \partial_{\xi}\tg_{ij} \right)
                     = B_{ij} - \xi\left[ \Bb^{2} \right]_{ij} \formComma \\
      \tch{i\xi}{k} &= \frac{1}{2} \tg^{kl}\left( \partial_{i}\tg_{\xi l} + \partial_{\xi}\tg_{il} - \partial_{l}\tg_{i\xi} \right)
                     = \left( g^{kl} + 2\xi B^{kl} +  \landau(\xi^{2}) \right)\left( -B_{il}  + \xi\left[ \Bb^{2} \right]_{il} \right)\\
                    &= -\tensor{B}{_{i}^{k}} - \xi\tensor{\left[ \Bb^{2} \right]}{_{i}^{k}} + \landau(\xi^{2}) 
    \end{align}
    and with the substitution \eqref{eq:shapesquare} the remaining statements of these two terms.
    For the pure tangential thin shell Christoffel symbols, we first determine \( \tensor{\beta}{_{ij}^{k}} \) at the surface in terms of partial derivatives 
    and take advantage of the symmetry of the
    shape operator,\ie,
    \begin{align}
      \tensor{\beta}{_{ij}^{k}} 
                   &= g^{kl}\left( B_{il|j} + B_{jl|i} - B_{ij|l} \right) \\
                   &\hspace{-20pt}= g^{kl}\left( \partial_{j}B_{il} - \ch{ij}{m}B_{ml} - \ch{jl}{m}B_{im} 
                                 + \partial_{i}B_{jl} - \ch{ij}{m}B_{ml} - \ch{il}{m}B_{jm} 
                                 - \partial_{l}B_{ij} + \ch{il}{m}B_{mj} + \ch{jl}{m}B_{im}\right)\\
                   &= g^{kl}\left( \partial_{j}B_{il} + \partial_{i}B_{jl} - \partial_{l}B_{ij} - 2\ch{ij}{m}B_{ml}  \right) \\
                   &= g^{kl}\left( \partial_{j}B_{il} + \partial_{i}B_{jl} - \partial_{l}B_{ij} \right)
                        -2\Gamma_{ijl}B^{kl} \\
                   &= g^{kl}\left( \partial_{j}B_{il} + \partial_{i}B_{jl} - \partial_{l}B_{ij} \right)
                        -B^{kl}\left( \partial_{j}g_{il} + \partial_{i}g_{jl} - \partial_{l}g_{ij} \right) \formPeriod
    \end{align}
    Hence, we get
    \begin{align}
      \tch{ij}{k} &= \frac{1}{2} \tg^{kl}\left( \partial_{i}\tg_{jl} + \partial_{j}\tg_{il} - \partial_{l}\tg_{ij} \right) \\
                  &=  \frac{1}{2} \left( g^{kl} + 2\xi B^{kl} +  \landau(\xi^{2}) \right)
                        \left( \partial_{j}g_{il} + \partial_{i}g_{jl} - \partial_{l}g_{ij} 
                                  -2\xi\left( \partial_{j}B_{il} + \partial_{i}B_{jl} - \partial_{l}B_{ij} \right)\right)\\
                  &= \ch{ij}{k} + \xi\left( B^{kl}\left( \partial_{j}g_{il} + \partial_{i}g_{jl} - \partial_{l}g_{ij} \right) 
                                               - g^{kl}\left( \partial_{j}B_{il} + \partial_{i}B_{jl} - \partial_{l}B_{ij} \right)\right)
                                  +  \landau(\xi^{2})\\
                  &= \ch{ij}{k} - \xi\tensor{\beta}{_{ij}^{k}} + \landau(\xi^{2})\formPeriod
    \end{align}
  \end{proof}


\bibliography{bibl}
\bibliographystyle{alpha}


\end{document}
