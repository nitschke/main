\documentclass[a4paper,10pt]{scrartcl}

\usepackage[english]{babel}
\usepackage[utf8]{inputenc}

\usepackage{hyperref}
\usepackage{color}

\usepackage{tensor}

\usepackage{amsmath}
\usepackage{amssymb}
\usepackage{amsfonts}
\usepackage{amsthm}

%\usepackage{nath}

\usepackage{bm}

\usepackage{hyperref}

\newcommand{\surf}{\mathcal{S}}
\newcommand{\surfh}{\surf_{h}}
\newcommand{\R}{\mathbb{R}}

\newcommand{\tangentspace}{T}
\newcommand{\tensorspace}{\mathcal{T}}

\newcommand{\Tr}{\text{Tr}}

\newcommand{\landau}{\mathcal{O}}
\newcommand{\adj}{\text{adj}}

\renewcommand{\Im}{\text{Im}}

\newcommand{\xb}{\mathbf{x}}
\newcommand{\txb}{\tilde{\xb}}
\newcommand{\nub}{\bm{\nu}}
\newcommand{\gb}{\mathbf{g}}
\newcommand{\tgb}{\tilde{\gb}}
\newcommand{\tg}{\tilde{g}}
\newcommand{\tb}{\mathbf{t}}
\newcommand{\Bb}{\mathbf{B}}
\newcommand{\alphab}{\bm{\alpha}}
\newcommand{\betab}{\bm{\beta}}
\newcommand{\gammab}{\bm{\gamma}}
\newcommand{\deltab}{\bm{\delta}}
\renewcommand{\div}{\operatorname{div}}
\newcommand{\rot}{\operatorname{rot}}

\newcommand{\ttt}{\tilde{t}}
\newcommand{\ttb}{\tilde{\tb}}
\newcommand{\talphab}{\tilde{\alphab}}
\newcommand{\talpha}{\tilde{\alpha}}
\newcommand{\tbetab}{\tilde{\betab}}
\newcommand{\tgamma}{\tilde{\gamma}}
\newcommand{\tgammab}{\tilde{\gammab}}
\newcommand{\tbeta}{\tilde{\beta}}
\newcommand{\tmu}{\tilde{\mu}}

\newcommand{\tF}{\tilde{F}}

\newcommand{\tnabla}{\tilde{\nabla}}
\newcommand{\tlaplace}{\tilde{\Delta}}
\newcommand{\tdiv}{\widetilde{\div}}

\newcommand{\tnorm}[1]{\left\Vert\mkern-1.9mu\left\vert #1 \right\vert\mkern-1.9mu\right\Vert}
\newcommand{\tscal}[1]{\left\langle\mkern-8mu\left\langle #1 \right\rangle\mkern-8mu\right\rangle}

\newcommand{\tch}[2]{\widetilde{\Gamma}_{#1}^{#2}}
\newcommand{\ch}[2]{\Gamma_{#1}^{#2}}

\newcommand{\boundary}[1]{\Upsilon_{h}^{#1}}
\newcommand{\AtBoundary}[1]{\big|_{\boundary{#1}}}
\newcommand{\AtSurface}{\big|_{\surf}}

\newcommand{\tLC}{\tilde{E}}

\newcommand{\meanc}{\mathcal{H}}
\newcommand{\gaussc}{\mathcal{K}}

\newcommand{\fraki}{\mathfrak{i}}
\newcommand{\frakk}{\mathfrak{k}}
\newcommand{\frakl}{\mathfrak{l}}

\newcommand{\formComma}{\,\text{,}}
\newcommand{\formPeriod}{\,\text{.}}
\newcommand{\ie}{i.\,e.}%
\newcommand{\eg}{e.\,g.}

\newcommand{\newterm}[1]{\textbf{\textit{#1}}}

\newtheorem{theorem}{Theorem}
\newtheorem{conclusion}{Conclusion}

\title{Thin Shell Stuff} 
\author{Ingo Nitschke}


\begin{document}
\maketitle
\tableofcontents

\section{Metric Quantities}
In the following, we consider a \newterm{thin shell} of constant thickness \( h\in\R \) around an oriented, boundarieless, compact Riemannian 2-manifold (surface) \( \surf \)
defined by
\begin{align}
 \surfh := \surf\times\left[ -\frac{h}{2}, \frac{h}{2} \right] \subset \R^{3} \formPeriod
\end{align}
Constant thickness means, that the orthogonal measurement of the  two disjoint boundaries \( \boundary{+}\sqcup\boundary{-} = \boundary{} :=\partial\surfh \) is \( h \)
at all boundary points.
Thereby, be \( h \) small enough, so that  \( \surfh\subset \R^{3} \) contains no overlaps,
\ie, it exists a surjection \( \surfh \twoheadrightarrow \surf \).

\subsection{Coordinates}
We define the coordinate in normal direction \( \nub \) of the surface \( \surf \) by \( \xi\in\left[ -\frac{h}{2}, \frac{h}{2} \right] \).
If we use any choice of local coordinates \( (u,v)\in U \) of the surface, so that the immersion 
\( \xb:U \rightarrowtail \R^{3} \) parameterize \( \surf= \Im(\xb)\),
then we can define the immersion \( \txb: U\times\left[ -\frac{h}{2}, \frac{h}{2} \right] \rightarrowtail \R^{3} \) with  \( \surfh= \Im(\txb)\) by
\begin{align}
  \txb &= \txb(u,v,\xi) := \xb(u,v) + \xi\nub(u,v) = \xb + \xi\nub\formPeriod 
\end{align}

\subsection{Arrangements}
\begin{description}
  \item[Lowercase letters] \( i,j,k,\ldots \) are used as index for \( u \) and \( v \), \eg, \( \alpha_{i}dx^{i} \) is an 1-form in \( T^{*}\surf \).
  \item[Uppercase letters] \( I,J,K,\ldots \)  are used for \( u \), \( v \) and \( \xi \), \eg, 
  \( \tilde{\alpha}^{I}\partial_{I}\txb = \tilde{\alpha}^{i}\partial_{i}\txb +\tilde{\alpha}^{\xi}\partial_{\xi}\txb \) is a (contravariant) vector in \( T\surfh \).
  \item[The Tilde] are used for quantities and relations in context of \( \surfh \),
        \eg, \(  \tilde{\alphab}\in T\surfh \) but \( \alphab\in  T\surf\) and we can construct a relation \( \tilde{\alphab} = \alphab + \alpha^{\xi}\nub \).
  \item[Full covariant descriptions] (lower indices) are always used, unless otherwise is defined,
        \eg, \( \Bb = \left\{ B_{ij} \right\} \) is the full covariant shape operator, \ie, the second fundamental form in this representation.
  \item[Indexing and collector brackets], \( \left[  \right] \) and \( \left\{  \right\} \), are used to switch between components and object representations, \eg,
      \( \left[ \tb \right]_{ij} = t_{ij} \) and \( \left\{ t_{ij} \right\} = \tb \).
  \item[Sharp and flat operator on tensors] are generalisations of the usual flat and sharp operator on vector valued quantities and can be realized by matrix multiplications with the
  metric tensor \( \gb \) and its inverse \( \gb^{-1} \),
          \eg, \( \tensor[^\flat]{\left\{ \tensor{t}{^{i}_{j}} \right\}}{^{\sharp}} = \gb\left\{ \tensor{t}{^{i}_{j}} \right\}\gb^{-1} 
                  =  \left\{ \tensor{t}{_{i}^{j}} \right\} = \tb^{\sharp} \).
  \item[Tensor product] means always the contraction of the last component of a tensor with the first of another tensor,
          \eg, \( \left[ \bm{s}\tb \right]_{ij} = \tensor{s}{_{i}^{k}}t_{kj} \), \eg, with an usual matrix product \( \cdot \), 
            this implies \( \bm{s}\tb = \bm{s}\cdot\gb^{-1}\cdot\tb \).
\end{description}

\subsection{The Metric Tensor and Shape Operator}
With an arbitrary choice of surface coordinates \( (u,v)\in U \), we can calculate the canonical basic vectors \( \partial_{I}\txb\in T\surfh \) by
\begin{align}
  \partial_{i}\txb &= \partial_{i}\xb + \xi\partial_{i}\nub \\
  \partial_{\xi}\txb &= \nub \formPeriod
\end{align}

The \newterm{metric tensor} (\newterm{first fundamental form}) of thin shell is given by its components \( \tg_{IJ} = \left\langle \partial_{I}\txb , \partial_{J}\txb \right\rangle_{\surfh} \).
Therefore, for the mixed tangential-normal components holds \( \tg_{i\xi} = \tg_{\xi i} = 0 \), 
because 
\begin{align}
 \left\langle \partial_{i}\nub , \nub \right\rangle_{\surfh} = \frac{1}{2}\partial_{i}\left\| \nub \right\|^{2}_{\surfh} = 0 \formPeriod
\end{align}
For the pure normal component, we obtain \( \tg_{\xi\xi} = \left\| \nub \right\|^{2}_{\surfh} = 1 \),
\ie, the co- and contravariant normal components of a tensor quantity are equivalently,
\eg, (detailed)
\begin{align}
 \tensor{\ttt}{^{I}_{\xi J}} &= \tg_{\xi K} \tensor{\ttt}{^{IK}_{J}} 
                          = \tg_{\xi k} \tensor{\ttt}{^{Ik}_{J}} + \tg_{\xi \xi} \tensor{\ttt}{^{I\xi}_{J}}
                          = \tensor{\ttt}{^{I\xi}_{J}} \formPeriod
\end{align}
For the pure tangential components, we get a second degree tensor polynomial in \( \xi \)
\begin{align}
  \tg_{ij} &= g_{ij} - 2\xi B_{ij} + \xi^{2}\left[ \Bb^{2} \right]_{ij}
            = \left[ \left( g - \xi\Bb \right)^{2} \right]_{ij}\\
           &= g_{ij} - 2\xi B_{ij} + \xi^{2}\left( \meanc B_{ij} - \gaussc g_{ij} \right) \formPeriod
\end{align}
where the \newterm{covariant shape operator} (\newterm{second fundamental form}) is given by 
\begin{align}
 B_{ij} = -\left\langle \partial_{i}\xb , \partial_{j}\nub  \right\rangle_{\surfh} 
\end{align}
and the \newterm{third fundamental form} by
\begin{align}
 \left[ \Bb^{2} \right]_{ij} &= \left\langle \partial_{i}\nub , \partial_{j}\nub  \right\rangle_{\surfh}\formComma
\end{align}
see \cite{HartmanWintner1953}.
\( \gaussc = \left| \Bb^{\sharp} \right| \) is the \newterm{Gaussian curvature} and \( \meanc := \Tr\Bb = \tensor{B}{^{i}_{i}} \) the \newterm{mean curvature}.
(In a more differential geometrical context on surfaces, this is minus twice the mean curvature.)
A more classical representation of the third fundamental form \( \Bb^{2} \) is
\begin{align}\label{eq:shapesquare}
  \Bb^{2} = \meanc\Bb - \gaussc\gb\formPeriod
\end{align}

\begin{theorem}
  For the inverse thin shell metric \( \tgb^{-1} \) holds
  \begin{align}
    \tg^{ij} &= \left( g^{ik} + \sum_{\frakl = 1}^{\infty} \xi^{\frakl}\left[ \Bb^{\frakl} \right]^{ik} \right)
               \left( \delta_{k}^{j} + \sum_{\frakk = 1}^{\infty} \xi^{\frakk}\tensor{\left[ \Bb^{\frakk} \right]}{_{k}^{j}} \right) \\
             &= \left[\left( \gb + \sum_{\frakk = 1}^{\infty} \xi^{\frakk} \Bb^{\frakk} \right)^{2}\right]^{ij}\formComma\\
    \tg^{\xi\xi} &= 1 \formComma\\
    \tg^{i\xi} &= \tg^{\xi i} = 0 \formPeriod
  \end{align}
\end{theorem}
\begin{proof}
  First we define the pure tangential components of the thin shell metric tensor as \( \tgb_{t} := \left\{ \tg_{ij} \right\} \).
  With \( \deltab = \left\{ \delta^{i}_{j} \right\} \) the Kronecker delta, we can write down in usual matrix notation
  \begin{align}
    \tgb\cdot\tgb^{-1} &=
      \begin{bmatrix}
        \tgb_{t} & O \\
          O      & 1
      \end{bmatrix}
      \cdot
      \begin{bmatrix}
         \left\{ \tg^{ij} \right\} & \left\{ \tg^{i\xi} \right\} \\
         \left\{ \tg^{\xi i} \right\} & \tg^{\xi\xi}
      \end{bmatrix}
          =
             \begin{bmatrix}
                \deltab & O \\
                O       & 1
             \end{bmatrix} \formPeriod
  \end{align}
  Thus, we obtain
  \begin{align}
    \tg^{\xi\xi} &= 1 \formComma\\
    \tg^{i\xi} &= \tg^{\xi i} = 0 \formComma\\
    \left\{ \tg^{ij} \right\} &= \tgb_{t}^{-1}
                         =\left( \gb - \xi\Bb \right)^{-2}
                         = \left( \gb - \xi\Bb \right)^{-1} \cdot \left( \deltab - \xi\Bb^{\sharp} \right)^{-1}\formPeriod
  \end{align}
  For \( h \) small enough, so that \( \xi\left\| \Bb \right\| \le h\left\| \Bb \right\| < 1 \) and exponent with a dot indicate matrix (endomorphism) power, we can use the Neumann serie 
  \begin{align}
    \left( \deltab - \xi\Bb^{\sharp} \right)^{-1} = \deltab + \sum_{\frakk = 1}^{\infty}\xi^{\frakk}\left( \Bb^{\sharp} \right)^{\cdot\frakk} \formComma
  \end{align}
  and therefore the assertion, 
  because 
  %for a linear idempotent invertible operator \( F \), like the left or right sharp or flat operator, and \( \frakk\in\mathbb{Z} \) is
  %\begin{align}
  %  \left( F[t] \right)^{l} &= F^{l}[t^{l}] = F[t^{l}]
  %\end{align}
  %valid on a tensor \( t \) (\( \gb \) is a homomorphism with respect to tensor product).
  with \( \Bb^{\frakk}  = \left( \Bb\cdot\gb^{-1} \right)^{\cdot\frakk} \cdot \gb\) we get
  \begin{align}
    \left( \Bb^{\sharp} \right)^{\cdot\frakk} &= \left( \Bb\cdot\gb^{-1} \right)^{\cdot\frakk} = \Bb^{\frakk}\cdot\gb^{-1} = \left( \Bb^{\frakk} \right)^{\sharp}
  \end{align}
  and
  \begin{align}
    \left( \gb - \xi\Bb \right)^{-1} 
          &= \left( \left( \deltab - \xi\Bb^{\sharp} \right)\cdot \gb \right)^{-1}
          = \gb^{-1} \cdot \left( \deltab - \xi\Bb^{\sharp} \right)^{-1}
          = \tensor[^{\sharp}]{\left( \deltab - \xi\Bb^{\sharp} \right)}{^{-1}} \formPeriod
  \end{align}
\end{proof}
Therefore, we get for \( \tgb_{t}^{-1} \) a polynomial in \( \xi\Bb \) and with successively applying \eqref{eq:shapesquare},\ie,
\( \Bb^{\frakk} = \meanc\Bb^{\frakk-1} - \gaussc\Bb^{\frakk-2} \), we can always find
polynomials \( p \) and \( q \) in \( \gaussc \), \( \meanc \) and \( \xi \), so that holds
\( \tgb_{t}^{-1} = p(\gaussc, \meanc, \xi)\gb^{-1} + q(\gaussc, \meanc, \xi)\tensor[^{\sharp}]{\Bb}{^{\sharp}} \).
We will not carry this out in full generality, but let us mention some developments in \( \xi \).
\begin{conclusion}
  The developments of \( \tgb_{t}^{-1} \) up to second degree in \( \xi \) are
  \begin{align}
    \tgb_{t}^{-1} &= \gb^{-1} + \landau(\xi)\formComma\\
    \tgb_{t}^{-1} &= \gb^{-1} + 2\xi\tensor[^{\sharp}]{\Bb}{^{\sharp}} + \landau(\xi^{2})\formComma \\
    \tgb_{t}^{-1} &= \gb^{-1} + 2\xi\tensor[^{\sharp}]{\Bb}{^{\sharp}} +  3\xi^{2}\tensor[^{\sharp}]{\left( \Bb^{2} \right)}{^{\sharp}} + \landau(\xi^{3})\\
                  &= \left( 1 - 3 \xi^{2}\gaussc \right)\gb^{-1} + \xi\left( 2 + 3\xi\meanc \right)\tensor[^{\sharp}]{\Bb}{^{\sharp}} + \landau(\xi^{3}) \formPeriod
  \end{align}
\end{conclusion}


\subsubsection{The Volume Element}
  To develop the thin shell volume element \( \tilde{\mu} \) in normal direction at the surface volume element \( \mu \), 
  we need a development of the determinant of the metric tensor \( \tgb \).
  \begin{theorem}
    For the determinant of the thin shell metric tensor \( \left| \tgb \right| \) holds
    \begin{align}
       \left| \tgb \right| &= \left( 1 - \xi\meanc + \xi^{2}\gaussc \right)^{2}\left| \gb \right|\formComma
    \end{align}
  \end{theorem}
  \begin{proof}
    The mixed components are zero, so we get
    \begin{align}
      \left| \tgb \right| &= \tg_{\xi\xi}\left| \tgb_{t} \right| = \left| \tgb_{t} \right|\formPeriod
    \end{align}
    Now, we define \( \sqrt{\tgb_{t}^{\sharp}}   := \left(\gb - \xi \Bb\right)^{\sharp} \) as a square root of \( \tgb_{t}^{\sharp} \),
    because
    \begin{align}
      \tgb_{t}^{\sharp} &= \left( \left( \gb -\xi\Bb \right)^{2} \right)^{\sharp}
              = \left( \left( \gb -\xi\Bb \right)^{\sharp}\left( \gb -\xi\Bb \right)  \right)^{\sharp}
              = \left( \gb -\xi\Bb \right)^{\sharp}\left( \gb -\xi\Bb \right)^{\sharp} 
              = \left( \sqrt{\tgb_{t}^{\sharp}} \right)^{2} \formPeriod
    \end{align}
    Hence, we can calculate
    \begin{align}
      \left| \tgb \right| &= \left| \tgb_{t} \right| = \left| \tgb_{t}^{\sharp}\gb \right|
                    =\left| \tgb_{t}^{\sharp} \right| \left| \gb \right|
                    =\left| \sqrt{\tgb_{t}^{\sharp}} \right|^{2} \left| \gb \right| \formPeriod
    \end{align}
    For the determinant of \( \sqrt{\tgb_{t}^{\sharp}} \), we regard that \( \gb^{\sharp} \) is the Kronecker delta,
    so we obtain
    \begin{align}
       \left| \sqrt{\tgb_{t}^{\sharp}} \right| 
            &= \left| \gb^{\sharp} - \xi\Bb^{\sharp} \right|
             = \left( 1 - \xi\tensor{B}{_{u}^{u}} \right)\left( 1 - \xi\tensor{B}{_{v}^{v}} \right)
                  - \xi^{2}\tensor{B}{_{u}^{v}}\tensor{B}{_{v}^{u}} \\
            &= 1 - \xi\left( \tensor{B}{_{u}^{u}} + \tensor{B}{_{v}^{v}} \right) 
                  + \xi^{2}\left( \tensor{B}{_{u}^{u}} \tensor{B}{_{v}^{v}} - \tensor{B}{_{u}^{v}}\tensor{B}{_{v}^{u}} \right)
             =\left( 1 - \xi\meanc + \xi^{2}\gaussc \right) \formPeriod
    \end{align}
  \end{proof}
  Therefore a representation of the \newterm{thin shell volume element} \( \tilde{\mu} \), depending on the surface volume element \( \mu \), is
  \begin{align}
    \tilde{\mu} &= \sqrt{\left| \tgb \right|} du\wedge dv \wedge d\xi
            = \left(  1 - \xi\meanc + \xi^{2}\gaussc\right)  \mu \wedge d\xi \\
                &= \left(  1 - \xi\meanc + \xi^{2}\gaussc\right)  d\xi \wedge \mu \formPeriod
  \end{align}

\subsubsection{The Levi-Civita Tensor and Hodge-Dualism}

\subsection{Christoffel Symbols}
  The \newterm{Christoffel symbols} are needed to define an unique metric compatible derivation (Levi-Civita connection).
  With a choice of coordinates, the christoffel symbols (of second kind) on the thin shell are
  \begin{align}
    \tch{IJ}{K} &= \frac{1}{2} \tg^{KL}\left( \partial_{I}\tg_{JL} + \partial_{J}\tg_{IL} - \partial_{L}\tg_{IJ} \right)\formPeriod
  \end{align}
  On the surface \( \surf \), they are equal defined, just omit the tilde and use lowercase letters for indexing.
  \begin{theorem}
    With \( \tensor{\beta}{_{ij}^{k}} := \tensor{B}{_{i}^{k}_{|j}} +  \tensor{B}{_{j}^{k}_{|i}} - \tensor{B}{_{ij}^{|k}}\)
    the second order expansions in normal direction of Christoffel symbols are
    \begin{align}
      \tch{ij}{k} &= \ch{ij}{k} - \xi\tensor{\beta}{_{ij}^{k}} + \landau(\xi^{2})\\
      \tch{ij}{\xi} &= B_{ij} - \xi \left[ \Bb^{2} \right]_{ij} 
                          &= \left( 1 - \xi\meanc \right)B_{ij} + \xi\gaussc g_{ij}\\
      \tch{i\xi}{k} = \tch{\xi i}{k} &= -\tensor{B}{_{i}^{k}} - \xi\tensor{\left[ \Bb^{2} \right]}{_{i}^{k}} + \landau(\xi^{2})
                                        &= -\left( 1 + \xi\meanc \right)\tensor{B}{_{i}^{k}} + \xi\gaussc\delta_{i}^{k} + \landau(\xi^{2})\\
      \tch{\xi\xi}{K} &= 0 \\
      \tch{I\xi}{\xi} = \tch{\xi I}{\xi} &= 0 \formPeriod
    \end{align}
  \end{theorem}
  \begin{proof}
    Properties of the thin shell metric \( \tgb \) are the mixed tangential-normal components are zero (the same holds for the inverse metric) and the pure normal component is constant.
    Hence, we get
    \begin{align}
      \tch{\xi\xi}{K} &= \frac{1}{2}\tg^{KL}\left( \partial_{\xi}\tg_{\xi L} + \partial_{\xi}\tg_{\xi L} - \partial_{L}\tg_{\xi\xi} \right)
                        = 0 \formComma\\
      \tch{I\xi}{\xi} &= \frac{1}{2} \tg^{\xi \xi}\left( \partial_{I}\tg_{\xi\xi} + \partial_{\xi}\tg_{I \xi} - \partial_{\xi}\tg_{I \xi} \right)
                        = 0\formPeriod
    \end{align}
    The partial derivative in normal direction of the tangential part of thin shell metric is
    \begin{align}
      \partial_{\xi}\tg_{ij} &= 2\left( -B_{ij} + \xi\left[ \Bb^{2} \right]_{ij} \right)\formPeriod
    \end{align}
    Therefore, we obtain
    \begin{align}
      \tch{ij}{\xi} &= \frac{1}{2} \tg^{\xi\xi}\left( \partial_{i}\tg_{j\xi} + \partial_{j}\tg_{i\xi} - \partial_{\xi}\tg_{ij} \right)
                     = B_{ij} - \xi\left[ \Bb^{2} \right]_{ij} \formComma \\
      \tch{i\xi}{k} &= \frac{1}{2} \tg^{kl}\left( \partial_{i}\tg_{\xi l} + \partial_{\xi}\tg_{il} - \partial_{l}\tg_{i\xi} \right)
                     = \left( g^{kl} + 2\xi B^{kl} +  \landau(\xi^{2}) \right)\left( -B_{il}  + \xi\left[ \Bb^{2} \right]_{il} \right)\\
                    &= -\tensor{B}{_{i}^{k}} - \xi\tensor{\left[ \Bb^{2} \right]}{_{i}^{k}} + \landau(\xi^{2}) 
    \end{align}
    and with the substitution \eqref{eq:shapesquare} the remaining statements of these two terms.
    For the pure tangential thin shell Christoffel symbols, we first determine \( \tensor{\beta}{_{ij}^{k}} \) at the surface in terms of partial derivatives 
    and take advantage of the symmetry of the
    shape operator,\ie,
    \begin{align}
      \tensor{\beta}{_{ij}^{k}} 
                   &= g^{kl}\left( B_{il|j} + B_{jl|i} - B_{ij|l} \right) \\
                   &\hspace{-20pt}= g^{kl}\left( \partial_{j}B_{il} - \ch{ij}{m}B_{ml} - \ch{jl}{m}B_{im} 
                                 + \partial_{i}B_{jl} - \ch{ij}{m}B_{ml} - \ch{il}{m}B_{jm} 
                                 - \partial_{l}B_{ij} + \ch{il}{m}B_{mj} + \ch{jl}{m}B_{im}\right)\\
                   &= g^{kl}\left( \partial_{j}B_{il} + \partial_{i}B_{jl} - \partial_{l}B_{ij} - 2\ch{ij}{m}B_{ml}  \right) \\
                   &= g^{kl}\left( \partial_{j}B_{il} + \partial_{i}B_{jl} - \partial_{l}B_{ij} \right)
                        -2\Gamma_{ijl}B^{kl} \\
                   &= g^{kl}\left( \partial_{j}B_{il} + \partial_{i}B_{jl} - \partial_{l}B_{ij} \right)
                        -B^{kl}\left( \partial_{j}g_{il} + \partial_{i}g_{jl} - \partial_{l}g_{ij} \right) \formPeriod
    \end{align}
    Hence, we get
    \begin{align}
      \tch{ij}{k} &= \frac{1}{2} \tg^{kl}\left( \partial_{i}\tg_{jl} + \partial_{j}\tg_{il} - \partial_{l}\tg_{ij} \right) \\
                  &=  \frac{1}{2} \left( g^{kl} + 2\xi B^{kl} +  \landau(\xi^{2}) \right)
                        \left( \partial_{j}g_{il} + \partial_{i}g_{jl} - \partial_{l}g_{ij} 
                                  -2\xi\left( \partial_{j}B_{il} + \partial_{i}B_{jl} - \partial_{l}B_{ij} \right)\right)\\
                  &= \ch{ij}{k} + \xi\left( B^{kl}\left( \partial_{j}g_{il} + \partial_{i}g_{jl} - \partial_{l}g_{ij} \right) 
                                               - g^{kl}\left( \partial_{j}B_{il} + \partial_{i}B_{jl} - \partial_{l}B_{ij} \right)\right)
                                  +  \landau(\xi^{2})\\
                  &= \ch{ij}{k} - \xi\tensor{\beta}{_{ij}^{k}} + \landau(\xi^{2})\formPeriod
    \end{align}
  \end{proof}

\section{Boundary Conditions}
  If we consider the limit case \( h\rightarrow 0 \) for an differential expression or especially an operator \( L_{h} \), 
  then we may get remaining partial normal derivation \( \partial_{\xi}^{p} \) with arbitrary order \( p\ge 0 \).
  Those expressions may be undetermined on the surface \( \surf \). 
  One way out is to set an additional condition on the whole thin shell, \eg \newterm{parallel transportation} of quantities,
  \eg for  a tensor \( \tilde{\tb} \), we set \( \nabla_{\xi}\tilde{\tb} \equiv 0 \).
  Another possibility is to take boundary conditions of the thin shell into account by expanding in normal directions.


  \subsection{No-Penetration Condition (NPC) for vector quantities}
    We consider the \newterm{No-Penetration Condition}
    \begin{align}
      \nub \cdot \talphab &= \talpha_{\xi} = 0 \text{ on }\boundary{} \formPeriod
      \tag{NPC}
    \end{align}
    Taylor expansion at the surface \( \surf \) results in
    \begin{align}
      0 &= \talpha_{\xi}\AtBoundary{\pm} 
          = \talpha_{\xi}\AtSurface \pm \frac{h}{2}\partial_{\xi}\talpha_{\xi}\AtSurface + \frac{h^{2}}{8}\partial_{\xi}^{2}\talpha_{\xi}\AtSurface + \landau(h^{3})\\
      0 &= \talpha_{\xi}\AtBoundary{+} + \talpha_{\xi}\AtBoundary{-}
         = 2 \talpha_{\xi}\AtSurface + \landau(h^{2}) 
       &\Rightarrow \boxed{\talpha_{\xi}\AtSurface = \landau(h^{2})}\\
      0 &= \talpha_{\xi}\AtBoundary{+} - \talpha_{\xi}\AtBoundary{-}
         = h \partial_{\xi}\talpha_{\xi}\AtSurface + \landau(h^{3})
        &\Rightarrow \boxed{\partial_{\xi}\talpha_{\xi}\AtSurface = \landau(h^{2})} \formPeriod
    \end{align}

  \subsection{Neumann Condition (NC) for vector quantities}
    We consider the \newterm{Neumann Condition}
    \begin{align}
      \tnabla_{\nub}\talphab = \left\{ \tnabla_{\xi}\talpha_{I} \right\} = \left\{ \tnabla_{\xi}\talpha^{I} \right\} = 0 \text{ on }\boundary{} \formPeriod
      \tag{NPC}
    \end{align}
    First we investigate the tangential parts 
    \begin{align}
      \tnabla_{\xi}\talpha_{i} 
              &= \partial_{\xi}\talpha_{i} - \tch{\xi i}{J}\talpha_{J} 
              = \partial_{\xi}\talpha_{i} - \tch{\xi i}{j}\talpha_{j}\\
              &= \partial_{\xi}\talpha_{i} + \tensor{B}{_{i}^{j}}\talpha_{j} + \xi\tensor{\left[ \Bb^{2} \right]}{_{i}^{j}}\talpha_{j} + \landau(\xi^{2})\\
      \tnabla_{\xi}\talpha_{i}\AtBoundary{\pm}
              &= \partial_{\xi}\talpha_{i}\AtBoundary{\pm} + \tensor{B}{_{i}^{j}}\talpha_{j}\AtBoundary{\pm} 
                      \pm \frac{h}{2} \tensor{\left[ \Bb^{2} \right]}{_{i}^{j}}\talpha_{j}\AtBoundary{\pm} + \landau(h^{2}) \formPeriod \label{eq:normalDerivationAtBoundary}
    \end{align}
    Taylor expansion at the surface \( \surf \) for \( p \ge 0 \) results in
    \begin{align}
      \partial_{\xi}^{p}\talphab\AtBoundary{+} + \partial_{\xi}^{p}\talphab\AtBoundary{-}
          &= 2\partial_{\xi}^{p}\talphab\AtSurface + \landau(h^{2}) \\
      \partial_{\xi}^{p}\talphab\AtBoundary{+} - \partial_{\xi}^{p}\talphab\AtBoundary{-}
          &= h \partial_{\xi}^{p+1}\talphab\AtSurface + \landau(h^{3}) \formPeriod
    \end{align}
    Therefor by making up the sum and the difference of \eqref{eq:normalDerivationAtBoundary} one obtain 
    \begin{align}
      0 = \tnabla_{\xi}\talpha_{i}\AtBoundary{+} + \tnabla_{\xi}\talpha_{i}\AtBoundary{-}
          &= 2\partial_{\xi}\talpha_{i}\AtSurface + 2\tensor{B}{_{i}^{j}}\alpha_{j} + \landau(h^{2})\\
          &\Longrightarrow \boxed{\partial_{\xi}\talpha_{i}\AtSurface = - \tensor{B}{_{i}^{j}}\alpha_{j} + \landau(h^{2})} \formComma \label{eq:NeumannFirstOrderFlat}\\
      0 = \tnabla_{\xi}\talpha_{i}\AtBoundary{+} - \tnabla_{\xi}\talpha_{i}\AtBoundary{-}
          &= h\partial_{\xi}^{2}\talpha_{i}\AtSurface + h\tensor{B}{_{i}^{j}}\partial_{\xi}\talpha_{j}\AtSurface + h\tensor{\left[ \Bb^{2} \right]}{_{i}^{j}}\alpha_{j} + \landau(h^{3})\\
          &= h\partial_{\xi}^{2}\talpha_{i}\AtSurface + \landau(h^{3})\\
        &\Longrightarrow \boxed{\partial_{\xi}^{2}\talpha_{i}\AtSurface = \landau(h^{2})} \formPeriod \label{eq:NeumannSecondOrderFlat}
    \end{align}
    But we are carefully about the meaning of the results relating to rising the indices,
    \ie, \( \partial_{\xi}\talpha^{i}\AtSurface \neq g^{ij}\partial_{\xi}\talpha_{j}\AtSurface  \) generally,
    because \(\partial_{\xi}\tgb \neq 0  \) neither at \( \surf \) nor in whole \( \surfh \).
    But with
    \begin{align}
      \partial_{\xi}\tg^{ij} &= 2B^{ij} + \landau(\xi) \formComma\\
      \partial_{\xi}^{2}\tg^{ij} &= 6\left[ \Bb^{2} \right]^{ij} + \landau(\xi) \formComma\\
      \partial_{\xi}\talpha^{i} &= \partial_{\xi}\left(\tg^{ij}\talpha_{j}\right)  \label{eq:dnuCoToContra}
                                = \tg^{ij}\partial_{\xi}\talpha_{j} + \talpha_{j}\partial_{\xi}\tg^{ij} \\\notag
                                &= \tg^{ij}\partial_{\xi}\talpha_{j} + 2B^{ij}\talpha_{j} + \landau(\xi) \formComma  \\
      \partial_{\xi}^{2}\talpha^{i} &= \partial_{\xi}^{2}\left(\tg^{ij}\talpha_{j}\right) \label{eq:dnudnuCoToContra}
                                = \tg^{ij}\partial_{\xi}^{2}\talpha_{j} + \talpha_{j}\partial_{\xi}^{2}\tg^{ij} 
                                      + 2\left( \partial_{\xi}\tg^{ij} \right)\left( \partial_{\xi}\talpha_{j}\right)\\\notag
                               &= \tg^{ij}\partial_{\xi}^{2}\talpha_{j} + 4B^{ij}\partial_{\xi}\talpha_{j} + 6\left[ \Bb^{2} \right]^{ij}\talpha_{j}  + \landau(\xi) \formComma
    \end{align}
    \eqref{eq:NeumannFirstOrderFlat}, and \eqref{eq:NeumannSecondOrderFlat}, we get for the restriction to the surface
    \begin{align}
      \boxed{\partial_{\xi}\talpha^{i}\AtSurface = \tensor{B}{^{i}_{j}}\alpha^{j} + \landau(h^{2})} \formComma\\
      \boxed{\partial_{\xi}^{2}\talpha^{i}\AtSurface = 2 \tensor{\left[ \Bb^{2} \right]}{^{i}_{j}}\alpha^{j} + \landau(h^{2})} \formPeriod
    \end{align}
    The former is also consistent to the covariant normal derivation formulation
    \begin{align}
      \boxed{\tnabla_{\xi}\talpha^{i}\AtSurface = g^{ij}\tnabla_{\xi}\talpha_{i}\AtSurface = \landau(h^{2})} \formPeriod
    \end{align}
    For the boundary condition in normal direction, namely
    \begin{align}
      \tnabla_{\xi}\talpha_{\xi}\AtBoundary{\pm} =  \partial_{\xi} \talpha_{\xi}\AtBoundary{\pm} = 0 \formComma
    \end{align}
    we have
    \begin{align}
      0 = \tnabla_{\xi}\talpha_{\xi}\AtBoundary{+} + \tnabla_{\xi}\talpha_{\xi}\AtBoundary{-}
        &= 2\partial_{\xi}\talpha_{\xi}\AtSurface + \landau(h^{2})
            &\Rightarrow \boxed{\partial_{\xi}\talpha_{\xi}\AtSurface = \partial_{\xi}\talpha^{\xi}\AtSurface = \landau(h^{2})} \formComma\\
      0 = \tnabla_{\xi}\talpha_{\xi}\AtBoundary{+} - \tnabla_{\xi}\talpha_{\xi}\AtBoundary{-}
        &= h\partial_{\xi}^{2}\talpha_{\xi}\AtSurface + \landau(h^{3})
            &\Rightarrow \boxed{\partial_{\xi}^{2}\talpha_{\xi}\AtSurface  = \partial_{\xi}^{2}\talpha^{\xi}\AtSurface = \landau(h^{2})} \formPeriod
    \end{align}

    \subsubsection{No-Penetration Neumann Condition (NPNC)}
      If we add up (NPC) and (NC) but omit the condition \( \tnabla_{\xi}\talpha_{\xi} = \partial_{\xi}\talpha_{\xi} = 0 \), then we have
      the boundary conditions
      \begin{align}
        \talpha_{\xi} = \tnabla_{\xi}\talpha_{i} = 0 \text{ on }\boundary{} \formPeriod
      \tag{NPNC}
      \end{align}
      In other words, we have Dirichlet (essential) BC for the normal component and Neumann (natural) BC for the tangential components at the Boundaries.
      However, the assignments to the surface are
      \begin{align}
        \partial_{\xi}\talpha_{i}\AtSurface &= - \tensor{B}{_{i}^{j}}\alpha_{j} + \landau(h^{2})
                &\Rightarrow\ \partial_{\xi}\talpha^{i}\AtSurface &= \tensor{B}{^{i}_{j}}\alpha^{j} + \landau(h^{2}) \\
        \partial_{\xi}^{2}\talpha_{i}\AtSurface &= \landau(h^{2})
                &\Rightarrow\ \partial_{\xi}^{2}\talpha^{i}\AtSurface &= 2\tensor{\left[\Bb^{2}\right]}{^{i}_{j}}\alpha^{j} + \landau(h^{2}) \\
        \talpha_{\xi}\AtSurface = \partial_{\xi}\talpha_{\xi}\AtSurface &= \landau(h^{2}) \formPeriod
      \end{align}
      Note that \( \partial_{\xi}^{2}\talpha_{\xi} \) keep undetermined and occupies the role of a function independent of \( \alphab =\left\{ \alpha_{i} \right\} \) on the surface \( \surf \).
        

  
  \subsection{Hodge Condition (HC) for vector quantities}
    We consider the \newterm{Hodge Condition}
    \begin{align}
      \nub \times \tnabla \times \talphab^{\sharp}= 0 \text{ on }\boundary{} \formPeriod
      \tag{HC}
    \end{align}
    That means that the tangential part of the vorticity vector of \( \talphab^{\sharp} \) has to vanish at the boundary, \ie, 
    \( \left[ \widetilde{\operatorname{rot}}\talphab \right]^{i}\AtBoundary{\pm} = 0 \), \ie,
    only rotation on the boundary surface is allowed but not through it.

    First, let us develop a general statement, how arbitrary tensor-like zero boundary conditions can be assigned to the surface.
    Consider a tensor \( \ttb = 0 \) on the boundary and its expansion at the surface
    \begin{align}
      0 &= \ttb\AtBoundary{\pm} = \ttb\AtSurface \pm \frac{h}{2}\partial_{\xi}\ttb\AtSurface + \frac{h^{2}}{8}\partial_{\xi}^{2}\ttb\AtSurface + \landau(h^{3})\formPeriod
    \end{align}
    Hence, we get
    \begin{align}
      0 &= \ttb\AtBoundary{+} - \ttb\AtBoundary{-} = h\partial_{\xi}\ttb\AtSurface + \landau(h^{3})
          &\Rightarrow\ \partial_{\xi}\ttb\AtSurface &= \landau(h^{2}) \formComma\\
      0 &= \ttb\AtBoundary{+} + \ttb\AtBoundary{-} = 2\ttb\AtSurface + \landau(h^{2})
          &\Rightarrow\ \ttb\AtSurface &= \landau(h^{2}) \formPeriod
    \end{align}

    Applying the Levi-Civita machinery in Hodge Condition to determine the cross products, we obtain a simpler representation of the boundary condition, namely 
    \begin{align}
      0 &= \left[ \nub \times \tnabla \times \talphab^{\sharp} \right]_{I}
         = \tLC_{I\xi K}\left( -\tensor{\tLC}{^{K}_{LM}}\left[ \tnabla\alphab \right]^{LM} \right)
         = \tLC_{K\xi I}\tensor{\tLC}{^{K}_{LM}}\left[ \tnabla\alphab \right]^{LM}\\\notag
        &= \left( \tgb_{\xi L}\tgb_{I M} - \tgb_{\xi M}\tgb_{I L}\right)\left[ \tnabla\alphab \right]^{LM}
         = \left[ \tnabla\alphab \right]_{\xi I} - \left[ \tnabla\alphab \right]_{I \xi}\\\notag
        &= \partial_{I}\talpha_{\xi} - \tch{\xi I}{K}\talpha_{K} - \left(\partial_{\xi}\talpha_{I} - \tch{\xi I}{K}\talpha_{K}\right)
         = \partial_{I}\talpha_{\xi} - \partial_{\xi}\talpha_{I}
    \end{align}
    at \( \boundary{} \).
    The normal part of this condition is always true and the tangential part is assigned to the surface by
    \begin{align}
    \boxed{
      \partial_{\xi}\talpha_{i}\AtSurface = \partial_{i}\talpha_{\xi}\AtSurface + \landau(h^{2}) \formComma}\\
    \boxed{
      \partial_{\xi}^{2}\talpha_{i}\AtSurface = \partial_{i}\partial_{\xi}\talpha_{\xi}\AtSurface + \landau(h^{2}) \formPeriod
     }
    \end{align}
    And for a contravariant formulation, we have with \eqref{eq:dnuCoToContra} and \eqref{eq:dnudnuCoToContra}
    \begin{align}
      \boxed{\partial_{\xi}\talpha^{i}\AtSurface = \partial^{i}\talpha_{\xi}\AtSurface + 2\tensor{B}{^{i}_{j}}\alpha^{j} + \landau(h^{2})} \formComma\\
      \boxed{
        \partial_{\xi}^{2}\talpha^{i}\AtSurface = \partial^{i}\partial_{\xi}\talpha_{\xi}\AtSurface
                                                  +4B^{ij}\partial_{j}\talpha_{\xi}\AtSurface
                                                  +6\tensor{\left[\Bb^{2}\right]}{^{i}_{j}}\alpha^{j} + \landau(h^{2})
      }\formPeriod
    \end{align}

    \subsubsection{No-Penetration Hodge Condition (NPHC)}
      The combination of the No-Penetration Condition and the Hodge Condition becomes to
      \begin{align}
        \talpha_{\xi} = \partial_{\xi}\talpha_{i} = 0 \text{ on }\boundary{} \formPeriod
      \end{align}
      With our previous work, we easily see, that holds
      \begin{align}
        \partial_{\xi}\talpha_{i}\AtSurface &= \landau(h^{2})
                &\Rightarrow\ \partial_{\xi}\talpha^{i}\AtSurface &= 2\tensor{B}{^{i}_{j}}\alpha^{j} + \landau(h^{2}) \\
        \partial_{\xi}^{2}\talpha_{i}\AtSurface &= \landau(h^{2})
                &\Rightarrow\ \partial_{\xi}^{2}\talpha^{i}\AtSurface &= 6\tensor{\left[\Bb^{2}\right]}{^{i}_{j}}\alpha^{j} + \landau(h^{2}) \\
        \talpha_{\xi}\AtSurface = \partial_{\xi}\talpha_{\xi}\AtSurface &= \landau(h^{2}) \formPeriod
      \end{align}
      Note that \( \partial_{\xi}^{2}\talpha_{\xi} \) keep undetermined and occupies the role of a function independent of \( \alphab =\left\{ \alpha_{i} \right\} \) on the surface \( \surf \).

  \subsection{Thin Shell Limit of the Vector-Poisson Problem}
    Let us develop a deeper understanding of the Poisson Problem and its thin shell limit depending on boundary conditions.
    In its differential form, the Poisson equation is given by
    \begin{align}
      \tlaplace\talphab &= \tbetab &\text{in }\surfh
    \end{align}
    with one of the following boundary conditions at \( \boundary{}\)
    \begin{align}
      \nub\cdot\talphab = 0 \text{ and } \tnabla_{\nub}\Pi\left[ \talphab \right] &= 0
        &\Leftrightarrow&& \talpha_{\xi} = \tnabla_{\xi}\talpha_{i} &= 0  \tag{NPNC}\formComma\\
      \nub\cdot\talphab = 0 \text{ and } \nub \times \tnabla \times \talphab &= 0
        &\Leftrightarrow&& \talpha_{\xi} = \partial_{\xi}\talpha_{i} &= 0 \tag{NPHC} \formPeriod
    \end{align}
    These two boundary value problems are one of the Euler-Lagrange equations of the action functionals
    \begin{align}
      \tF_{N}\left[ \talphab \right] &:= \int_{\surfh} \frac{1}{2}\tnorm{\tnabla\talphab}^{2} + \tscal{\tbetab,\talphab}\tmu \formComma\\
      \tF_{H}\left[ \talphab \right] &:= \int_{\surfh} \frac{1}{2}\left( \left( \tdiv\talphab \right)^{2} + \tnorm{\tnabla\times\talphab}^{2} \right) + \tscal{\tbetab,\talphab}\tmu\formPeriod
    \end{align}
    Another representation of the rotation part is
    \begin{align}
      \tnorm{\tnabla\times\talphab}^{2}
          &= \tLC_{IJK}\tensor{\tLC}{^{I}_{LM}} \talpha^{J:K}\talpha^{L:M}
          = \left( \tg_{JL}\tg_{KM} - \tg_{JM}\tg_{KL} \right) \talpha^{J:K}\talpha^{L:M}\\
          &=      \talpha^{J:K}\talpha_{J:K} -    \talpha^{J:K}\talpha_{K:J}  
          = \tnorm{\tnabla\talphab}^{2} - \tscal{\tnabla\talphab, \left( \tnabla\talphab \right)^{T}}\formPeriod
    \end{align}
    Integration by parts of all terms containing derivatives, results in
    \begin{align}
      \int_{\surfh} \tnorm{\tnabla\talphab}^{2} \tmu
          &= \int_{\surfh}\tensor{\talpha}{_{I}^{:J}}\tensor{\talpha}{^{I}_{:J}}\tmu
           = \int_{\surfh} -\tensor{\talpha}{_{I}^{:J}_{:J}}\talpha^{I} + \left( \tensor{\talpha}{_{I}^{:J}} \talpha^{I} \right)_{:J} \tmu \\
          &= -\int_{\surfh} \tscal{\tlaplace\talphab , \talphab} \tmu
             + \sum_{s\in\{+,-\}}s\int_{\Upsilon^{s}} \talpha^{I}\talpha_{I:\xi} \mu_{\Upsilon^{s}}\\
      \int_{\surfh} \tscal{\tnabla\talphab, \left( \tnabla\talphab \right)^{T}} \tmu
          &= \int_{\surfh} \tensor{\talpha}{^{I}_{:J}} \tensor{\talpha}{^{J}_{:I}} \tmu
           = \int_{\surfh} - \tensor{\talpha}{^{I}_{:J:I}}\talpha^{J} + \left( \tensor{\talpha}{^{I}_{:J}}\talpha^{J} \right)_{:I} \tmu \\
          &= -\int_{\surfh} \tensor{\talpha}{^{I}_{:J:I}}\talpha^{J} \tmu
             + \sum_{s\in\{+,-\}}s\int_{\Upsilon^{s}} \talpha^{I} \talpha_{\xi:I} \mu_{\Upsilon^{s}} \\
      \int_{\surfh}\left( \tdiv\talphab \right)^{2} \tmu
          &=  \int_{\surfh} \tensor{\talpha}{^{I}_{:I}} \tensor{\talpha}{^{J}_{:J}} \tmu
           = \int_{\surfh} -\tensor{\talpha}{^{I}_{:I:J}}\talpha^{J} + \left( \tensor{\talpha}{^{I}_{:I}} \talpha^{J} \right)_{:J} \tmu \\
          &= -\int_{\surfh} \tensor{\talpha}{^{I}_{:I:J}}\talpha^{J} \tmu
             + \sum_{s\in\{+,-\}}s\int_{\Upsilon^{s}} \talpha_{\xi}\tensor{\talpha}{^{I}_{:I}}  \mu_{\Upsilon^{s}} \formPeriod
    \end{align}
    We observe that the thin shell \( \surfh \) is a flat three-dimensional manifold,
    therefore the Riemannian curvature tensor \( \left\{ \tensor{R}{^{I}_{JKL}} \right\} \) vanish and the Levi-Civita connection \( \tnabla \) 
    commutes\footnote{We see this behaviour also by coordinate transformation to Euclidian \( \R^{3} \), where \( \tnabla_{I}=\partial_{I} \).}, 
    \ie, \( \tensor{\talpha}{^{K}_{:J:I}} = \tensor{\talpha}{^{K}_{:I:J}} \), generally.
    Hence, the difference between the two action functionals is
    \begin{align}
      \left( \tF_{N} - \tF_{H} \right)\left[ \talphab \right]
          &= \frac{1}{2} \sum_{s\in\{+,-\}}s\int_{\Upsilon^{s}} \talpha^{I} \talpha_{\xi:I} - \talpha_{\xi}\tensor{\talpha}{^{I}_{:I}} \mu_{\Upsilon^{s}}\\
          &= \frac{1}{2} \sum_{s\in\{+,-\}}s\int_{\Upsilon^{s}} \talpha^{i} \talpha_{\xi:i} - \talpha_{\xi}\tensor{\talpha}{^{i}_{:i}} \mu_{\Upsilon^{s}}\formComma
    \end{align}
    which does not vanish for an fixed thickness \( h \), generally, neither for No-Penetration Condition \(\talpha_{\xi}\AtBoundary{\pm} = 0  \).
    Remember, \( \talpha_{\xi:i} \) is not supposed to be a surface derivative on \( \Upsilon^{s} \), 
    so this term does not support integration by parts on the boundary.
    If we define a family of layers \( \surf(\xi) := \text{Im}\txb(\cdot,\cdot,\xi) \) for arbitrary fixed \( \xi \) and
    \begin{align}
      b\left[ \talphab \right]\left( \xi \right) &:= \frac{1}{2}\int_{\surf(\xi)} \talpha^{i} \talpha_{\xi:i} - \talpha_{\xi}\tensor{\talpha}{^{i}_{:i}} \mu_{\surf(\xi)}\formComma
    \end{align}
    then we obtain
    \begin{align}
      \frac{1}{h}\left( \tF_{N} - \tF_{H} \right)\left[ \talphab \right]
          = \frac{1}{h}\left( b\left[ \talphab \right]\left( \frac{h}{2} \right) -  b\left[ \talphab \right]\left( -\frac{h}{2} \right)\right)
          = \partial_{\xi}  b\left[ \talphab \right]\left( 0 \right) + \landau(h^{2})\formPeriod
    \end{align}
    The rescaling of the left hand side with \( \frac{1}{h} \) is necessary, because all thin shell action functional are vanishing with at least \( \landau(h) \),
    independent of their density.
    Moreover, this is the natural way to define the limiting process in \( h \), which is consistency respect to the volumes,
    \ie, \( |\surfh|\rightarrow|\surf| \) for \( h\rightarrow 0 \).
    For \(\talpha_{\xi}\AtBoundary{\pm} = 0  \), which fulfill both boundary conditions (NPHC) and (NPNC), we have
    \begin{align}
      \partial_{\xi}  b\left[ \talphab \right]\left( 0 \right) 
            &=  -\frac{1}{2}\partial_{\xi}  \int_{\surf(\xi)} \tg_{kj}\tch{\xi i}{k} \talpha^{i}\talpha^{j} \mu_{\surf(\xi)}\Big|_{\xi=0} \\
            &= \frac{1}{2}\partial_{\xi} \int_{\surf(\xi)}\left( B_{ij} - \xi\left[ \Bb^{2} \right]_{ij} \right) \talpha^{i}\talpha^{j} \left( 1 - \xi\meanc + \xi^{2}\gaussc\right)\mu\Big|_{\xi=0}\\
            &=  \frac{1}{2} \int_{\surf} -\left[ \Bb^{2} \right]_{ij}\alpha^{i}\alpha^{j} 
                                         +2B_{ij}\alpha^{i}\partial_{\xi}\talpha^{j}\big|_{\surf}
                                         -\meanc B_{ij}\alpha^{i}\alpha^{j} \mu \\
            &=              
              \begin{cases}
                 \frac{1}{2} \int_{\surf} \left[ \Bb^{2} - \meanc\Bb \right]_{ij}\alpha^{i}\alpha^{j} \mu + \landau(h^{2}) & \text{for (NPNC)} \\
                  \frac{1}{2} \int_{\surf} \left[ 3\Bb^{2} - \meanc\Bb \right]_{ij}\alpha^{i}\alpha^{j} \mu + \landau(h^{2}) & \text{for (NPHC)}
              \end{cases}\\
             &=
              \begin{cases}
                \frac{1}{2} \int_{\surf} -\gaussc\left\| \alphab \right\| \mu + \landau(h^{2}) & \text{for (NPNC)} \\
                \frac{1}{2} \int_{\surf} -\gaussc\left\| \alphab \right\| + \left\| \Bb\alphab \right\|^{2}\mu + \landau(h^{2}) & \text{for (NPHC)}
              \end{cases} \\
             &= \landau(1) \text{ , generally}
            \formPeriod
    \end{align}
    Consequently, \( \tF_{N} \) and \( \tF_{H} \) differ at the boundaries, also in the limit case, and hence, 
    we have to equip both functionals with their own boundary conditions to fulfill the associated Euler-Lagrange (EL) equations.
    The functional derivatives are
    \begin{align}
      \int_{\surfh} \tscal{\frac{\delta\tF_{N}}{\delta\talphab}, \tgammab} \tmu
            &= \int_{\surfh} \tscal{-\tlaplace\talphab + \tbetab, \tgammab}\tmu
             + \frac{1}{2}\sum_{s\in\{+,-\}}s\int_{\Upsilon^{s}} \tgamma^{I}\talpha_{I:\xi} \mu_{\Upsilon^{s}} \\
            &= \int_{\surfh} \tscal{-\tlaplace\talphab + \tbetab, \tgammab}\tmu \quad\text{ for (NPNC)} \\
      \int_{\surfh} \tscal{\frac{\delta\tF_{H}}{\delta\talphab}, \tgammab} \tmu
            &= \int_{\surfh} \tscal{-\tlaplace\talphab + \tbetab, \tgammab}\tmu \\
            &\quad+\frac{1}{2}\sum_{s\in\{+,-\}}s\int_{\Upsilon^{s}} 
                    \tgamma_{\xi}\tensor{\talpha}{^{I}_{:I}}
                  +\tgamma^{I}\left( \talpha_{I:\xi} - \talpha_{\xi : I} \right) \tmu \\
            &= \int_{\surfh} \tscal{-\tlaplace\talphab + \tbetab, \tgammab}\tmu \quad\text{ for (NPHC)}
    \end{align}
    Therefor, the EL equations give us the boundary value problems formulated at the beginning of this section.

    In the next two sections, we will show, that the limiting process \( h\rightarrow 0 \) commutes with the EL equation for our two functionals and associated boundary conditions,
    \ie, there is no difference between first \( \tF\left[ \talphab \right] \rightarrow F\left[ \alphab \right] \) under BCs and than formulate the EL equation on the surface without
    boundaries, or first formulate the EL equation in the thin shell and than do the limiting process under BCs to the surface.
    We leave a general statement for arbitrary action functionals and convenient BCs as an open question.

    \subsubsection{Limit of the Functionals}
      First we approximate all necessary derivations at the surface and incorporate the BCs.
      \begin{align}
        \talpha_{i:j}\big|_{\surf} &= \partial_{j}\talpha_{i}\big|_{\surf} - \tch{ij}{K}\talpha_{K}\big|_{\surf}
                       = \alpha_{i|j} - B_{ij}\talpha_{\xi}\big|_{\surf}\\
                       &= \left[ \nabla\alphab \right]_{ij} + \landau(h^{2})\quad\text{ for (NP)} \\
        \talpha_{\xi :j}\big|_{\surf} &= \partial_{j}\talpha_{\xi}\big|_{\surf} - \tch{j\xi}{K}\talpha_{K}\big|_{\surf}
                                       = \partial_{j}\talpha_{\xi}\big|_{\surf} + \tensor{B}{_{j}^{k}}\alpha_{k}\\
                                      &= \left[ \Bb\alphab \right]_{j} + \landau(h^{2})\quad\text{ for (NP)}\\
        \talpha_{i :\xi}\big|_{\surf} &= \partial_{\xi}\talpha_{i}\big|_{\surf} - \tch{i\xi}{K}\talpha_{K}\big|_{\surf}
                                       =\partial_{\xi}\talpha_{i}\big|_{\surf} + \tensor{B}{_{i}^{k}}\alpha_{k}\\
                                      &=
                                          \begin{cases}
                                            \landau(h^{2}) & \text{ for (NPNC)} \\
                                            \left[ \Bb\alphab \right]_{i} + \landau(h^{2}) & \text{ for (NPHC)}
                                          \end{cases} \\
        \talpha_{\xi :\xi}\big|_{\surf} &= \partial_{\xi}\talpha_{\xi}\big|_{\surf} = \landau(h^{2})\quad\text{ for (NP)}
      \end{align}
      Hence, we obtain for the limit cases
      \begin{align}
        \frac{1}{h}\tF_{N}\left[ \talphab \right]
            &= \frac{1}{2h} \int_{\surfh} \left\| \nabla\alphab \right\|^{2} + \left\| \Bb\alpha \right\|^{2} 
                    + 2\left\langle \betab,\alphab \right\rangle + \landau\left( \xi + h^{2} \right)\tmu\\
            &\rightarrow \frac{1}{2}\int_{\surf} \left\| \nabla\alphab \right\|^{2} + \left\| \Bb\alpha \right\|^{2} + 2\left\langle \betab,\alphab \right\rangle\mu
             =: F_{N}\left[ \alphab \right]\\
        \frac{1}{h}\tF_{H}\left[ \talphab \right]
            &= \frac{1}{2h} \int_{\surfh} \left( \div\alphab \right)^{2} 
                                           + \left\| \nabla\alphab \right\|^{2} + 2\left\| \Bb\alpha \right\|^{2} 
                            - \left\langle \nabla\alphab, \left( \nabla\alphab \right)^{T} \right\rangle \\
                            &\hspace{40pt} - 2\left\| \Bb\alpha \right\|^{2} + 2\left\langle \betab,\alphab \right\rangle
                                          + \landau\left( \xi + h^{2} \right)\tmu \\
            &\rightarrow \frac{1}{2}\int_{\surf} \left( \div\alphab \right)^{2} + \left( \rot\alphab \right)^{2} + 2\left\langle \betab,\alphab \right\rangle \mu
            =: F_{H}\left[ \alphab \right]
      \end{align}
      Therefor, the EL equations for \( F_{N} \) and \( F_{H} \) on the surface \( \surf \) are
      \begin{align}
        \Delta^{dG}\alphab - \Bb^{2}\alphab &= \betab &&\text{for (NPNC)} \formComma\\
        \Delta\alphab &= \betab &&\text{for (NPHC)} \formComma
      \end{align}
      where \( \Delta \) is the Laplace-DeRham operator and \( \Delta^{dG}\alphab = \div\left( \nabla\alphab \right) = \Delta\alphab + \gaussc\alphab \),
      so we can write down equivalently 
      \begin{align}
        \Delta\alphab + 2\gaussc\alphab - \meanc\Bb\alphab &= \betab &&\text{for (NPNC)} \formPeriod
      \end{align}

    \subsubsection{Limit of the EL Equations}
      The Laplace operator is given by
      \begin{align}
        \left[ \tlaplace\talphab \right]_{I}
            &= \tensor{\talpha}{_{I}^{:J}_{:J}}
             = \partial_{J}\tensor{\talpha}{_{I}^{:J}} + \tch{JK}{J}\tensor{\talpha}{_{I}^{:K}} - \tch{JI}{L}\tensor{\talpha}{_{L}^{:J}} \\
            &= \partial_{j}\tensor{\talpha}{_{I}^{:j}} + \tch{jK}{j}\tensor{\talpha}{_{I}^{:K}} - \tch{jI}{L}\tensor{\talpha}{_{L}^{:j}}
                +\partial_{\xi}\talpha_{I:\xi} - \tch{\xi I}{l}\talpha_{l:\xi}\formPeriod
      \end{align}
      We observe, that we need a higher order expansion on \( \talpha_{i:\xi} \), that is
      \begin{align}
        \talpha_{i:\xi} &= \partial_{\xi}\talpha_{i} + \tensor{B}{_{i}^{j}}\talpha_{j} + \xi\tensor{\left[ \Bb^{2} \right]}{_{i}^{j}}\talpha_{j} +\landau(\xi^{2})\formPeriod
      \end{align}
      Hence, the normal partial derivative evaluatet on \( \surf \) is
      \begin{align}
        \partial_{\xi}\talpha_{i:\xi}\big|_{\surf} &= \partial_{\xi}^{2} \talpha_{i}\big|_{\surf} 
                            + \tensor{B}{_{i}^{j}}\partial_{\xi}\talpha_{j}\big|_{\surf}
                            + \tensor{\left[ \Bb^{2} \right]}{_{i}^{j}}\alpha_{j}\\
               &=
                  \begin{cases}
                    \landau(h^{2}) & \text{ for (NPNC)} \\
                    \left[ \Bb^{2}\alphab \right]_{i} + \landau(h^{2}) & \text{ for (NPHC)}
                  \end{cases} \formPeriod
      \end{align}
      Apply all necessary identities from this section, we obtain for the tangential components on \( \surf \)
      \begin{align}
        \left[ \tlaplace\talphab \right]_{i}\big|_{\surf}
            &= \partial_{j} \tensor{\talpha}{_{i}^{:j}}\big|_{\surf} + \ch{jk}{j}\tensor{\talpha}{_{i}^{:k}}\big|_{\surf} 
                - \tensor{B}{_{j}^{j}}\talpha_{i:\xi}\big|_{\surf} - \ch{ij}{l} \tensor{\talpha}{_{l}^{:j}}\big|_{\surf}
                -B_{ij} \tensor{\talpha}{_{\xi}^{:j}}\big|_{\surf} \\
                &\quad+ \partial_{\xi}\talpha_{i:\xi}\big|_{\surf}
                +\tensor{B}{_{i}^{l}}\talpha_{l:\xi}\big|_{\surf}\\
            &= \left[ \Delta^{dG}\alphab \right]_{i} - \left[ \Bb^{2}\alphab \right]_{i} +
                \begin{cases}
                  \landau(h^{2}) & \text{ for (NPNC)} \\
                  2\left[ \Bb^{2}\alphab \right]_{i} - \meanc\left[ \Bb\alphab \right]_{i} +\landau(h^{2}) & \text{ for (NPHC)} 
                \end{cases} \\
            &=
              \begin{cases}
                \left[ \Delta^{dG}\alphab - \Bb^{2}\alphab\right]_{i} + \landau(h^{2}) & \text{ for (NPNC)} \\
                \left[ \Delta\alphab\right]_{i} + \landau(h^{2}) & \text{ for (NPHC)} \formPeriod
              \end{cases} 
      \end{align}
      This means for the limit case \( h \rightarrow 0 \)
      \begin{align}
        \Pi\left[\tlaplace\talphab = \tbetab  \right] &\longrightarrow 
            \begin{cases}
                \Delta^{dG}\alphab - \Bb^{2}\alphab = \betab & \text{ for (NPNC)} \\
                \Delta\alphab = \betab & \text{ for (NPHC)} \formComma
              \end{cases}
      \end{align}
      and therefor commutes with our results from the previous section.

      For the normal part, we derive
      \begin{align}
        \left[ \tlaplace\talphab \right]_{\xi}\big|_{\surf}
            &= \partial_{j}\tensor{\talpha}{_{\xi}^{:j}}\big|_{\surf} + \ch{jk}{j}\tensor{\talpha}{_{\xi}^{:k}}\big|_{\surf}
              - \tensor{B}{_{j}^{j}}\partial_{\xi}\talpha_{\xi}\big|_{\surf}
              +  \tensor{B}{_{j}^{l}}\tensor{\talpha}{_{l}^{:j}}\big|_{\surf} + \partial_{\xi}^{2}\talpha_{\xi}\big|_{\surf}\\
            &= \div(\Bb\alphab) + \left\langle \Bb,\nabla\alphab \right\rangle + \partial_{\xi}^{2}\talpha_{\xi}\big|_{\surf} + \landau(h^{2})
              \quad\text{ for (NP)}\formComma
      \end{align}
      and hence, for \( q:=\partial_{\xi}^{2}\talpha_{\xi}\big|_{\surf} \) and \( r:= \tbeta_{\xi}\big|_{\surf} \)
      \begin{align}
        \nub\cdot\left[\tlaplace\talphab = \tbetab  \right] &\longrightarrow
               \div(\Bb\alphab) + \left\langle \Bb,\nabla\alphab \right\rangle + q
                  =  r  \quad\text{ for (NP)}\formPeriod
      \end{align}
      But we see that the normal part equation is decoupled completely from the tangential part by the additional scalar valued degree of freedom \( q \)
      and this equation can be omitted as a consequence.

\bibliography{bibl}
\bibliographystyle{alpha}


\end{document}
