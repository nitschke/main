\usepackage[german]{babel}
\usepackage[latin1]{inputenc}
\usepackage{csquotes}
\usepackage{times}
\usepackage{txfonts}
\usepackage{amsmath}
\usepackage{amsfonts}
\usepackage{upgreek}
\usepackage{amssymb}
\usepackage{multimedia}
\usepackage{rotating}
\usepackage{graphicx}
\usepackage{xy}
\usepackage{paralist}
\usepackage{pdfpages}
% \usepackage[autoplay,every=1,type=png]{animate} % \usepackage[every=1]{animate}
\usepackage{pdfanim}
\setlength{\fboxrule}{5pt}
\usepackage{tikz}
\usetikzlibrary{arrows,decorations.pathmorphing,backgrounds,positioning,fit,petri,shapes,topaths}
\usetikzlibrary{calc}
\usepackage{multirow}

\usepackage{tensor}
\usepackage{multicol}

% %%%%%%%%%%%%%%%%%%%%%%%%%%%%%%%%%%%%%%%%%%%%%%% USER DEFINED COMMANDS AND PATHS %%%%%%%%%%%%%%%%%%%%%%%%%%%%%%%%%%
\newcommand{\insertpicture}[2]{\begin{minipage}{#1}\centering\includegraphics[width=\textwidth]{#2}\end{minipage}}
%%%%%%%%%%%%%%%%%%%%%%%%%%%%%%%%%%%%%%%%%%%%%%% COLOR DEFINITIONS %%%%%%%%%%%%%%%%%%%%%%%%%%%%%%%%%%%%%%%%%%%%%%%%
\definecolor{tublue}{rgb}{0.04,0.16,0.32}
\definecolor{greycust}{rgb}{0.50,0.50,0.50}
\definecolor{gwhite}{rgb}{0.99,0.99,0.99}
\definecolor{tubluelight}{rgb}{0.12,0.45,0.87}
\definecolor{altcol}{rgb}{0.8,0.0,0.0}      % AlertColor
\definecolor{mygray}{gray}{0.8}
\definecolor{mygray2}{gray}{0.6}
\definecolor{headgray}{gray}{0.4}
\newcommand{\red}{\color{red}}
\newcommand{\black}{\color{black}}
\newcommand{\tublue}{\color{tublue}}
\setbeamercolor{normal text}{fg=tublue}%
\setbeamercolor{structure}{fg=tublue}%
\setbeamercolor{alerted text}{fg=tubluelight}%
\setbeamercolor{section in toc}{fg=tublue}%
\setbeamercolor{item}{fg=tubluelight}%
\setbeamercolor{block title}{fg=tubluelight}%
\setbeamercolor{footline}{fg=greycust}
\setbeamercolor{background canvas}{bg=}
%%%%%%%%%%%%%%%%%%%%%%%%%%%%%%%%%%%%%%%%%%%%%%% TITLE CONFIGURATION %%%%%%%%%%%%%%%%%%%%%%%%%%%%%%%%%%%%%%%%%%%%%%
\newcommand{\nframetitle}[1]{\frametitle{\hspace{9.4mm}\textcolor{tublue}{#1}}\vspace{-8mm}\vfill} % \put(3.5,4.8){\tiny \textbf{Department of Mathematics~}Institute of Scientific Computing}
\newcommand{\newframesubtitle}[1]{\framesubtitle{\hspace{7mm} \textcolor{tublue}{#1}}}
\newcommand{\newframesubtitletwo}[1]{\frametitle{\centering \Large #1}}
%%%%%%%%%%%%%%%%%%%%%%%%%%%%%%%%%%%%%%%%%%%%%%% HEADLINE %%%%%%%%%%%%%%%%%%%%%%%%%%%%%%%%%%%%%%%%%%%%%%%%%%%%%%%%%
\setbeamertemplate{navigation symbols}{}
\useoutertheme[hoptionsi]{miniframes}

% HEADER STYLE (16:10)
\setbeamertemplate{headline}{
	\begin{beamercolorbox}{section in head}
		\vskip4pt
		%%% TU LOGO
		\begin{minipage}{33.5mm}
			{\rule{2.5mm}{0pt}\rule[-0.5mm]{0pt}{3.5mm}\includegraphics[width=30mm]{pic/logo/tu_blau.pdf}}
		\end{minipage}
		\hfill
		%%% NAVIGATION ITEMS
		\begin{minipage}{85mm} % ADJUST THE WIDTH OF THIS MINIPAGE TO YOUR CONTENT
			{\rule{4.5mm}{0pt}\insertnavigation{83mm}} % AND THIS
		\end{minipage}
		\hfill
		%%% ADDITIONAL LOGO (if needed)
		\begin{minipage}{15mm}
			{\rule{2.5mm}{0pt}
			%\includegraphics[height=10mm]{logo_ilr.pdf}
		}
		\end{minipage}
		\vskip2pt
		% DIVIDER BETWEEN HEADER AND CONTENT (TWO LINES)
		\begin{minipage}{128mm}
			\setlength{\unitlength}{1mm}
			\begin{picture}(128,1.5)
			\linethickness{.1mm} \put(-10,1.3){\line(1,0){175}}
			\linethickness{.1mm} \put(-10,0){\line(1,0){175}}
			\end{picture}
		\end{minipage}
	\end{beamercolorbox}
} % 197.2 x 166.5
% FOOTER STYLE
\setbeamertemplate{footline}{
	\vfill
	\vspace{-10mm}
	\begin{minipage}{\textwidth}
		\rule{\textwidth}{.1mm}
		\vskip2pt 
		$\quad$\tiny Institute of Scientific Computing 
		\hfill
		\insertframenumber%\inserttotalframenumber
		$\quad$\\
		\vskip3pt
	\end{minipage}
}
\setbeamersize{text margin left=12.9mm,text margin right=12.9mm}

% Bibliography
\usepackage[backend=bibtex,style=alphabetic,language=american]{biblatex}
\bibliography{lib}
\usepackage[draft]{todonotes}   % notes showed

%% FOOTNOTE INDENTATION
\makeatletter
%\renewcommand\@makefntext[1]{\leftskip=2em\hskip-2em\@makefnmark#1}
\renewcommand\@makefntext[1]{\leftskip=10pt\hskip-3.7pt\@makefnmark#1}
\makeatother

% user commands 
\newcommand{\eg}{e.\,g.}%
\newcommand{\formComma}{\,\text{,}}
\newcommand{\formPeriod}{\,\text{.}}
\newcommand{\R}{\mathbb{R}}%
\newcommand{\xb}{\mathbf{x}}%
\newcommand{\ub}{\mathbf{u}}%
\newcommand{\tub}{\tilde{\ub}}%
\newcommand{\tu}{\tilde{u}}%
\newcommand{\vb}{\mathbf{v}}%
\newcommand{\eb}{\mathbf{e}}%
\newcommand{\gb}{\mathbf{g}}%
\newcommand{\Xb}{\mathbf{x}}% parametrization X:{(theta,phi)} --> R3
\newcommand{\fv}{\underline{\mathbf{f}}}
\newcommand{\pv}{\underline{\mathbf{p}}}
\newcommand{\scalarprod}[1]{\big\langle{#1}\big\rangle}%
\newcommand{\exd}{\mathbf{d}} %exterior derivative
\newcommand{\lie}{\mathcal{L}} %Lie-Ableitung
\newcommand{\surf}{\mathcal{S}}
\newcommand{\gaussianCurvature}{\kappa}
\newcommand{\Grad}{\operatorname{grad}}
\newcommand{\Div}{\operatorname{div}}%
\newcommand{\Rot}{\operatorname{rot}}%
\newcommand{\DivSurf}{\Div_{\surf}}%
\newcommand{\GradSurf}{\Grad_{\surf}}
\newcommand{\RotSurf}{\Rot_{\surf}}
\newcommand{\laplaceBeltrami}{\Delta_{\surf}}
\newcommand{\vecLaplace}{\boldsymbol{\Delta}}
\newcommand{\laplaceDeRham}{\vecLaplace^{\textup{dR}}}
\newcommand{\laplaceRotRot}{\vecLaplace^{\textup{RR}}}
\newcommand{\laplaceGradDiv}{\vecLaplace^{\textup{GD}}}
\newcommand{\LaplaceDeRham}{Laplace-deRham }
\newcommand{\ProjectSurf}{\pi_\surf}
\newcommand{\Tangent}{\mathsf{T}}
\newcommand{\vect}[1]{\mathbf{#1}}
\newcommand{\landau}{\mathcal{O}} %landau symbol O
\newcommand{\SC}{\mathcal{K}} % simplicial complex
\newcommand{\Vs}{\mathcal{V}} % set of vertices
\newcommand{\Es}{\mathcal{E}} % set of edges
\newcommand{\Fs}{\mathcal{T}} % set of faces
\newcommand{\face}{T} % one face
\newcommand{\FormSpace}{\Lambda^{1}} % space of 1 forms
\newcommand{\U}{u} %Komponenten des Geschwindigkeitfeldes
\newcommand{\Ub}{\mathbf{\U}} %Geschwindigkeitsfeld
\newcommand{\tU}{\tilde{u}} %Komponenten des Geschwindigkeitfeldes
\newcommand{\tUb}{\mathbf{\tU}} %Geschwindigkeitsfeld
\newcommand{\lc}{\mathbf{E}} % Levi-Civita-Tensor

