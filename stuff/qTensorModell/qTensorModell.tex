\documentclass[a4paper,11pt]{scrartcl}

\usepackage{ngerman}
\usepackage[utf8]{inputenc}
%\usepackage[T1]{fontenc}

\usepackage{hyperref}
\usepackage{color}
\usepackage{amsmath}
\usepackage{amsfonts}
\usepackage{amssymb}
\usepackage{amsbsy}
\usepackage{bm}
\usepackage{graphicx}
\usepackage{extarrows}

\newcommand{\ie}{i.\,e.}%
\newcommand{\eg}{e.\,g.}%
\newcommand{\etal}{et\,al.}%
\newcommand{\wrt}{w.r.t.}%
\newcommand{\sign}{\operatorname{sign}}

\newcommand{\leviCivita}{\varepsilon}

% comma and period at end of formulas
\newcommand{\formComma}{\,\text{,}}
\newcommand{\formPeriod}{\,\text{.}}

% real and complex numbers
\newcommand{\R}{\mathbb{R}}%
% \newcommand{\C}{\mathbb{C}}%

% bold symbols
\newcommand{\xb}{\mathbf{x}}%
\newcommand{\rb}{\mathbf{r}}%
\newcommand{\ub}{\mathbf{u}}%
\newcommand{\nb}{\mathbf{n}}%
\newcommand{\eb}{\bm{e}}%
\newcommand{\gb}{\mathbf{g}}%
\newcommand{\Xb}{\mathbf{x}}% parametrization X:{(theta,phi)} --> R3

\newcommand{\alphab}{\bm{\alpha}}% for \pb^{\flat}
\newcommand{\betab}{\bm{\beta}}% for \qb^{\flat}

\newcommand{\eqAngle}{\phi}

% director and its components
\newcommand{\pb}{\mathbf{p}}%
\newcommand{\p}{\textup{p}}%
% test funciton and its components
\newcommand{\q}{\textup{q}}%
\newcommand{\qb}{\mathbf{q}}%
\newcommand{\tb}{\mathbf{t}}%
\newcommand{\Eb}{\mathbf{E}}%
\newcommand{\Mbb}{\mathbf{M}}%
\renewcommand{\sb}{\mathbf{s}}%
\newcommand{\vb}{\mathbf{q}}%to unify test functions

\newcommand{\alphav}{\underline{\bm{\alpha}}} % for vector-1-forms, like PD-1-form
\newcommand{\betav}{\underline{\bm{\beta}}} % for vector-1-forms, like PD-1-form

\newcommand{\abs}[1]{\lvert#1\rvert}%
\newcommand{\norm}[1]{\lVert#1\rVert}%
\newcommand{\scalarprod}[1]{\big\langle{#1}\big\rangle}%
\newcommand{\Scalarprod}[1]{\left\langle{#1}\right\rangle}%

% derivatives
\newcommand{\dif}{\textup{d}}
\newcommand{\exd}{\mathbf{d}} %exterior derivative
\newcommand{\ds}{\,\dif{s}}
\newcommand{\dr}{\,\dif\rb}
\newcommand{\dx}{\,\dif\xb}
\newcommand{\dxhat}{\,\dif\hat{\xb}}
\newcommand{\dt}{\partial_t}

% differential forms for integration
\newcommand{\dS}{\,\dif{\surf}}
\newcommand{\dV}{\,\dif{V}}

% functional derivative
\newcommand{\fdif}{\operatorname{\delta}\!}
\newcommand{\Fdif}[2]{\frac{\fdif{#1}}{\fdif{#2}}}% dF / du	
\newcommand{\FFdif}[3]{\frac{\fdif^2{#1}}{\fdif{#2}\fdif{#3}}}% d^2F / dudv	

\newcommand{\FF}{\mathrm{F}}
\newcommand{\F}[1]{\FF_\mathrm{#1}}

% surface and domain
\newcommand{\Sp}{\mathbb{S}^2}
\newcommand{\ellipsoid}{\mathcal{E}}
\newcommand{\surf}{\mathcal{S}}
\newcommand{\domain}{\Omega}

\newcommand{\meanCurvature}{\mathcal{H}}
\newcommand{\gaussianCurvature}{\kappa}

% surface (differential operators)
\newcommand{\Grad}{\operatorname{Grad}}
\renewcommand{\div}{\operatorname{div}}%
\newcommand{\Rot}{\operatorname{Rot}}%
\newcommand{\rot}{\operatorname{rot}}%
\newcommand{\DivSurf}{\Div_{\surf}}%
\newcommand{\GradSurf}{\Grad_{\surf}}
\newcommand{\laplace}{\Delta}
\newcommand{\laplaceBeltrami}{\Delta_{\surf}}
\newcommand{\vecLaplace}{\boldsymbol{\Delta}}
%\newcommand{\laplaceDeRahm}{\vecLaplace^{\textup{dR}}_{\surf}}
\newcommand{\laplaceDeRham}{\vecLaplace^{\textup{dR}}}
\newcommand{\laplaceDeRahm}{\laplaceDeRham}
\newcommand{\laplaceRotRot}{\vecLaplace^{\textup{RR}}}
\newcommand{\laplaceGradDiv}{\vecLaplace^{\textup{GD}}}

\newcommand{\laplaceDeRhamTilde}{\widehat{\vecLaplace}^{\textup{dR}}}
\newcommand{\NablaSurf}{\nabla_{\surf}}
\newcommand{\gDerivative}{D}

\newcommand{\laplaceDeRhamDiffuse}{\widehat{\vecLaplace}^{\textup{dR}}_{\phi}}

\newcommand{\surfNormal}{\boldsymbol{\nu}}
\newcommand{\surfNormalI}{\nu}

\newcommand{\ProjectSurf}{\pi_\surf}

\newcommand{\Tangent}{\mathsf{T}}

% General
\newcommand{\vect}[1]{\mathbf{#1}}
\newcommand{\tensor}[1]{\mathbf{#1}}

\newcommand{\Span}[1]{\operatorname{Span}\!\left\{ #1 \right\}}

\newcommand{\AMDIS}{\texttt{AMDiS}}
\newcommand{\PETSC}{\texttt{PETSc}}

% Constants
\newcommand{\K}{{K}} % one-constant
\newcommand{\Ki}{{K_1}} % frank-constant K1
\newcommand{\Kii}{{K_2}} % frank-constant K2
\newcommand{\Kiii}{{K_3}} % frank-constant K3
\newcommand{\Kn}{{\omega_n}} % penalty constant for normality
\newcommand{\Kt}{{\omega_t}} % penalty constant for tangentiality

% Energies
\newcommand{\EF}{E^{\textup{F}}} % 3D Frank-Energy
\newcommand{\EFSurf}{E^{\textup{F}}_{\surf}} % Frank-Energy on surface
\newcommand{\ESurf}{E_{\surf}} % intrinsic Frank-Energy plus normalizing term on surface

\newcommand{\EE}{\F{\omega}^\surf}
\newcommand{\E}[1]{\EE[#1]}


\newcommand{\Ltwo}[2]{L^2( #1;\, #2)}%
\newcommand{\LS}{L^2(\surf)}%
\newcommand{\LtwoProd}[1]{\big({#1}\big)_{\LS}}%

\newcommand{\C}[2]{{C}( #1;\, #2)}
\newcommand{\Csurf}[1]{{C}^{#1}(\surf)}
\newcommand{\Cdomain}[1]{{C}^{#1}(\domain)}
\newcommand{\extend}[1]{\widehat{#1}}

\newcommand{\Hdr}[2]{H^{\textup{DR}}( #1;\, #2)} %
\newcommand{\HdrExt}[2]{H^{\textup{DR}}( #1;\, #2)} % \extend{H}^{\textup{DR}}\left( #1 ; #2 \right)
\newcommand{\HdrDiffuse}[2]{H^{\textup{DR}}( #1;\, #2)} %\extend{H}^{\textup{DR}}_{\phi}\left( #1 ; #2 \right)

\newcommand{\Hi}[1]{H^{1}(#1)}

\newcommand{\pExt}{\extend{\pb}}
\newcommand{\qExt}{\extend{\qb}}
\newcommand{\nExt}{\extend{\surfNormal}}

% rotation angle
\newcommand{\rotAngle}{\psi}

% DEC declarations
\newcommand{\SC}{\mathcal{K}} % simplicial complex
\newcommand{\Vs}{\mathcal{V}} % set of vertices
\newcommand{\Es}{\mathcal{E}} % set of edges
\newcommand{\Fs}{\mathcal{T}} % set of faces
\newcommand{\face}{T} % one face
\newcommand{\FormSpace}{\Lambda^{1}} % space of 1 forms
\newcommand{\PDT}{\mathfrak{T}} % PD-Tangential-Space

% for commends among each other
\newcommand{\simon}[1]{{\color{red}#1}}
\newcommand{\ingo}[1]{{\color{blue}#1}}
\newcommand{\micha}[1]{{\color{green}#1}}

\newcommand{\hidden}[1]{}

%\theoremstyle{plain}
\newtheorem{lem}{Lemma}
\newtheorem{thm}{Theorem}
\newtheorem{prop}[thm]{Proposition}
\newtheorem{rem}{Remark}[section]
\newtheorem{exmp}{Example}[section]

%\theoremstyle{definition}
\newtheorem{defn}{Definition}

\makeatother
\providecommand*{\thmautorefname}{Theorem}
\providecommand*{\remautorefname}{Remark}
\providecommand*{\defnautorefname}{Definition}
\providecommand*{\algorithmautorefname}{Algorithm}
%\addto\extrasenglish{%
  %\renewcommand{\chapterautorefname}{Chapter}%
  %\renewcommand{\sectionautorefname}{Section}%
  %\renewcommand{\subsectionautorefname}{Section}%
  %\renewcommand{\subsubsectionautorefname}{Section}%
  %\renewcommand{\paragraphautorefname}{Section}%
%}
\newcommand*{\End}{\hfill\ensuremath{\vartriangleleft}}%
\newcommand*{\END}{\hfill\ensuremath{\blacktriangleleft}}%




\usepackage{tensor}

\newcommand{\qspace}{\mathcal{Q}(\surf)}
\newcommand{\qs}{\mathcal{Q}}
\newcommand{\tspace}{\mathcal{T}^{(2)}(\surf)}
\newcommand{\Tr}{\text{Tr}}
\newcommand{\qproject}{\pi_{\qspace}}
\newcommand{\Mb}{\mathbf{M}}
\newcommand{\RSpan}[1]{\text{Span}_{\R}\left\{ #1 \right\}}

\newcommand{\lGd}{\Delta^{\text{Gd}}}
\newcommand{\ldG}{\Delta^{\text{dG}}}
\newcommand{\lRr}{\Delta^{\text{Rr}}}

\newcommand{\fid}{f_{\text{id}}}
\newcommand{\fexc}{f_{\text{exc}}}

\newcommand{\Fid}{F_{\text{id}}}
\newcommand{\Fexc}{F_{\text{exc}}}

\title{Notizen zum Q-Tensor-Modell} 
\author{Ingo Nitschke}

\begin{document}
\maketitle

(Beachte Notations- und Operatordefinitionen im Appendix)

\section{Q-Tensor Fakten}

Q-Tensoren auf Oberflächen \( \surf \) (ohne Rand) sind Tensoren 2. Stufe, 
d.h. \( \qspace \subset \tspace \simeq \Tangent\surf\otimes\Tangent\surf \).  
Es gilt
\begin{align}
  \qspace :&= \left\{ \qb\in \tspace \middle| \Tr\qb=0, \qb^{T}=\qb \right\}\\
          &= \left\{ \qb\in \tspace \middle| \qb:\gb= \qb:\Eb = 0 \right\} \formPeriod
\end{align}
Der Raum \( \tspace \) lässt sich orthogonal zerlegen in
\begin{align}
  &\tspace = \qspace \cup \mathcal{Q}^{\bot}(\surf) \text{ mit } \\
&\qspace \cap \mathcal{Q}^{\bot}(\surf) = \left\{ \mathcal{O} \right\} \text{ und }\\
   &\qspace : \mathcal{Q}^{\bot}(\surf) = 0\formComma
\end{align}
wobei
\begin{align}
  \mathcal{Q}^{\bot}(\surf) = \RSpan{\gb,\Eb} = \R\text{SO}(\Tangent\surf)
\end{align}
ist der zweidimensionale \( \R \)-Vektorraum über alle (Oberflächen-)Drehungen und
\begin{align}
  \qspace = \RSpan{\Mbb, *\Mbb} = \R(\text{O}(\Tangent\surf)\setminus\text{SO}(\Tangent\surf))
\end{align}
ist der zweidimensionale \( \R \)-Vektorraum über alle (Oberflächen-)Drehspiegelungen \( \Mbb \) und \( *\Mbb \).
Für die Basistensoren gilt \( \left\| \gb \right\| = \left\| \Eb \right\| = \left\| \Mbb \right\| = \left\| *\Mbb \right\| = \sqrt{2}\)
und alle vier Basistensoren sind orthogonal bzgl. dem Doppelpunktprodukt ''\( : \)''.
(Im Gegensatz zu \( \mathcal{Q}^{\bot}(\surf) \) bildet \( \qspace \) mit der Tensormultiplikation ''\( \cdot \)'' keine multiplikative Gruppe, 
da das neutrale Element fehlt.)

Eine orthogonale Projektion \( \pi_{\qs}:\tspace\rightarrow\qspace \) ergibt sich dementsprechend durch
\begin{align}
  \pi_{\qs}(\tb) &= \tb - \frac{\tb:\gb}{2}\gb - \frac{\tb:\Eb}{2}\Eb
                  = \frac{1}{2}\left( \tb + *\tb* \right)
                  = \frac{1}{2}\left( \tb + \tb^{T} - (\Tr\tb)\gb \right) \formPeriod
\end{align}
Es lässt sich zeigen, dass für alle \( \qb\in\qspace \)
\begin{align}
  \Delta\qb = \lGd\qb + \lRr\qb = 2\pi_{\qs}(\lGd\qb) = 2\pi_{\qs}(\lRr\qb)
\end{align}
gilt.
Für Q-Tensoren \( \qb\in\qspace \) ist die Rotation und Divergenz zu einander Hodgedual (orthogonal und gleichlang), d.h.
\( *\rot\qb = \div\qb \) und somit \( \left\| \rot\qb \right\| = \left\| \div\qb \right\| \).

\section{Q-Tensor-Modell}
Ausgehend von einem freien Energiefunktional auf \( \qspace \)
\begin{align}
  F[\qb] = \int_{\surf}f[\qb,\Grad\qb]\mu
\end{align}
stellen wir folgende Mindestanforderungen an die Energiedichte \( f \):
\begin{itemize}
  \item \( f \) soll Koordinatenunabhängig sein (Kovarianzprinzip/Forminvarianz), 
      d.h. jeder Koordinatenwechsel ändert nichts an das ''Wirken'' von \( f \).
      Somit ist \( f \) eine ''physikalisch sinnvolle'' Abbildung.
  \item \( f \) soll gerade sein, d.h. \( f[\qb,\Grad\qb] = f[-\qb,-\Grad\qb] \)
  \item \( f \) soll positiv definit sein, d.h. \( f\ge 0 \)
\end{itemize}
Wir zerlegen die Energiedichte zu
\begin{align}
  f[\qb,\Grad\qb] = \fid[\qb] + \fexc[\Grad\qb]\formComma
\end{align}
so dass \( \fid\) (alle linear unabhängigen) Kontraktionen in \( \qb \) bis zur 4. Ordnung enthält und
\( \fexc \) (alle linear unabhängigen) Kontraktionen in \( \Grad\qb \) bis zur 2. Ordnung enthält.
Außerdem sollen an \( \fid \) und \( \fexc \) die selben Mindestanforderungen wie an \( f \) gestellt sein.
Wir legen fest
\begin{align}
  \fexc[\Grad\qb] :&=  \frac{C}{2}\left( \left\| \div\qb \right\|^{2} + \left\| \rot\qb \right\|^{2} \right) \\
                  &=C\left\| \div\qb \right\|^{2} = C\left\| \rot\qb \right\|^{2} \\
                  &= \frac{C}{2}\left( \left\| \Grad\qb \right\|^{2} - 2\mathcal{K}\left\| \qb \right\|^{2} \right) \formComma
\end{align}
wobei die 2. Zeile aus \( *\rot\qb = \div\qb \) (s.o.) folgt
und die 3. Zeile eine (schwache) Folgerung aus der Weizenböck-Identität
\begin{align}
  \Delta\qb = \ldG\qb - 2\mathcal{K}\qb
\end{align}
für Q-Tensoren \( \qb\in\qspace \) ist. 
Ableiten bzgl. dem \( L^{2} \) Skalarprodukt über \( \qspace \)
\begin{align}
  \left\langle \sb , \tb \right\rangle_{L^{2}(\qspace)} := \int_{\surf} \sb : \tb \mu 
    \qquad (\sb,\tb\in\qspace)
\end{align}
ergibt für alle \( \tb\in\qspace \)
\begin{align}
  \left\langle \frac{\delta\Fexc}{\delta\qb}, \tb \right\rangle_{L^{2}(\qspace)}
      = \left\langle -C\Delta\qb , \tb\right\rangle_{L^{2}(\qspace)} \formPeriod
\end{align}
Weiterhin sei
\begin{align}
  \fid[\qb] :&= \frac{k_{1}}{2}\Tr\qb^{2} + \frac{k_{2}}{2}\Tr\qb^{4}
              = \frac{k_{1}}{2}\Tr\qb^{2} + \frac{k_{2}}{4}(\Tr\qb^{2})^{2} \\
             &= \frac{\left\| \qb \right\|^{2}}{2}\left( k_{1} + \frac{k_{2}}{2}\left\| \qb \right\|^{2} \right) \formPeriod
\end{align}
Die Zweite Zeile begründet sich aus der Symmetrie von \( \qb \).
Zu beachten sei hierbei, dass für alle \( \qb\in\qspace \) und \( n\in\mathbb{N} \) gilt  \( \Tr\qb^{2n+1} = 0\).
Ableiten ergibt
\begin{align}
  \left\langle \frac{\delta\Fid}{\delta\qb}, \tb \right\rangle_{L^{2}(\qspace)}
      = \left\langle \left( k_{1} + k_{2}\left\| \qb \right\|^{2} \right)\qb , \tb\right\rangle_{L^{2}(\qspace)}
\end{align}
für alle \( \tb\in\qspace \).
Setzen wir
\begin{align}
  \left\langle \frac{\delta F}{\delta\qb}, \tb \right\rangle_{L^{2}(\qspace)} 
      = \left\langle -\partial_{t}\qb, \tb \right\rangle_{L^{2}(\qspace)}
\end{align}
erhalten wir covarianten Differentialgleichungen für \( \qb\in\qspace \):
\begin{align}
  \partial_{t}\qb - C\Delta\qb + \left( k_{1} + k_{2}\left\| \qb \right\|^{2} \right)\qb = \mathcal{O} \in \qspace \formPeriod
\end{align}

\section{Diskussion}
\begin{itemize}
  \item \( \fexc = \frac{C}{2}\left( \left\| \div\qb \right\|^{2} + \left\| \rot\qb \right\|^{2} \right) 
        =  \frac{C}{2}\left( \left\| \Grad\qb \right\|^{2} - 2\mathcal{K}\left\| \qb \right\|^{2} \right)\)
        versus \( \fexc = \frac{C}{2}\left\| \Grad\qb \right\|^{2} \);
        Ein Kompromiss könnte \( \fexc = \frac{C}{2}\left\| \Grad\qb \right\|^{2} - \frac{C_{24}}{2}\mathcal{K}\left\| \qb \right\|^{2} \)
        sein. Wäre das sinnvoll? Wie beeinflusst der 0te-Ordnungsterm die Lösung im Verhältnis zu \( \fid \)?
  \item Parameterwahl \( C>0,k_{1}<0,k_{2}>0(,C_{24}\ge 0) \). 
  \item Gleichungsreduktion:
        \begin{itemize}
          \item Nutze Spurfreiheit und Symmetrie der Differentialgleichung oben. 
                (Nachteil: Komponentenweisebetrachtung ist i.A. nicht Koordinatenunabhängig)
          \item \( [\partial_{t}\qb - C\Delta\qb + \left( k_{1} + k_{2}\left\| \qb \right\|^{2} \right)\qb]\cdot\pb 
                = \mathcal{O}\in\mathcal{T}^{(1)}(\surf) \) mit Tangentialvektor \( \pb\in\mathcal{T}^{(1)}(\surf) \).
                (Vorteil: Koordinatenunabhängig; Im diskreten könnte man für \( \pb \) Kantenvektoren nutzen.)
        \end{itemize}
  \item Q-Tensor-Ansatz:
      \begin{itemize}
        \item  Nutze Spurfreiheit und Symmetrie zur Reduktion der Q-Tensorkoordinaten.
              (Nachteil: Koordinatenabhängig)
        \item Nutze Vektorraumstruktur von \( \qspace \),
              d.h. \( \qb = q_{1}\Mbb + q_{2}(*\Mbb) \).
              (führt evtl. ''nur'' zu zwei skalarwertigen Problemen; 
                im diskreten könnte man für \( \Mbb \) die Spiegelungen am Kantenvetor nehmen,
                somit wäre \( *\Mbb \) die Spiegelung an der Dualkante)
        \item Nutze freies Tensorprodukt: \( \qb=2\pi_{\qs}(\rho\otimes\pb)= \rho\otimes\pb -  (*\rho)\otimes(*\pb)\)
              mit beliebig aber fest gewählten \( \rho\in\mathcal{T}^{(1)}(\surf) \) und vektorwertigen Freiheitsgrad \(
              \pb\in\mathcal{T}^{(1)}(\surf) \).
              (Nachteil: u.U. komplexe Gleichungen je nach Wahl von \( \rho \);
                Es könnte wieder für \( \rho \) der Kantenvektor im diskreten gewählt werden.)
      \end{itemize}
\end{itemize}

\section{Appendix}
\subsection{Notation}
\begin{itemize}
  \item \( \simeq \) bedeutet Gleichheit bis auf die Höhe der Indizes. (semantisch gleich)
  \item \( \gb = \left\{ g_{ij} \right\} \) metrischer Tensor.
  \item \( \left| \gb \right| \) Determinante des metrischen Tensors
  \item \( \mu = \sqrt{\left| \gb \right|} dx^{i}\wedge dx^{j} \) Volumenform (2-Form)
  \item \( \Eb = \left\{ E_{ij} \right\} = \sqrt{\left| \gb \right|}\varepsilon_{ij} = \mu(\partial_{i},\partial_{j})\)
          Levi-Civita-Tensor (\( \varepsilon_{ij} \) Levi-Civita-Symbole).
  \item \( \left\{ \tensor{\Gamma}{_{ij}^{k}} \right\} 
            = \left\{ \frac{1}{2} g^{kl}\left(\partial_{i}g_{jl} + \partial_{j}g_{il} - \partial_{l}g_{ij}  \right) \right\}\)
           Christoffel-Tensor
  \item \( \mathcal{K} \) Gaußsche Krümmung
\end{itemize}

\subsection{Operationen}
\begin{itemize}
  \item Tensorprodukt (1-Punktkontraktion) von \( \tb,\sb\in\mathcal{T}^{(2)}(\surf) \):
        \begin{align}
          \left[\tb \cdot \sb\right]_{ij} = t_{ik}\tensor{s}{^{k}_{j}}\in\tspace
        \end{align}
  \item Doppelpunktprodukt (2-Punktkontraktion) von \( \tb,\sb\in\mathcal{T}^{(2)}(\surf) \):
        \begin{align}
          \tb : \sb = t_{ij}s^{ij}\in\R
        \end{align}
  \item Hodgedualer 2-Tensor (in der ersten Komponente) von \( \tb\in\tspace \):
        \begin{align}
          *\tb = -\Eb\cdot\tb \in \tspace \qquad \Rightarrow *\tb:\tb = 0
        \end{align}
  \item Transponierter Tensor \( \tb^{T}\in\mathcal{T}^{(2)}(\surf) \): 
       \begin{align}
         \left[ \tb^{T} \right]_{ij} = t_{ji} \simeq t^{ji}
       \end{align}
       (Vorsicht: Niemals Indizes unterschiedlicher Höhe tauschen!)
  \item Spur (Trace) von \( \tb\in\mathcal{T}^{(2)}(\surf) \):
    \begin{align}
      \Tr t = \tensor{t}{_{i}^{i}} = \tb:\gb \in \R
    \end{align}
  \item (Frobenius) Norm von  \( \tb\in\mathcal{T}^{(n)}(\surf) \):
    \begin{align}
      \left\| \tb \right\|^{2} = \tb \overset{(n)}{\vdots} \tb
                              = t_{i_{1}\ldots i_{n}} t^{i_{1}\ldots i_{n}}
    \end{align}
  \item Gradient von \( \pb \in \mathcal{T}^{(1)}(\surf) \simeq \Tangent\surf \):
      \begin{align}
        \left[ \Grad\pb \right]_{ij} &= \left[ \partial\pb - \Gamma\cdot\pb \right]_{ij}
                  = p_{i|j}
                  = \partial_{j}p_{i} - \tensor{\Gamma}{_{ji}^{k}}p_{k}
      \end{align}
  \item Gradient von \( \tb\in\tspace \):
    \begin{align}
        \left[ \Grad\tb \right]_{ijk} &= t_{ij|k}
              = \partial_{k}t_{ij} - \tensor{\Gamma}{_{ki}^{l}}t_{lj} - \tensor{\Gamma}{_{kj}^{l}}t_{il}
    \end{align}
  \item Gradient von \( \tb\in\mathcal{T}^{(3)}(\surf) \):
    \begin{align}
        \left[ \Grad\tb \right]_{ijkl} &= t_{ijk|l}
              = \partial_{l}t_{ijk} - \tensor{\Gamma}{_{li}^{m}}t_{mjk} - \tensor{\Gamma}{_{lj}^{m}}t_{imk} -
              \tensor{\Gamma}{_{lk}^{m}}t_{ijm}
    \end{align}
  \item Divergenz von \( \pb \in \mathcal{T}^{(1)}(\surf) \simeq \Tangent\surf \):
    \begin{align}
      \div\pb = \Tr(\Grad\pb) = (\Grad\pb):\gb = \tensor{p}{_{i}^{|i}}
    \end{align}
  \item Divergenz (in der zweiten/letzten Komponente) von \( \tb\in\tspace \):
      \begin{align}
        \left[ \div\tb \right]_{i} &= \left[ (\Grad\tb) : \gb \right]_{i} = \tensor{t}{_{ij}^{|j}}
      \end{align}
  \item Divergenz (in der dritten/letzten Komponente) von \( \tb\in\mathcal{T}^{(3)}(\surf) \):
      \begin{align}
        \left[ \div\tb \right]_{ij} &= \left[ (\Grad\tb) : \gb \right]_{ij} = \tensor{t}{_{ijk}^{|k}}
      \end{align}
  \item Divergenz (in der letzten Komponente) von \( \tb\in\mathcal{T}^{(n)}(\surf) \):
      \begin{align}
        \left[ \div\tb \right]_{i_{1}\ldots i_{n-1}} &= \left[ (\Grad\tb) : \gb \right]_{i_{1}\ldots i_{n-1}} 
                      = \tensor{t}{_{i_{1}\ldots i_{n}}^{|i_{n}}}
      \end{align}
  \item (Stufen) reduzierende Rotation (in der letzten Komponente) von \( \tb\in\tspace \):
      \begin{align}
        \left[ \rot\tb \right]_{i} &= -\left[(\Grad\tb):\Eb\right]_{i} = \left[((*\Grad)\tb):\gb  \right]_{ij}
                          = E_{kj}\tensor{t}{_{i}^{j|k}}
      \end{align}
  \item (Stufen) reduzierende Rotation (in der letzten Komponente) von \( \tb\in\mathcal{T}^{(n)}(\surf)\):
      \begin{align}
        \left[ \rot\tb \right]_{i_{1}\ldots i_{n-1}} &= -\left[(\Grad\tb):\Eb\right]_{i_{1}\ldots i_{n-1}} 
              = \left[((*\Grad)\tb):\gb  \right]_{i_{1}\ldots i_{n-1}} \\
                          &= E_{ki_{n}}\tensor{t}{_{i_{1}\ldots i_{n-1}}^{i_{n}|k}}
      \end{align}
  \item (Stufen) erweiternde Rotation von \( \pb \in \mathcal{T}^{(1)}(\surf) \simeq \Tangent\surf \):
      \begin{align}
        \left[ \Rot\pb \right]_{ij} = \left[ (\Grad\pb)\cdot \Eb \right]_{ij} = \left[ (*\Grad)\pb \right]_{ij}
                  = E_{kj}\tensor{p}{_{i}^{|k}}
      \end{align}
  \item (Stufen) erweiternde Rotation von \( \tb\in\mathcal{T}^{(n)}(\surf)\):
      \begin{align}
        \left[ \Rot\tb \right]_{i_{1}\ldots i_{n+1}} &= \left[ (\Grad\tb)\cdot \Eb \right]_{i_{1}\ldots i_{n+1}} 
                = \left[ (*\Grad)\pb \right]_{i_{1}\ldots i_{n+1}}\\
                  &= E_{ki_{n+1}}\tensor{t}{_{i_{1}\ldots i_{n}}^{|k}}
      \end{align}
  \item Grad-div-Laplace von \( \tb\in\tspace \):
      \begin{align}
        \left[ \lGd\tb \right]_{ij} = \left[ \Grad\div\tb \right]_{ij}
                      = \tensor{t}{_{i}^{k}_{|k|j}}
      \end{align}
  \item Rot-rot-Laplace von \( \tb\in\tspace \):
      \begin{align}
        \left[ \lRr\tb \right]_{ij} = \left[ \Rot\rot\tb \right]_{ij}
                      = \tensor{t}{_{ij|k}^{|k}} - \tensor{t}{_{ik|j}^{|k}}
      \end{align}
  \item div-Grad-Laplace von \( \tb\in\tspace \):
      \begin{align}
        \left[ \ldG\tb \right]_{ij} = \left[ \div\Grad\tb \right]_{ij}
                      = \tensor{t}{_{ij|k}^{|k}}
      \end{align}
  \item Laplace-Operator von \( \tb\in\tspace \):
      \begin{align}
        \Delta \tb &= \lGd\tb + \lRr\tb \overset{i.A.}{\ne} \ldG\tb
      \end{align}
\end{itemize}

\end{document}
