\documentclass[a4paper,11pt]{scrartcl}

\usepackage{ngerman}
\usepackage[utf8]{inputenc}
%\usepackage[T1]{fontenc}

\usepackage{hyperref}
\usepackage{color}
\usepackage{amsmath}
\usepackage{amsfonts}
\usepackage{amssymb}
\usepackage{amsbsy}
\usepackage{bm}
\usepackage{graphicx}
\usepackage{extarrows}

\newcommand{\ie}{i.\,e.}%
\newcommand{\eg}{e.\,g.}%
\newcommand{\etal}{et\,al.}%
\newcommand{\wrt}{w.r.t.}%
\newcommand{\sign}{\operatorname{sign}}

\newcommand{\leviCivita}{\varepsilon}

% comma and period at end of formulas
\newcommand{\formComma}{\,\text{,}}
\newcommand{\formPeriod}{\,\text{.}}

% real and complex numbers
\newcommand{\R}{\mathbb{R}}%
% \newcommand{\C}{\mathbb{C}}%

% bold symbols
\newcommand{\xb}{\mathbf{x}}%
\newcommand{\rb}{\mathbf{r}}%
\newcommand{\ub}{\mathbf{u}}%
\newcommand{\nb}{\mathbf{n}}%
\newcommand{\eb}{\bm{e}}%
\newcommand{\gb}{\mathbf{g}}%
\newcommand{\Xb}{\mathbf{x}}% parametrization X:{(theta,phi)} --> R3

\newcommand{\alphab}{\bm{\alpha}}% for \pb^{\flat}
\newcommand{\betab}{\bm{\beta}}% for \qb^{\flat}

\newcommand{\eqAngle}{\phi}

% director and its components
\newcommand{\pb}{\mathbf{p}}%
\newcommand{\p}{\textup{p}}%
% test funciton and its components
\newcommand{\q}{\textup{q}}%
\newcommand{\qb}{\mathbf{q}}%
\newcommand{\tb}{\mathbf{t}}%
\newcommand{\Eb}{\mathbf{E}}%
\newcommand{\Mbb}{\mathbf{M}}%
\renewcommand{\sb}{\mathbf{s}}%
\newcommand{\vb}{\mathbf{q}}%to unify test functions

\newcommand{\alphav}{\underline{\bm{\alpha}}} % for vector-1-forms, like PD-1-form
\newcommand{\betav}{\underline{\bm{\beta}}} % for vector-1-forms, like PD-1-form

\newcommand{\abs}[1]{\lvert#1\rvert}%
\newcommand{\norm}[1]{\lVert#1\rVert}%
\newcommand{\scalarprod}[1]{\big\langle{#1}\big\rangle}%
\newcommand{\Scalarprod}[1]{\left\langle{#1}\right\rangle}%

% derivatives
\newcommand{\dif}{\textup{d}}
\newcommand{\exd}{\mathbf{d}} %exterior derivative
\newcommand{\ds}{\,\dif{s}}
\newcommand{\dr}{\,\dif\rb}
\newcommand{\dx}{\,\dif\xb}
\newcommand{\dxhat}{\,\dif\hat{\xb}}
\newcommand{\dt}{\partial_t}

% differential forms for integration
\newcommand{\dS}{\,\dif{\surf}}
\newcommand{\dV}{\,\dif{V}}

% functional derivative
\newcommand{\fdif}{\operatorname{\delta}\!}
\newcommand{\Fdif}[2]{\frac{\fdif{#1}}{\fdif{#2}}}% dF / du	
\newcommand{\FFdif}[3]{\frac{\fdif^2{#1}}{\fdif{#2}\fdif{#3}}}% d^2F / dudv	

\newcommand{\FF}{\mathrm{F}}
\newcommand{\F}[1]{\FF_\mathrm{#1}}

% surface and domain
\newcommand{\Sp}{\mathbb{S}^2}
\newcommand{\ellipsoid}{\mathcal{E}}
\newcommand{\surf}{\mathcal{S}}
\newcommand{\domain}{\Omega}

\newcommand{\meanCurvature}{\mathcal{H}}
\newcommand{\gaussianCurvature}{\kappa}

% surface (differential operators)
\newcommand{\Grad}{\operatorname{Grad}}
\renewcommand{\div}{\operatorname{div}}%
\newcommand{\Rot}{\operatorname{Rot}}%
\newcommand{\rot}{\operatorname{rot}}%
\newcommand{\DivSurf}{\Div_{\surf}}%
\newcommand{\GradSurf}{\Grad_{\surf}}
\newcommand{\laplace}{\Delta}
\newcommand{\laplaceBeltrami}{\Delta_{\surf}}
\newcommand{\vecLaplace}{\boldsymbol{\Delta}}
%\newcommand{\laplaceDeRahm}{\vecLaplace^{\textup{dR}}_{\surf}}
\newcommand{\laplaceDeRham}{\vecLaplace^{\textup{dR}}}
\newcommand{\laplaceDeRahm}{\laplaceDeRham}
\newcommand{\laplaceRotRot}{\vecLaplace^{\textup{RR}}}
\newcommand{\laplaceGradDiv}{\vecLaplace^{\textup{GD}}}

\newcommand{\laplaceDeRhamTilde}{\widehat{\vecLaplace}^{\textup{dR}}}
\newcommand{\NablaSurf}{\nabla_{\surf}}
\newcommand{\gDerivative}{D}

\newcommand{\laplaceDeRhamDiffuse}{\widehat{\vecLaplace}^{\textup{dR}}_{\phi}}

\newcommand{\surfNormal}{\boldsymbol{\nu}}
\newcommand{\surfNormalI}{\nu}

\newcommand{\ProjectSurf}{\pi_\surf}

\newcommand{\Tangent}{\mathsf{T}}

% General
\newcommand{\vect}[1]{\mathbf{#1}}
\newcommand{\tensor}[1]{\mathbf{#1}}

\newcommand{\Span}[1]{\operatorname{Span}\!\left\{ #1 \right\}}

\newcommand{\AMDIS}{\texttt{AMDiS}}
\newcommand{\PETSC}{\texttt{PETSc}}

% Constants
\newcommand{\K}{{K}} % one-constant
\newcommand{\Ki}{{K_1}} % frank-constant K1
\newcommand{\Kii}{{K_2}} % frank-constant K2
\newcommand{\Kiii}{{K_3}} % frank-constant K3
\newcommand{\Kn}{{\omega_n}} % penalty constant for normality
\newcommand{\Kt}{{\omega_t}} % penalty constant for tangentiality

% Energies
\newcommand{\EF}{E^{\textup{F}}} % 3D Frank-Energy
\newcommand{\EFSurf}{E^{\textup{F}}_{\surf}} % Frank-Energy on surface
\newcommand{\ESurf}{E_{\surf}} % intrinsic Frank-Energy plus normalizing term on surface

\newcommand{\EE}{\F{\omega}^\surf}
\newcommand{\E}[1]{\EE[#1]}


\newcommand{\Ltwo}[2]{L^2( #1;\, #2)}%
\newcommand{\LS}{L^2(\surf)}%
\newcommand{\LtwoProd}[1]{\big({#1}\big)_{\LS}}%

\newcommand{\C}[2]{{C}( #1;\, #2)}
\newcommand{\Csurf}[1]{{C}^{#1}(\surf)}
\newcommand{\Cdomain}[1]{{C}^{#1}(\domain)}
\newcommand{\extend}[1]{\widehat{#1}}

\newcommand{\Hdr}[2]{H^{\textup{DR}}( #1;\, #2)} %
\newcommand{\HdrExt}[2]{H^{\textup{DR}}( #1;\, #2)} % \extend{H}^{\textup{DR}}\left( #1 ; #2 \right)
\newcommand{\HdrDiffuse}[2]{H^{\textup{DR}}( #1;\, #2)} %\extend{H}^{\textup{DR}}_{\phi}\left( #1 ; #2 \right)

\newcommand{\Hi}[1]{H^{1}(#1)}

\newcommand{\pExt}{\extend{\pb}}
\newcommand{\qExt}{\extend{\qb}}
\newcommand{\nExt}{\extend{\surfNormal}}

% rotation angle
\newcommand{\rotAngle}{\psi}

% DEC declarations
\newcommand{\SC}{\mathcal{K}} % simplicial complex
\newcommand{\Vs}{\mathcal{V}} % set of vertices
\newcommand{\Es}{\mathcal{E}} % set of edges
\newcommand{\Fs}{\mathcal{T}} % set of faces
\newcommand{\face}{T} % one face
\newcommand{\FormSpace}{\Lambda^{1}} % space of 1 forms
\newcommand{\PDT}{\mathfrak{T}} % PD-Tangential-Space

% for commends among each other
\newcommand{\simon}[1]{{\color{red}#1}}
\newcommand{\ingo}[1]{{\color{blue}#1}}
\newcommand{\micha}[1]{{\color{green}#1}}

\newcommand{\hidden}[1]{}

%\theoremstyle{plain}
\newtheorem{lem}{Lemma}
\newtheorem{thm}{Theorem}
\newtheorem{prop}[thm]{Proposition}
\newtheorem{rem}{Remark}[section]
\newtheorem{exmp}{Example}[section]

%\theoremstyle{definition}
\newtheorem{defn}{Definition}

\makeatother
\providecommand*{\thmautorefname}{Theorem}
\providecommand*{\remautorefname}{Remark}
\providecommand*{\defnautorefname}{Definition}
\providecommand*{\algorithmautorefname}{Algorithm}
%\addto\extrasenglish{%
  %\renewcommand{\chapterautorefname}{Chapter}%
  %\renewcommand{\sectionautorefname}{Section}%
  %\renewcommand{\subsectionautorefname}{Section}%
  %\renewcommand{\subsubsectionautorefname}{Section}%
  %\renewcommand{\paragraphautorefname}{Section}%
%}
\newcommand*{\End}{\hfill\ensuremath{\vartriangleleft}}%
\newcommand*{\END}{\hfill\ensuremath{\blacktriangleleft}}%




\usepackage{tensor}

\newcommand{\Tr}{\text{Tr}}
\newcommand{\RSpan}[1]{\text{Span}_{\R}\left\{ #1 \right\}}

\newcommand{\surfh}{\surf_{h}}
\newcommand{\landau}{\mathcal{O}}
\newcommand{\adj}{\text{adj}}
\newcommand{\tildeCh}[3]{\widetilde{\Gamma}_{#1 #2}^{#3}} %Chrisoffel-Symb. 2. Art in der Dünnschicht
\newcommand{\Ch}[3]{\Gamma_{#1 #2}^{#3}} %Chrisoffel-Symb. 2. Art auf der Oberfläche


\newcommand{\Qb}{\mathbf{Q}}
\newcommand{\QExt}{\extend{\Qb}}
\newcommand{\QIExt}{\extend{Q}}

\newcommand{\Bb}{\mathbf{B}}


\title{Uniaxialer Ansatz und Taylor-Approximationen zum Q-Tensor-Modell (ohne Beweise)} 
\author{Ingo Nitschke}

\begin{document}
\maketitle

Bezeichner usw. siehe \textit{Notizen zum Q-Tensor-Modell} und \textit{Notizen zur Konvergenz der Frank-Oseen-Energie im Dünnschichtmodell gegen die Oberflächenformulierung}.
Das Prozedere ist analog zum Vorgehen in \textit{Notizen zur Konvergenz der Frank-Oseen-Energie im Dünnschichtmodell gegen die Oberflächenformulierung}.

Wir verfolgen hier den Ansatz\footnote{\( \extend{\mathbf{g}} \equiv \mathbf{I} - \surfNormal \otimes \surfNormal \) im Euklidischen \( \R^{3} \)} (voll covariant)
\begin{align}
  \QExt &= \extend{s}\left( \pExt\otimes\pExt - \frac{1}{2}\extend{\mathbf{g}} \right)
\end{align}
mit parallel-erweiterten und rein tangentialen Director, d.h. \( \tensor{\pIExt}{^{I}_{|\xi}} = 0 \) und \( \pIExt^{\xi} = 0 \).
Dieser Ansatz ist äquivalent zum Ansatz \(  \QIExt_{IJ|\xi} = 0 \) und \( \QIExt_{I\xi} = 0 \).

Taylor-Approximation liefert (voll covariant)
\begin{align}
  \QExt = \Qb - \xi \left( \Bb\Qb + \Qb\Bb \right) + \landau(\xi^{2})
\end{align}
mit Oberflächen Q-Tensor \( \Qb \) und Shape-Operator \( \Bb \).
Somit ergeben sich die Spuren der Q-Tensor-Potenzen zu
\begin{align}
  \Tr\QExt = \Tr\QExt^{3} &= 0\\
  \Tr\QExt^{2} &= \Tr\Qb^{2} + \landau(\xi) = \left\| \Qb \right\|^{2} + \landau(\xi) \\
  \Tr\QExt^{4} &= \frac{1}{2}\left( \Tr\QExt^{2} \right)^{2} = \Tr\Qb^{4} + \landau(\xi) = \frac{1}{2}\left\| \Qb \right\|^{4} + \landau(\xi)
\end{align}
und die Kontraktionen der Ableitung zu
\begin{align}
  \left\| \nabla\QExt \right\|^{2} &= \left\| \nabla\Qb \right\|^{2} + 2\left\| \Bb\Qb \right\|^{2} + \landau(\xi) \\
                                   &= 2\left( \left\| \div\Qb \right\|^{2} + \gaussianCurvature\left\| \Qb \right\|^{2} + \left\| \Bb\Qb \right\|^{2}\right) + \landau(\xi) \\
  \left\| \div\QExt \right\|^{2} &= \left\| \div\Qb \right\|^{2} + \left( \Tr(\Bb\Qb) \right)^{2} + \landau(\xi) \formPeriod
\end{align}
Folgen wir den Energiedichten in \cite{Emmerich2012}
mit \( \psi = 1 \),
dann ergibt sich (103) zu
\begin{align}
  f_{\textup{id}} &= \frac{17}{3} + \frac{4}{15} \Tr\QExt^{2} + \frac{8}{315}\Tr\QExt^{4} \\
                  &= \frac{17}{3} + \frac{4}{15} \Tr\Qb^{2} + \frac{8}{315}\Tr\Qb^{4} + \landau(\xi) \\
                  &= \frac{17}{3} + \frac{4}{15}\left( \left\| \Qb \right\|^{2} + \frac{1}{21}\left\| \Qb \right\|^{4} \right) + \landau(\xi)
\end{align}
und (105) zu (\(\tilde{K}_{2}\) partiell integriert)
\begin{align}
  f_{\textup{exc}}^{(2)} &= A_{1} + B_{1}\Tr\QExt^{2} + \tilde{K}_{1}\left\| \div\QExt \right\|^{2} - \tilde{K}_{2}\left\| \nabla\QExt \right\|^{2} \\
                         &= A_{1} + B_{1}\left\| \Qb \right\|^{2} + \left( \tilde{K}_{1} - 2\tilde{K}_{2} \right)\left\| \div\Qb \right\|^{2}
                              + \tilde{K}_{1}\left( \Tr(\Bb\Qb) \right)^{2} - 2\tilde{K}_{2}\left( \gaussianCurvature\left\| \Qb \right\|^{2} + \left\| \Bb\Qb \right\|^{2} \right)
                              +\landau(\xi)
\end{align}

Folgen wir der Energiedichte in \cite{Mucci2012} (1.9), dann erhalten wir
\begin{align}
  f &= \frac{L}{2}\left\| \nabla\QExt \right\|^{2} - \frac{a}{2}\Tr\QExt^{2}  -\frac{b}{3}\Tr\QExt^{3} + \frac{c}{2}\Tr\QExt^{4} \\
    &= L \left( \left\| \div\Qb \right\|^{2} + \gaussianCurvature\left\| \Qb \right\|^{2} + \left\| \Bb\Qb \right\|^{2}\right) 
       -\frac{a}{2} \left\| \Qb \right\|^{2}  + \frac{c}{4}\left\| \Qb \right\|^{4 } + \landau(\xi)
\end{align}



\bibliography{bibl}
\bibliographystyle{alpha}

\end{document}
